%%% Šablona pro jednoduchý soubor formátu PDF/A, jako treba samostatný abstrakt práce.

\documentclass[12pt]{report}

\usepackage[a4paper, hmargin=1in, vmargin=1in]{geometry}
\usepackage[a-2u]{pdfx}
\usepackage[czech]{babel}
\usepackage[utf8]{inputenc}
\usepackage[T1]{fontenc}
\usepackage{lmodern}
\usepackage{textcomp}

\begin{document}

Extrakce melodie patří mezi nejdůležitější a nejtěžší úlohy oboru Music Information Retrieval, právě melodie je totiž tím hlavním, co si člověk po poslechu skladby odnáší a z podstaty se tedy často jedná o její nejvýraznější rys. Přítomnost hudebního doprovodu, který melodii podbarvuje, však pro algoritmické metody znemožňuje její průběh spolehlivě zachytit. V posledních letech se proto obor posouvá směrem k využívání metod hlubokého učení, které jsou schopny dřívější pravidlové systémy překonat. Na tyto práce navazujeme, představujeme tři nové metody a experimentálně ověřujeme volby, které jsme při jejich návrhu učinili. Ukazujeme, že nová architektura \emph{Harmonic Convolutional Neural Network}, založená na úpravě vnitřního uspořádání obvyklé konvoluční sítě, díky které je schopna lépe zachytit harmonickou povahu jednotlivých tónů ze vstupních spektrogramů s logaritmickou osou frekvence, překonává state-of-the-art metody pro extrakci melodie na většině veřejně dostupných datasetech.

\end{document}
