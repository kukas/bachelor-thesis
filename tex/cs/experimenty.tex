\chapter{Experimenty}

Práce obsahuje souhrnné výsledky experimentů zejména nad datasetem MedleyDB, aby modely byly dobře porovnatelné se state-of-the-art výsledky a výhoda prezentovaných metod netkvěla pouze v použití více dat. U vybraných experimentů došlo k přetrénování na větší trénovací množině, aby bylo možné posoudit vliv množství dat na výsledný výkon. 

V první části se zabývám zejména odhadováním *výšky tónů*. U úspěšných architektur pak implementuji i *detekci melodie*.

\section{Architektura CREPE}

První sada experimentů se zakládá na architektuře popsané v článku od \cite{Kim2018} použité pro *monopitch tracking*. Přestože se nejedná o úlohu extrakce melodie, cílem monopitch trackingu je určit konturu základní frekvence melodického nástroje v monofonní nahrávce, která se skládá ze součtu čistého signálu a šumu v pozadí. Pokud rozšíříme pojem šumu v pozadí tak, aby zahrnoval i melodický doprovod, pak dostáváme formální definici signálu zpracovávaného algoritmy pro přepis melodie \cite{Salamon2014}.

Jinými slovy - *monopitch tracking* je speciálním případem extrakce melodie a tudíž přinejmenším stojí za zkoušku pokusit se tuto architekturu pro extrakci využít. Mimo to monofonní stopy často obsahují přeslech ostatních nástrojů, pokud nahrávka vznikala při společném hraní, tudíž by model trénovaný na výsledných mixech mohl být robustní vůči tomuto druhu rušení. 

Architektura CREPE se sestává ze šesti konvolučních a pooling vrstev, pro regularizaci používá batch normalization a dropout po každé konvoluční vrstvě, jako aktivační funkce používá ReLU. Po konvolucích následuje výstupní plně propojená vrstva se sigmoid aktivací. Vstupem modelu je okno o velikosti 1024 samplů, audio je převzorkováno na 16 kHz. Před první konvolucí je vstup normalizován tak, aby každé jednotlivé okno se vzorky mělo střední hodnotu 0 a směrodatnou odchylku 1. Přesná podoba modelu je naznačena na obrázku.

Výsledný vektor o 640 složkách aproximuje pravděpodobnostní rozdělení výšky základní frekvence uprostřed vstupního okna, přičemž tento vektor pokrývá rozsah od noty $C_{-1}$ po $G_{9}$, mezi dvěma sousedními predikovanými tóny je vzdálenost 20 centů. Výšky tónů v centech označíme $\cent_1, \cent_2, \dots, \cent_{640}$. Rozsah tedy bezpečně pokrývá obvyklé hudební nástroje a na jednu notu připadá 5 složek (tónů) výsledného vektoru.

    $$\cent(f) = 1200 \log_2{\frac{f}{f_{\mathrm{ref}}}}$$

Pro trénování modelu potřebujeme také cílové diskrétní pravděpodobnostní rozdělení základní frekvence tónu. Jako cílovou pravděpodobnostní funkci použijeme normální rozdělení se střední hodnotou v bodě cílové základní frekvence $\cent(f_{\mathrm{ref}})$ a se směrodatnou odchylkou 25 centů. Toto rozdělení dikretizujeme, aby měl cílový vektor stejné dimenze jako odhadovaný.

    $$y_i = \frac{1}{\sqrt{2 \pi \sigma^2}}\exp{(-\frac{(\cent_i - \cent_{\mathrm{ref}})^2}{2 \sigma^2})}$$

Převod z výstupního vektoru na výšky not provedeme pomocí střední hodnoty výstupního vektoru. Jelikož by ale výšku tónu ovlivňoval i další melodický šum, který se na výstupním vektoru také objevuje, spočítáme střední hodnotu pouze z okolí maxima výstupu.

    $$ \left. \hat{\cent} = \sum_{\scaleto{i, \lvert \cent_i - \cent_m \rvert < 50}{8pt}} {\hat{y}_i \cent_i} \middle/ \sum_{\scaleto{i, \lvert \cent_i - \cent_m \rvert < 50}{8pt}} \hat{y}_i \right., m = \mathrm{argmax}_i(\hat{y}_i)$$

Optimalizovaná loss funkce modelu $\mathcal{L}(\mathbf{y}, \mathbf{\hat{y}})$ se počítá jako binární vzájemná korelace mezi vektorem cílových pravděpodobností $y$ a výstupním vektorem $\hat{y}$.

    $$\mathcal{L}(\mathbf{y}, \mathbf{\hat{y}}) = \sum_{i = 1}^{640}{(-y_i\log\hat{y}_i - (1-y_i)\log(1-\hat{y_i}))}$$

Optimalizace probíhá pomocí algoritmu Adam \citep{Kingma2014} s learning rate 0.0002.

% TODO: Přidat obrázek modelu (draw.io)


% rozepsat:
% - obhajoba raw signálu

% diskuze:
% - převzorkování na 16kHz
% - normalizace vstupu
% - formulace jako klasifikační úloha, nikoli regresní
% - je lepší odhadovat opravdové pravděpodobnostní rozdělení a nebo jejich škálované? (přijde mi, že kvůli sigmoid aktivaci bude jednodušší 1.0 = Truth, protože ty vstupní logity do sigmoid aktivace můžou být crazyshit velký)
% - crepe model - např. nedává vůbec smysl velikost kernelu 64 v posledních vrstvách, zbytečně se tam přidávají nuly jako padding


\subsection{Replikace výsledků CREPE}


Pro ověření správnosti implementace architektury *monopitch trackeru CREPE* spustíme model na syntetických, monofonních datech používaných v článku \cite{Salamon2017}. Na rozdíl od článku \cite{Kim2018} jsem model netrénoval na všech datech pomocí postupu *5 fold cross validation*, jiné zásadní rozdíly mezi implementacemi jsem však na základě článku a veřejně dostupného kódu neidentifikoval.

Po jedné epoše trénování model dosáhl vyšší přesnosti, než je uváděná v literatuře, tento rozdíl přičítám zejména zmiňované odlišné evaluační strategii.

    \begin{tabular}{llrr}
    \toprule
    Metrika & Práh & Průměrná hodnota & Hodnota \cite{Kim2018} \\
    \midrule
    RCA & 50 centů & 0.988 & 0.970 \\
    RPA  & 50 centů & 0.986 & 0.967 \\
    RPA  & 25 centů & 0.975 & 0.953 \\
    RPA  & 10 centů & 0.937 & 0.909 \\
    \bottomrule
    \end{tabular}

Při replikaci experimentu jsem narazil na důležitost správného promíchání dat. Framework Tensorflow použitý pro trénování promíchává data vždy pomocí bufferu pevné velikosti pro dvojice vstupů a cílových výstupů. V praxi je však potřeba buď nastavit buffer na velikost větší než je celková velikost datasetu, a nebo implementovat vlastní míchání přes všechna dostupná data. Při nedostatečně promíchaných datech totiž trénovací dávky (batch) nejsou reprezentativní pro celý dataset, ale pouze pro jeho podmnožinu, což se negativně projevuje kolísající validační přesností modelu.

\subsection{CREPE pro extrakci melodie}

Jako první experiment nad melodickými daty spustíme nezměněnou architekturu CREPE, v následujících experimentech se tuto baseline pokusíme překonat. Abychom urychlili trénování následujících experimentů, přesnost určíme pro sítě s různou kapacitou, pokud se výsledky při různých kapacitách příliš neliší, můžeme experimenty provádět s architekturou s nižší kapacitou. Kapacity upravíme pomocí multiplikátoru počtu filtrů u všech konvolučních vrstev, počty filtrů jsou uvedeny v tabulce.


    \begin{tabular}{lrrrrrrr}
    \toprule
    Vrstva     &  1.   &  2.  &  3.  &  4.  &  5.   &  6.  &  Celkový počet parametrů \\
    \midrule
    CREPE 4x   &  128  &  16  &  16  &  16  &  32   &  64  &  558240 \\
    CREPE 8x   &  256  &  32  &  32  &  32  &  64   &  128 &  1771200 \\
    CREPE 16x  &  512  &  64  &  64  &  64  &  128  &  256 &  6163200 \\
    \bottomrule
    \end{tabular}


    \begin{tabular}{lrr}
    \toprule
    Model      &  RPA    &  RCA \\
    \midrule
    CREPE 4x   &  0.634  &  0.753 \\
    CREPE 8x   &  0.661  &  0.766 \\
    CREPE 16x  &  0.666  &  0.771 \\
    Salamon    &  0.547  &  0.608 \\
    Bittner    &  0.735  &  0.791 \\
    Basaran    &  0.737  &  0.803 \\
    \bottomrule
    \end{tabular}

Z výsledků na validačních datech po 200k iteracích (přibližně 6 epoch) je zřejmé, že překonání state-of-the-art metod založených na pravidlovém zpracování zvuku \citep{salamon2012musical} není obtížné. Zároveň také vidíme, že se výsledek modelů CREPE 8x a CREPE 16x liší řádově o desetiny procentních bodů a přitom model s větší kapacitou se trénuje o 35% delší dobu. Proto pro další experimenty zvolíme architektury s multiplikátorem 8x a případně přetrénujeme s vyšší kapacitou pouze nadějné konfigurace.


\subsection{Vliv rozlišení diskretizace výšky noty}

Otestujeme nastavení granularity výstupního vektoru. V článku \cite{Kim2018} se totiž důvod volby pěti frekvencí na notu nediskutuje. Intuitivně by však mělo vyšší rozlišení spíše pomáhat, důvodem je, že nástroje a zejména lidský hlas se často při hraní odchylují od přesných frekvencí hraných not a vyšší rozlišení tyto odchylky může lépe zachytit.

    \begin{tabular}{llrr}
    \toprule
    Kapacita & Diskretizace &  RPA &  RCA \\
    \midrule
     4x &        hrubá      & 0.606 & 0.708 \\
     4x &        jemná      & 0.634 & 0.753 \\
     8x &        hrubá      & 0.614 & 0.724 \\
     8x &        jemná      & 0.661 & 0.766 \\
    16x &        hrubá      & 0.612 & 0.711 \\
    16x &        jemná      & 0.666 & 0.771 \\
    \bottomrule
    \end{tabular}

% TODO: Přidat graf

Jak je vidět z tabulky a grafů, jemná granularita výstupu jednoznačně zlepšuje přesnost sítě. Abychom potvrdili hypotézu, že vyšší rozlišení pomáhá zmenšit počet chyb o půltón, můžeme vytvořit histogram vzdáleností cílového a odhadovaného tónu, v tomto histogramu by pak měl být vidět pokles v příslušných třídách.

% TODO: Přidat histogramy

Podle histogramu se počet chyb o půltón mezi zkoumanými modely liší téměř o polovinu, zlepšení tohoto druhu chyb je tedy podstatné.

\subsection{Vliv rozptylu cílové pravděpodobnostní distribuce výšky noty}

Podle \cite{Bittner2017} pomáhá cílová distribuce s vyšším rozptylem snížit penalizaci sítě za téměř korektní odhady výšek tónů. Mimo to u dostupných dat často nejsou anotace naprosto perfektní, jisté rozostření hranice anotace tudíž pomáhá i v případě nepřesné cílové anotace, síť pak není tolik penalizována za svou případnou správnou odpověď. 

V článku se však nediskutuje nastavení směrodatné odchylky na 20 centů, \cite{Kim2018} používá odchylku 25 centů a není na první pohled zřejmé, jaká je optimální hodnota. Příliš vysoký rozptyl způsobí, že síť bude tolerovat více chyb o půltón, příliš nízký rozptyl naopak penalizuje i téměř správné odhady. Intuitivně se nejlepší nastavení pravděpodobně bude pohybovat kolem používaných 25 centů, jelikož to je hranice chybné klasifikace, na druhou stranu optimální hodnota jistě bude závislá na nastavení rozlišení výstupního vektoru, jelikož nižší rozlišení bude jistě vyžadovat vyšší hodnotu rozptylu (v extrémním případě rozptylu blížícího se k nule a cílové frekvence mimo kvantizační hladiny by vzniklý cílový vektor nemusel obsahovat žádné ostré maximum).

Poznamenám také technický detail, který je důležitý při samotné implementaci. Přestože jsem cílový výstup sítě zadefinoval jako diskrétní pravděpodobnostní rozdělení, při trénování je tento vektor hodnot pronásoben koeficientem tak, aby $\max(\mathbf{y}) = 1.0$ a tedy součet prvků vektoru není roven jedné (a o pravděpodobnostní rozdělení se doopravdy nejedná). Důvodem je použití aktivační funkce *sigmoid* u výstupní vrstvy, která nezaručuje výstup korektního rozdělení. Díky tomu se na výstupu může objevit různé množství stejně pravděpodobných kandidátů na melodii.

Testovaná síť má vstupní okno široké 4096 vzorků, používá multiplikátor kapacity 16x a vstup zpracovává 6 různě širokými konvolučními vrstvami (viz experiment *Vliv násobného rozlišení první konvoluční vrstvy*).

    \begin{tabular}{lrr}
    \toprule
    Směrod. &  Raw Pitch Accuracy &  Raw Chroma Accuracy \\
    \midrule
    0.000   &               0.657 &                0.759 \\
    0.088   &               0.672 &                0.775 \\
    0.177   &               0.689 &                0.784 \\
    0.354   &               0.669 &                0.773 \\
    0.707   &               0.654 &                0.757 \\
    \bottomrule
    \end{tabular}

Z experimentů vyplývá, že optimální směrodatná odchylka se pohybuje kolem hodnoty $0.177$, tedy níže než v porovnávaných pracích. 

% ------
% - cílová distribuce doopravdy není distribuce
% - ty zvláštní testované směrod. odchylky jsou kvůli mé chybné implementaci rozostřování
% - zde můžu přidat obrázek, jak vypadají anotace
%     mám to rozpracované na: http://jirkabalhar.cz:6088/notebooks/bakalarka/algoritmy/ismir2017-deepsalience/deepsalience/out/io_comparison.ipynb#

\subsection{Vliv šířky vstupního okna}

Architektura CREPE byla navržena pro monopitch tracking, dá se předpokládat, že jelikož je v monofonních nahrávkách oproti polyfonním daleko méně (melodického) šumu, není pro určení výšky tónu potřeba větší kontext než použitých 1024 vzorků (při vzorkovací frekvenci 16kHz toto odpovídá 64 milisekundám audia). To ale nemusí platit pro složitější signály, kde by síť mohla z delšího kontextu těžit. Otestujeme tedy vliv většího vstupního okna na výslednou přesnost.

    \begin{tabular}{lrr}
    \toprule
    Šířka vstupního okna &  Raw Pitch Accuracy &  Raw Chroma Accuracy \\
    \midrule
    512 (32 ms)          &               0.634 &                0.748 \\
    1024 (64 ms)         &               0.645 &                0.763 \\
    2048 (128 ms)        &               0.648 &                0.760 \\
    4096 (256 ms)        &               0.650 &                0.762 \\
    8192 (512 ms)        &               0.675 &                0.775 \\
    \bottomrule
    \end{tabular}

% ------

% TODO: možná by to chtělo taky přetrénovat

% - širší okno se také hodí pro onsety a offsety

\subsection{Vliv násobného rozlišení první konvoluční vrstvy}

Podle \cite{Kim2018} se přesnost CREPE snižuje s výškou tónu. Autoři si tuto skutečnost vysvětlují neschopností modelu generalizovat na barvy a výšky tónů neobsažených v trénovací množině, generalizaci by ale mohla pomoci také úprava modelu. Protože k rozpoznání vyšších frekvencí stačí méně vzorků než pro rozpoznání nižších, mohli bychom se pokusit upravit první konvoluční vrstvu sítě, která tento úkol zastává, a rozdělit ji na množiny různě širokých konvolucí, jejichž kanály následně sloučíme zpět do jednotné vrstvy. To by mělo mít za následek, že rozpoznávání vysokých tónů budou zastávat užší konvoluce a jejich kernel bude jednodušší než široké kernely s vysokou mírou redundance.

První vrstvu s kernelem s 256 filtry (tj. počet filtrů první vrstvy s multiplikátorem 8x, viz první experiment) jsem rozdělil na vícero různě širokých kernelů s menším počtem filtrů, tak aby kapacita sítě zůstala přibližně stejná a sítě byly porovnatelné. 


    \begin{tabular}{lrrrrrrrrr}
    \toprule
    Počet/šířka kernelů & 512 & 256 & 128 & 64 & 32 & 16 & 8  & 4  & Celkový počet parametrů  \\
    \midrule
    1                   & 256 &     &     &    &    &    &    &    & 2098880 \\
    2                   & 128 & 128 &     &    &    &    &    &    & 2066112 \\
    3                   & 85  & 85  & 85  &    &    &    &    &    & 2041918 \\
    4                   & 64  & 64  & 64  & 64 &    &    &    &    & 2029248 \\
    5                   & 51  & 51  & 51  & 51 & 51 &    &    &    & 2016350 \\
    6                   & 42  & 42  & 42  & 42 & 42 & 42 &    &    & 2001944 \\
    7                   & 36  & 36  & 36  & 36 & 36 & 36 & 36 &    & 1996184 \\
    8                   & 32  & 32  & 32  & 32 & 32 & 32 & 32 & 32 & 2000448 \\
    \bottomrule
    \end{tabular}

Experiment jsem provedl na síti se vstupním oknem 978 vzorků, multiplikátorem kapacity 8, 

    \begin{tabular}{lrr}
    \toprule
    Počet konvolučních vrstev &  Raw Pitch Accuracy &  Raw Chroma Accuracy \\
    \midrule
    1                         &               0.629 &                0.734 \\
    2                         &               0.628 &                0.732 \\
    3                         &               0.632 &                0.734 \\
    4                         &               0.636 &                0.739 \\
    5                         &               0.643 &                0.740 \\
    6                         &               0.638 &                0.737 \\
    7                         &               0.636 &                0.736 \\
    8                         &               0.640 &                0.737 \\
    \bottomrule
    \end{tabular}

Zlepšení výsledků se pohybuje v řádu desetin procentních bodů, tedy není příliš vysoké. Zlepšení je nejvíce patrné v případě pěti různě širokých konvolučních vrstev, kde dosahuje $1.3$ procentního bodu. Analýzou výsledků přesnosti podle výšky noty se mi nepodařilo prokázat hypotézu, že by konvoluce s více rozlišeními pomáhala u odhadu not vyšších frekvencí. Její přínos je drobný a projevuje se na většině frekvenčních pásem.


\section{Wavenet}

Generativní model WaveNet popsaný týmem \cite{Oord2016} je architektura navržená pro generování zvukového signálu, autoři však síť testovali i pro převod mluvené řeči na text (dataset TIMIT) a dosáhli výsledků srovnatelných se state-of-the-art. Síť se však pro *Music Information Retrieval* od svého zveřejnění příliš neuchytila. Její použití se v oblasti hudby se omezuje na generativní úlohy (\cite{Hawthorne2018a}, \cite{Yang2017}, \cite{Engel2017} a další), případně *source-separation* \citep{Stoller2018}. Jediný publikovaný pokus s použitím architektury WaveNet pro automatický přepis podnikli \cite{Martak2018} nad datasetem MusicNet. Jejich model však netestovali na standardních evaluačních datasetech ze soutěže MIREX, tudíž není zřejmé, jakých výsledků v porovnání s existujícími metodami autoři dosáhli.

Architektura spočívá v důmyslném vrstvení dilatovaných konvolucí. Díky exponenciálně rostoucím dilatacím se také exponenciálně zvětšuje receptivní pole jednotlivých konvolučních vrstev. Díky této vlastnosti pak například stačí pro pokrytí 1024 vzorků vstupu pouze 9 vrstev s šířkou kernelu 2 a dilatacemi 1,2,4,8 ... 512. Pokud bychom stejného receptivního pole chtěli dosáhnout pomocí obvyklých konvolucí počet potřebných vrstev by byl lineární vzhledem k šířce pole. Vrstvení konvolucí je porovnáno na schématu. 

% TODO: přidat schéma konvolucí

\subsection{Baseline na základě \cite{Martak2018}}

Pro srovnání spustíme architekturu popsanou ve zmíněném článku pro úlohu extrakce melodie. Jelikož byla architektura zamýšlena pro dataset MusicNet, který obsahuje celý přepis skladeb do MIDI not, výstupem jsou diskrétní noty. Jak jsme zjistili v předchozím experimentu na architektuře CREPE, hrubá diskretizace výrazně zhoršuje přesnost výsledků, upravíme proto architekturu tak, aby měla výstupní distribuce jemnější rozlišení.
