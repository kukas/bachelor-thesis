%%% Hlavní soubor. Zde se definují základní parametry a odkazuje se na ostatní části. %%%

%% Verze pro jednostranný tisk:
% Okraje: levý 40mm, pravý 25mm, horní a dolní 25mm
% (ale pozor, LaTeX si sám přidává 1in)
\documentclass[12pt,a4paper]{report}
\setlength\textwidth{145mm}
\setlength\textheight{247mm}
\setlength\oddsidemargin{15mm}
\setlength\evensidemargin{15mm}
\setlength\topmargin{0mm}
\setlength\headsep{0mm}
\setlength\headheight{0mm}
% \openright zařídí, aby následující text začínal na pravé straně knihy
\let\openright=\clearpage

%% Pokud tiskneme oboustranně:
% \documentclass[12pt,a4paper,twoside,openright]{report}
% \setlength\textwidth{145mm}
% \setlength\textheight{247mm}
% \setlength\oddsidemargin{14.2mm}
% \setlength\evensidemargin{0mm}
% \setlength\topmargin{0mm}
% \setlength\headsep{0mm}
% \setlength\headheight{0mm}
% \let\openright=\cleardoublepage
\usepackage[rgb,hyperref,usenames,table]{xcolor}  % barevná sazba

%% Vytváříme PDF/A-2u
\usepackage[a-2u]{pdfx}

%% Přepneme na českou sazbu a fonty Latin Modern
\usepackage[czech]{babel}
\usepackage{lmodern}
\usepackage[T1]{fontenc}
\usepackage{textcomp}

%% Použité kódování znaků: obvykle latin2, cp1250 nebo utf8:
\usepackage[utf8]{inputenc}

%%% Další užitečné balíčky (jsou součástí běžných distribucí LaTeXu)
\usepackage{amsmath}        % rozšíření pro sazbu matematiky
\usepackage{amsfonts}       % matematické fonty
\usepackage{amsthm}         % sazba vět, definic apod.
\usepackage{bbding}         % balíček s nejrůznějšími symboly
			    % (čtverečky, hvězdičky, tužtičky, nůžtičky, ...)
\usepackage{bm}             % tučné symboly (příkaz \bm)
\usepackage{graphicx}       % vkládání obrázků
\usepackage{fancyvrb}       % vylepšené prostředí pro strojové písmo
\usepackage{indentfirst}    % zavede odsazení 1. odstavce kapitoly
\usepackage{natbib}         % zajištuje možnost odkazovat na literaturu
			    % stylem AUTOR (ROK), resp. AUTOR [ČÍSLO]
\usepackage[nottoc]{tocbibind} % zajistí přidání seznamu literatury,
                            % obrázků a tabulek do obsahu
\usepackage{icomma}         % inteligetní čárka v matematickém módu
\usepackage{dcolumn}        % lepší zarovnání sloupců v tabulkách
\usepackage{booktabs}       % lepší vodorovné linky v tabulkách
\usepackage{paralist}       % lepší enumerate a itemize

\usepackage{scalerel}
\usepackage{ wasysym }

\usepackage[graphicx]{realboxes}
\usepackage{tablefootnote}

%%% Údaje o práci

% Název práce v jazyce práce (přesně podle zadání)
\def\NazevPrace{Extrakce melodie pomocí hlubokého učení}

% Název práce v angličtině
\def\NazevPraceEN{Melody Extraction using Deep Learning}

% Jméno autora
\def\AutorPrace{Jiří Balhar}

% Rok odevzdání
\def\RokOdevzdani{2019}

% Název katedry nebo ústavu, kde byla práce oficiálně zadána
% (dle Organizační struktury MFF UK, případně plný název pracoviště mimo MFF)
\def\Katedra{Ústav formální a aplikované lingvistiky}
\def\KatedraEN{Institute of Formal and Applied Linguistics}

% Jedná se o katedru (department) nebo o ústav (institute)?
\def\TypPracoviste{Ústav}
\def\TypPracovisteEN{Institute}

% Vedoucí práce: Jméno a příjmení s~tituly
\def\Vedouci{Mgr. Jan Hajič \textcolor{red}{Ph.D.??}}

% Pracoviště vedoucího (opět dle Organizační struktury MFF)
\def\KatedraVedouciho{ústav}
\def\KatedraVedoucihoEN{institute}

% Studijní program a obor
\def\StudijniProgram{Informatika}
\def\StudijniObor{Programování a softwarové systémy}

% Nepovinné poděkování (vedoucímu práce, konzultantovi, tomu, kdo
% zapůjčil software, literaturu apod.)
\def\Podekovani{%
Poděkování.

.
}

% Abstrakt (doporučený rozsah cca 80-200 slov; nejedná se o zadání práce)
\def\Abstrakt{%
Extrakce melodie patří mezi nejdůležitější a nejtěžší úlohy oboru Music Information Retrieval, jejíž šíře uplatnění zahrnuje zlepšení metod pro vyhledávání v hudebních datech (například pomocí zabroukání části skladby), automatický přepis improvizovaných sólových pasáží nebo zpracování zvukových nahrávek pro muzikologické studie. Právě melodie je totiž tím hlavním, co si člověk po poslechu skladby odnáší a zpětně vybavuje a z podstaty se tedy často jedná o její nejvýraznější rys. Přítomnost hudebního doprovodu, který melodii podbarvuje a dává možnost jí vyniknout, však pro dosavadní rigidní algoritmické metody znemožňuje její průběh spolehlivě zachytit. V posledních letech se proto obor posouvá směrem k využívání metod hlubokého učení, díky kterému je možné napodobit intuitivní schopnost člověka rozeznat melodii v nahrávce. Na tyto práce navazujeme a představujeme nové metody, zejména pak architekturu \emph{Harmonic Convolutional Neural Network}, založenou na úpravě vnitřního uspořádání obvyklých konvolučních sítí, pro snažší zachycení harmonické povahy jednotlivých tónů, která překonává poslední state-of-the-art metody na většině veřejně dostupných datasetech.


}
\def\AbstraktEN{%

\textcolor{red}{koukám, že šablona mi tu pasivně agresivně naznačuje, že abstrakt mám moc dlouhý, zítra to ještě seškrtám.}

}

% 3 až 5 klíčových slov (doporučeno), každé uzavřeno ve složených závorkách
\def\KlicovaSlova{%
{Music Information Retrieval} {Extrakce Melodie} {Odhad F0} {Hluboké učení} {Harmonická konvoluční neuronová síť}
}
\def\KlicovaSlovaEN{%
{Music Information Retrieval} {Melody Extraction} {F0 estimation} {Deep Learning} {Harmonic Convolutional Neural Network}
}

%% Balíček hyperref, kterým jdou vyrábět klikací odkazy v PDF,
%% ale hlavně ho používáme k uložení metadat do PDF (včetně obsahu).
%% Většinu nastavítek přednastaví balíček pdfx.
\hypersetup{unicode}
\hypersetup{breaklinks=true}
% \hypersetup{
%     colorlinks,
%     linkcolor={red!50!black},
%     citecolor={blue!50!black},
%     urlcolor={blue!100!black}
% }

%% Definice různých užitečných maker (viz popis uvnitř souboru)
\include{makra}

%% Titulní strana a různé povinné informační strany
\begin{document}
\include{titulka}

%%% Strana s automaticky generovaným obsahem bakalářské práce

\tableofcontents

%%% Jednotlivé kapitoly práce jsou pro přehlednost uloženy v samostatných souborech
\chapter{Úvod}\label{chap:uvod}
\addcontentsline{toc}{chapter}{Úvod}

Spolu s harmonií a rytmem představuje melodie základní stavební kámen většiny existující hudby. V průběhu vývoje od folklórních zpěvů přes orchestrální skladby po soudobou elektroniku si melodie téměř vždy zachovávala své dominantní postavení nositele esence jednotlivých písní. Melodie je to hlavní, co si člověk po poslechu skladby odnáší a nejsnadněji vybaví, a její důležitost je zejména v našem kulturním kontextu natolik jednoznačná, že je občas těžké si bez ní vůbec hudbu představit. 

Tato práce se zabývá metodami odhadu fundamentální frekvence melodie ze zvukové nahrávky. Jinými slovy chceme získat v každém bodě vstupní skladby informaci o tom, zda melodie zní a její případnou výšku. Jde o jednu z nejdůležitějších a zároveň nejtěžších úloh z oboru \textit{Music Information Retrieval}, jejíž rozsah využití v této doméně pokrývá významnou část aktivně řešených, otevřených problémů. Spolehlivý přepis melodie by usnadnil vyhledávání v hudebních datech, ať už na základě notového zápisu (\textit{Symbolic Melodic Similarity}), pomocí nekvalitní nahrávky z rádia (\textit{Audio Fingerprinting}), pomocí broukání (\textit{Query by Singing/Humming}) nebo dokonce pomocí coveru hledané písně (\textit{Audio Cover Song Identification}). Mimo vyhledávání by byl algoritmus užitečný pro další zpracování zvukového signálu, ať už pro manipulaci a úpravu melodického hlasu (například software Melodyne), nebo naopak jeho odstranění a vytvoření karaoke doprovodu (\textit{Informed Source Separation}). V neposlední řadě by extrakce melodie pomohla při kategorizaci hudebních dat, například podle žánru (\textit{Genre Classification}, \cite{Salamon2012}) nebo podle zpěváka (\textit{Singer Characterization}). A konečně široké spektrum využití by nalezla i v muzikologii (případně etnomuzikologii) pro kvantitativní i kvalitativní studii hudebních motivů a postupů (V jazzu například \cite{Pfleiderer}).

\begin{figure}[h]\centering
\includegraphics[width=\textwidth,height=\textheight,keepaspectratio]{../img/input_output}
\caption{Příklad vstupu a výstupu metody pro extrakci melodie. \textcolor{red}{přidat zvukový přílkad do přílohy}}
\label{obr:input_output}
\end{figure}

Extrakce melodie však nemusí sloužit pouze jako mezikrok pro řešení jiné úlohy, užitečný je i samotný výstup algoritmu, znázorněný na obrázku \ref{obr:input_output}. Motivačním příkladem použití může být pomoc při transkripci. Představíme-li si začínajícího hráče na saxofon, který si chce do not přepsat své oblíbené jazzové sólo, výstup algoritmu mu dá užitečnou informaci o tom, jaký tón zní v jakou chvíli. Z této reprezentace už pak hráči zbývá nalezené tóny projít a zapsat je do notové osnovy.

Proč je ale extrakce melodie otevřený problém? Příbuzná úloha, která spočívá v přepisu nahrávky jednoho izolovaného nástroje, je v podstatě vyřešena \citep{Mauch2014a}, proč se tato úloha po přidání hudebního doprovodu stává výrazně obtížnější? Pro vysvětlení dvou zásadních obtíží, které se s přepisem nahrávky pojí, musíme nejdříve přiblížit vůbec povahu zvuku a možnosti jeho zkoumání.


Naše zkušenost se zvukem probíhá primárně skrze sluch. Teprve na hlasitém koncertu však člověk pocítí, že zvuk je ve své fyzikální podstatě změna tlaku vzduchu, putující od zdroje k posluchači. Díky sluchu z těchto vibrací dokážeme oddělit jednotlivé zdroje a identifikovat v nich i velmi jemné rozdíly. Ačkoli jde o subjektivní vjemy, zvuky lze částečně rozřadit podle toho, jak snadno v nich rozeznáme nějakou konkrétní výšku. 

\vspace*{0.5cm}

Čtenář této práce si nyní může postupně vybavit: hrající violoncello, odbíjení kostelního zvonu, cinknutí příboru, štěkot psa, plynutí potoka, šelest listí stromů, trhání papíru, tlesknutí a prasknutí balónku.

\vspace*{0.5cm}

Se ztrácející se zřetelností výšky nejprve přijdeme o možnost zpívat společně se zdrojem zvuku v harmonii a posléze i o možnost si představit \uv{vyšší} a \uv{nižší} instance toho samého zvuku (jak zní vysoké a nízké prasknutí balónku?). To, co mají první z uvedených příkladů společné, je výrazná a stabilní periodicita jejich signálu --- daný tlakový průběh se opakuje v čase. Díky sluchu tuto periodicitu interpretujeme jako výšku, přičemž různé výšky se od sebe liší frekvencí, se kterou se signál opakuje. Hudební nástroje jsou jedním ze zdrojů těchto pravidelných vibrací, jejichž frekvenci lze zpravidla měnit (pomocí klapek, pohybu prstu po struně, atd.). Hlas nástroje však není charakteristický pouze svou výškou, nýbrž i barvou. Ta je určena podobou signálu v rámci jedné periody. 

\begin{figure}[h]\centering
\includegraphics[width=\textwidth,height=\textheight,keepaspectratio]{../img/audio_clarinet}
\caption{Zvuk klarinetu, tóny s různou výškou a dynamikou, 25 milisekund signálu se vzorkovací frekvencí $44\,100\,\rm Hz$. \textcolor{red}{nějak uvést zdroj zvuku \url{https://www.philharmonia.co.uk/explore/sound_samples/clarinet?p=3}}}
\label{obr:audio_clarinet}
\end{figure}

Na obrázku \ref{obr:audio_clarinet} můžeme srovnat tři tóny hrané klarinetem, první dva mají stejnou výšku, jsou ale zahrané s různou intenzitou (dynamikou) \textcolor{red}{je tohle správně formulované, vím že dynamika není intenzita, ale \uv{zahrané s různou dynamikou} mi zní divně?}. Jejich vizuální rozdíl částečně odpovídá i rozdílu v barvě tónu, první tón má příjemný, měkký zvuk; druhý je výraznější a hrubší. Třetí tón se od zbylých liší svou výškou, což lze pozorovat na kratší periodě signálu, která je na obrázku vyznačená úsečkami. 

\begin{figure}[h]\centering
\includegraphics[width=\textwidth,height=\textheight,keepaspectratio]{../img/audio_clarinet_dft}
\caption{Zvuk klarinetu, absolutní hodnota výstupu Fourierovy tranformace signálu délky 4096 s oknem typu Hamming.}
\label{obr:audio_clarinet_dft}
\end{figure}

Jedním ze způsobů analýzy zvukového signálu je pomocí Fourierovy transformace (DFT). Základní myšlenkou je, že na signál lze hledět jako na vážený součet jednodušších signálů. Podobně, jako když se barvy na obrazovce míchají ze tří základních, libovolný zvuk můžeme smíchat ze sady sinusoid. Na obrázku \ref{obr:audio_clarinet_dft} vidíme část výsledku Fourierovy transformace zvuků klarinetu z předchozího příkladu. To zásadní, co na spektru tónu můžeme pozorovat, je jeho podstata jakožto součet \emph{harmonických složek}. Tón, kterému posluchač přisoudí výšku $f_0$, se zpravidla skládá ze součtu sinusoid, jejichž frekvence je celočíselným násobkem základní frekvence $f_0$. Například tedy tón E3 se na obrázku \ref{obr:audio_clarinet_dft} skládá z frekvencí $165\,\rm Hz$, $330\,\rm Hz$, $495\,\rm Hz$, \dots, zároveň intenzita těchto harmonických frekvencí určuje barvu hlasu.

Ukazuje se, že práce s touto reprezentací zvuku je pro analýzu signálu užitečnější, než práce s nezpracovaným signálem. Ze spektrální reprezentace je například na první pohled zřejmý vztah frekvencí porovnávaných signálů, který odpovídá lidské intuici o výšce zvuků --- tón E3 je na obrázku \ref{obr:audio_clarinet_dft} opravdu \uv{výš} než tón H3. Jak si také můžeme všimnout na spektru, pro hlas klarinetu platí, že liché harmonické frekvence jsou mnohem výraznější než sudé (například druhá harmonická frekvence je na obrázku vidět pouze u tónu E3 hraném piano), naopak například lidský zpěv je charakteristický výraznějšími sudými harmonickými složkami. Dále pak můžeme pozorovat, že vyšší harmonické jsou u tónu hraném fortissimo mnohem výraznější než u tónu hraném piano, tyto vyšší frekvence způsobují zmiňovanou výraznější barvu. \textcolor{red}{a tohle se dá říct? Tón hraný fortissimo/piano. Moje znalosti hudební terminologie jsou nulové}

Harmonická struktura, která je vlastní téměř všem zvukům hudebních nástrojů i lidskému hlasu, je zásadní pro metody extrakce melodie. Je to vlastnost, která zvuky potenciálně nesoucí melodii odlišuje například od bubnového doprovodu nebo od šumu. Díky ní se také můžeme pokoušet rozložit souzvuk různě vysokých tónů na jejich původní, čisté signály. 

\textcolor{red}{spektrogram hudby - basa, zpěv a bubny. Pod tím obrázek kompletního přepisu melodických kontur}

\begin{figure}[h]\centering
% \includegraphics[width=\textwidth,height=\textheight,keepaspectratio]{../img/audio_mix_stft}
\caption{lorem ipsum}
\label{obr:audio_mix_stft}
\end{figure}

Nyní, po objevení harmonické struktury zvuků, jimž lze přiřknout výšku, můžeme přejít k jejich hledání v hudební nahrávce. Obrázek \ref{obr:audio_mix_stft} vznikl pomocí opakované Fourierovy transformace, která byla aplikována na po sobě jdoucí, krátké časové úseky vstupní nahrávky, přičemž intenzity frekvenčních složek v každém časovém okamžiku zvuku jsou nyní znázorněny odstínem barvy. Časově-frekvenční reprezentaci signálu nazýváme obecně \emph{spektrogram}, a jeho výpočet je prvním krokem většiny metod pro extrakci melodie.

- vybudovat intuici o spektrogramu - ukázat bubny
- harmonická povaha intervalů! - ukázat překryv melodie a doprovodu

% Pro $h(n, t)$ funkci vah měnících se v čase, lze signál tónu definovat jako:

%	$$ X_{f_0}(t) = \sum_{n=1}^\infty h(n, t) \cdot \sin(n \cdot f_0 \cdot (2\pi t)) $$



% \begin{figure}[h]\centering
% \includegraphics[width=\textwidth,height=\textheight,keepaspectratio]{../img/audio_clarinet_dft_mix}
% \caption{Zvuk klarinetu, postupně vznikající přimícháváním dalších sinusoid}
% \label{obr:audio_clarinet_dft_mix}
% \end{figure}


% tedy rozlišuje jestli se dané vlnění vzduchu opakuje v čase. První z uvedených příkladů zvuků, tedy ty, kterým se dá připsat výška tónu, spojuje právě tato 

% U prvních příkladů  lze snadno představit \uv{vyšší} a \uv{nižší} příklady toho samého zvuku

% Dva zásadní problémy
% - dva zásadní problémy, jako kdyby představovaly osy spektrogramu
% 	- překryv
% 	- definice

\section{Definice melodie}

Rozpoznání melodie hrající skladby je pro většinu posluchačů intuitivní schopností, která je součástí prožitku poslechu hudby, a která jejímu poslechu vůbec dává smysl. Slyšet hudbu a nevnímat melodii je podobné jako poslouchat řeč a nerozumět větám. Ačkoli je melodie tedy termín, který je subjektivně jasný, formální, obecně přijímanou muzikologickou definici, která by se zpětně neodkazovala k posluchači, nemá. 

Z tohoto důvodu si výzkumné týmy zabývající se automatickou transkripcí melodie volí pragmaticky spíše užší definice melodie, se kterými se v jejich kontextu lépe pracuje. Práce \cite{Goto1999}, která je považována za jednu z prvních a důležitých prací v oboru, chápe melodii jako konturu fundamentální frekvence sestávající se z nejsilnějších tónů hrajících v omezeném frekvenčním rozsahu. Tato definice je poměrně úzká, tóny melodie se totiž jistě mohou vyskytovat i mimo autory specifikovaný rozsah a nemusí být vždy v poměru s doprovodem nejsilnější složkou signálu. Z technického hlediska však umožnila autorům implementaci algoritmu běžícího v reálném čase, který poskytoval sémanticky bohatý popis vstupních nahrávek. Navazující články pracují s volnějšími definicemi, které lépe reflektují podstatu melodie. Mimo to se používaná definice proměňuje díky novým datasetům, jejichž autoři tvoří protipól k ryze technickým a objektivním cílům algoritmických metod. Zatímco pro tvorbu algoritmů je praktické zvolit co nejkonkrétnější cíl, při tvorbě datasetu se naopak projevuje lidská subjektivita autorů anotací. 

Kompromisem mezi subjektivní a praktickou definicí se na dlouhou dobu stala \uv{extrakce základní frekvence hlavního melodického hlasu}. Ačkoli melodii v reálném hudebním materiálu obvykle nese více hlasů, které se v hraní střídají (například píseň se zpěvem a kytarovým sólem), v letech 2005 -- 2015 se v soutěži MIREX provádí evaluace pouze nad krátkými výňatky, kde tato definice není omezující. Tento pohled však otevírá také jiné přístupy, například extrakci melodie pomocí modelování hudebního záznamu jako součtu signálu jednoho hlasu a doprovodu \citep{Durrieu2010}, \citep{Bosch2016b} nebo přímo omezení se na separaci lidského zpěvu a doprovodu \citep{Ikemiya2016}. Nově se objevují práce, které \uv{hlavní} melodický hlas neinterpretují nutně jako \uv{nejsilnější}. Skladatelé a hráči používají množství různých postupů, které melodii zvýrazňují --- krom dynamiky ji ovlivňuje například také barva hlasu, vibrato nebo délka not. \cite{Salamon2012a} využívá těchto rysů pro výběr mezi kandidáty na melodickou konturu.

Posunem v rámci MIR komunity bylo zveřejnění nových datasetů MedleyDB \citep{Bittner2014} a ORCHSET \citep{Bosch2016}, oba přináší nová data, ve kterých již melodii nenese pouze jeden hlas po celou dobu skladby. V porovnání s do té doby dostupnými daty jde o mnohem rozmanitější kolekce. V případě MedleyDB jde o první volně dostupný dataset, ve kterém se objevují celé skladby, nikoli pouze výňatky a autoři předkládají rovnou tři verze anotací:

\begin{enumerate}
    \item Základní frekvence nejvýraznějšího melodického hlasu, jehož zdroj zůstává po dobu nahrávky neměnný.
    \item Základní frekvence nejvýraznějšího melodického hlasu, jehož zdroje se mohou měnit.
    \item Základní frekvence všech melodických hlasů, potenciálně pocházejících z více zdrojů.
\end{enumerate}

První formulace je v souladu s doposud používanou definicí. Zbylé dvě se snaží posouvat možné cíle budoucích metod a předložit komunitě nové výzvy, podle \cite{Salamon2014} totiž výzkum začal v letech 2009--2012 stagnovat. Zatímco anotace s jednou melodickou linkou (1. a 2. definice) se v navazujících pracích často používají, zatím žádný článek se nepokusil představit metodu, jejímž cílem by bylo extrahovat více melodických linek (3. definice).

\cite{Bosch2016} při práci na datasetu ORCHSET vychází z článku \cite{Poliner2007}, který definuje melodii jako \uv{jednohlasou sekvenci tónů, kterou bude posluchač nejspíše reprodukovat, pokud jej požádáme o zapískání či zabroukání příslušné skladby}. Přestože nejde o objektivní definici, v praxi se posluchači často na jedné konkrétní sekvenci tónů shodnou, a to jak u populární hudby, kde melodii často nese lidský zpěv, tak u orchestrálních skladeb. Ačkoli se definice neujala pro metody extrakce, \cite{Bosch2016} ji využili pro anotaci výňatků z orchestrálních skladeb, u kterých by předchozí zmíněné definice selhávaly, jelikož pojem melodie je u orchestrální hudby mnohdy komplikovanější než u jiných žánrů. Anotace tak spočívala v přezpívání orchestrálních výňatků skupinou posluchačů a následném srovnání a zpracování těchto nahrávek.

\section{Hluboké učení}

Motivací pro použití metod strojového učení je překonání limitů člověkem navržených, rigidních, pravidlových systémů. Cílem je automatické nalezení optimálního postupu pro řešení úlohy, na základě množství dat, ve kterých strojové učení dokáže nalézt a využít jejich pravidelnosti. V našem případě pak po metodě založené na strojovém učení požadujeme, aby na základě příkladů z trénovací množiny vytvořila funkci salience. 

Výhodou tohoto přístupu je, že o vstupních datech nemusíme dělat žádné předpoklady. Vzniklá metoda pak může v praxi zohledňovat více faktorů ovlivňujících přítomnost melodie, jako je její barva, frekvenční modulace (vibrato, glissando) nebo hlubší vzájemné srovnání současně znějících tónů. Na základě trénovacích příkladů může být tato metoda robustnější vůči většímu spektru barev hlasů nástrojů --- zatímco předešlé metody pro extrakci melodie často uvažují signály s postupně se snižujícím podílem harmonických frekvencí, opravdové signály hudebních nástrojů často tento předpoklad nesplňují (viz obrázek \ref{obr:audio_clarinet_dft}).

První pokus o využití těchto metod představili \cite{Poliner}, vstupní signál transformovali pomocí krátkodobé Fourierovy transformace a část spektra po jednoduché normalizaci použili jako vstupní data pro metodu podpůrných vektorů (SVM). Jejich metoda měla své limitace, výstup byl kvantizován na úroveň jednoho půltónu a tudíž metoda nedokázala dobře postihnout například vibrata. I přesto však tým dosáhl srovnatelných výsledků s tehdejším state-of-the-art. 

Po roce 2005 jakékoli pokusy o aplikaci strojového učení ustávají a na nové metody se čeká až do roku 2016, jedním z důvodů byl jistě nedostatek dat, tuto situaci zlepšil například dataset MedleyDB \citep{Bittner2014} nebo dnes již zaniklý iKala \citep{Chan2015}. Zájem o strojové učení však znovu stoupá s úspěšným využitím hlubokých neuronových sítí napříč obory a tak se na konferenci ISMIR 2016 objevují dva články týmů \cite{Kum2016} a \cite{Rigaud2016}, založené právě na hlubokém učení. V roce 2017 publikuje své metody \cite{Bittner2017} (ISMIR 2017), \cite{Balke2017} (ICASSP 2017), následující rok přináší metody \cite{DBasaranSEssid2018} (ISMIR 2018), \cite{Bittner2018}. V oboru lze tedy od roku 2016 vidět velmi výrazný trend právě směrem k hlubokému učení, a stejný směr je patrný i v příbuzných úlohách přepisu hudby. Tým z laboratoře Google Brain dokázal výrazně zlepšit přepis klavírních skladeb pomocí kombinace konvoluční a rekurentní architektury \citep{Hawthorne2018}. Neuronové sítě také zlepšují výsledky na poli oddělení signálů \citep{Stoller2018}.


% \cite{Thickstun2016} - musicnet
% \cite{Hawthorne2018} - google magenta

% V oboru transkripce hudby se trend v použití hlubokých sítí
\chapter{Související práce}\label{cha:souvisejici}

Pokusy o vytvoření automatické metody pro kompletní transkripci hudby se podle \cite{Poliner2007} objevují již od sedmdesátých let, z důvodu značné obtížnosti této úlohy, která strmě roste s každým dalším přidaným hlasem ve zkoumaném signálu, však dodnes jedná o otevřený problém. Z tohoto důvodu se od devadesátých let objevují práce, které se pokouší alespoň o částečný automatický popis některých muzikálních aspektů skladeb.

Jednou z prvních je práce týmu \cite{Goto1999}, který se záměrně omezuje na identifikaci jedné, nejhlasitější, spojité křivky fundamentní frekvence hlasu (F0) v omezeném frekvenčním rozsahu. Vzniklé transkripce pak sice nejsou kompletní, na druhou stranu je jejich získání výpočetně nenáročné a přitom poskytují sémanticky bohatý popis nahrávek, který je poměrně často shodný s melodií. Ustanovením úlohy pojmenované jako \uv{Predominant-F0 Estimation} (PreFEst) byly položeny základy pro vznik navazujících prací a soutěží zabývající se automatickým přepisem melodie.

Největší rozkvět v oboru začal od roku 2004. Uspořádáním první soutěže pro porovnání systémů pro automatický popis hudby v rámci konference ISMIR (ISMIR 2004 Audio Description Contest), se ustanovily priority, formalizovaly podmínky evaluace a byly sestaveny první kolekce dat pro testování algoritmů (\cite{Downie2010}). Soutěž se v následujícím roku přerodila do samostatné každoroční události, v níž soutěží stále více týmů v rostoucím počtu úloh. 

V této kapitole představíme existující metody a společné přístupy k řešení úlohy extrakce melodie. Nejprve úlohu dekomponujeme na podúlohy a pro tyto podúlohy uvedeme příklady existujících metod. V závěru kapitoly pak provádíme kvantitativní srovnání existujících metod na základě dat ze soutěže MIREX, abychom vybrali nejnadějnější metody, které replikovat a na něž v této práci navázat.

\section{Průzkum existujících metod}

Jen do soutěže MIREX se od roku 2005 přihlásilo 45 týmů s 62 různými metodami pro extrakci melodie, s různou mírou přesnosti přepisu. Mezi přístupy k tomuto problému tedy existuje veliká rozmanitost, jejíž kompletní popis přesahuje rámec této práce. Zaměříme se proto na společné rysy a celkové trendy v oboru. 

Shrnující práce od \cite{Poliner2007} a \cite{Salamon2014} se při charakterizaci systémů pro transkripci odkazují na příbuznou úlohu odhadu fundamentální frekvence monofonní nahrávky. Algoritmy pro monofonní tracking na základě vstupního signálu $x_{mono}(t)$ počítají \emph{funkci salience} $S_{x_{mono}}(f_\tau, \tau)$ pro každý krátký časový okamžik (okno) $\tau$ a frekvenci $f_\tau$. Výsledkem této funkce je relativní ohodnocení (příp. pravděpodobnost) jednotlivých frekvencí obsažených ve vstupním signálu, které značí, zda-li je daná frekvence fundamentální frekvencí znějícího hlasu. 

Pro zvýšení odolnosti vůči šumu, přeslechu, dozvuku a jiným vlivům, které zhoršují kvalitu odhadu salience, se využívá také spojitosti fundamentální frekvence. Pro zajištění kontinuity extrahovaných frekvenčních kontur se zohledňuje také faktor temporálních závislostí $C(\mathbf{f})$, jejímž vstupem je kandidátní kontura $\mathbf{f}$ a výstupem je ohodnocení této celé kontury na její spojitost. Například tato funkce může penalizovat odhady, ve kterých výstupní F0 často přeskakuje o oktávu, což je u skutečného signálu nepravděpodobné a naopak jde o častou chybu při výpočtu salienční funkce signálů se silnými sudými harmonickými frekvencemi.

Výstupem monofonního trackingu je posloupnost frekvencí s maximální saliencí a spojitostí, tedy posloupnost frekvencí, které jsou nejlépe ohodnocenými kandidáty na fundamentální frekvenci a zároveň tato celá sekvence má také vysoké ohodnocení spojitosti.

    $$\hat{\mathbf{f}} = \argmax_{\mathbf{f}}{[\sum_{\tau}{S_{x_{mono}}(f_\tau, \tau)} + C(\mathbf{f})]}$$

Přejdeme-li k úloze extrakce melodie, vstupní polyfonní signál $x(t) = x_m(t) + x_d(t)$ se zde obecně skládá ze směsi melodického hlasu $x_m(t)$ a hudebního doprovodu $x_d(t)$, cílem metod pro extrakci je z pohledu přepisu fundamentální frekvence zvýšení odolnosti algoritmu vůči mnohem výraznějšímu druhu šumu - hudebnímu doprovodu $x_d(t)$. Výstupem našeho systému tedy bude posloupnost odhadů frekvence v každém časovém okně vstupního signálu, reprezentovaná vektorem $\hat{\mathbf{f}}$:

    $$\hat{\mathbf{f}} = \argmax_{\mathbf{f}}{[\sum_{\tau}{S'_x(f_\tau, \tau)} + C'(\mathbf{f})]}$$

kde $f_\tau$ je frekvence na pozici $\tau$ ve vektoru $\mathbf{f}$. $S'_x(f_\tau, \tau)$ je upravená funkce salience, která při výpočtu zohledňuje vliv doprovodu a složka $C'(\mathbf{f})$ představuje ohodnocení celého průběhu melodie.

Spolu s odhadem frekvencí by také měl systém na výstupu určit úseky, ve kterých v nahrávce melodie zní a kdy nikoli. K výstupu tedy patří také vektor $\hat{\mathbf{v}}$, se stejným počtem složek jako $\hat{\mathbf{f}}$, který indikuje přítomnost melodie v každém časovém okně $\tau$.

Většina existujících metod sdílí podobnou základní strukturu při řešení extrakce, která se zakládá na popsané formalizaci. Prvním krokem je transformace zvuku do frekvenční domény a následný odhad znějících výšek tónů v polyfonním signálu (výpočet funkce salience), druhým krokem je pak zpracování těchto odhadů a výběr melodie (tedy zpřesnění výsledné $\hat{\mathbf{f}}$ pomocí $C'(\mathbf{f})$). Přístupy k řešení těchto dvou kroků již s konkrétními příklady nastíníme v dalších sekcích.

% \subsection{Odhad výšek tónů}

\subsection{Spektrální analýza}

Zvuk hraného tónu na melodickém nástroji je z fyzikálního pohledu periodická změna tlaku vzduchu. Perioda tohoto signálu se nazývá fundamentální frekvence (označujeme F0) a zpravidla je tento signál složen ze součtu řady sinusoid, jejichž frekvence jsou celočíselným násobkem fundamentální frekvence. V čase měnící se amplitudy těchto \emph{harmonických frekvencí} udávají hlasitost a barvu hlasu, výška první harmonické frekvence (tj. výška fundamentální frekvence) pak ve většině případů odpovídá posluchačem vnímané výšce tónu. 

\begin{figure}[h]\centering
\includegraphics[width=\textwidth,height=\textheight,keepaspectratio]{../img/sig_spec_sal_grey}
\caption{Znázornění kroků spektrální analýzy a výpočtu funkce salience.}
\label{obr:sig_spec_sal}
\end{figure}


Prvním krokem metod pracujících s hudebním signálem je proto provedení spektrální analýzy, jde o převod zvuku do frekvenční reprezentace, která odhaluje tyto harmonické struktury tónů a umožňuje s nimi dále pracovat. 

\subsubsection{Krátkodobá Fourierova transformace}

Ačkoliv je přístupů ke spektrální analýze je více, nejpřímočařejší a podle \cite{Dressler2016} nejčastěji používaná metoda je \emph{krátkodobá Fourierova transformace} (STFT). Jejím principem je rozdělení vstupního signálu na množinu překrývajících se oken konstantní délky a výpočet Fourierovy transformace těchto krátkých zvukových úseků. Komplexní výsledek transformace umocníme a získáme tzv. výkonové spektrum signálu, které obsahuje informaci o poměrech energie frekvencí, ze kterých se signál v okně skládá. Spektrogram $X(f, \tau)$ vypočteme v čase $\tau$ a frekvenční složce $f$ jako:

    % $$X(f, \tau) = \int_{-W/2}^{W/2}{w(t)y(\tau + t)e^{-j2\pi f t} \mathrm{d}t}$$

$$ X(f, \tau) = \abs{\sum^{\infty}_{n=-\infty}{x(n)w(n-\tau)e^{-2\pi i \cdot n \cdot f}}}^2 $$

Pro diskrétní vstupní signál $x(n)$ a okénkovou funkci $w(n)$. V této práci budeme pro výpočty spektrogramů používat Hannovo okno \citep{1455106}, které omezuje tzv. prosakování ve spektru pomocí zmenšení diskontinuity na okrajích okna Fourierovy transformace postupným snížením hodnot signálu k nule na těchto okrajích. 
\[
w_{hann}(n) =
\begin{cases}
    \cos^2{\frac{\pi n}{N}}, & \text{pro} -\frac{N}{2} \leq n \leq \frac{N}{2},\\
    0, & \text{jinak}.
\end{cases}
\]

Kde $N$ je požadovaná velikost okna STFT tranformace (kvůli obvyklé praxi výpočtu STFT pomocí algoritmu rychlé Fourierovy transformace (FFT) se hodnota $N$ volí z $N \in \{512, 1024, 2048, \dots\}$).

\subsubsection{Logaritmická osa spektrogramu}

\begin{figure}[h]\centering
\includegraphics[width=\textwidth,height=\textheight,keepaspectratio]{../img/stft_triangular_filters_grey}
\caption{Příklad trojúhelníkových filtrů pro transformaci frekvenční domény na logaritmickou škálu.}
\label{obr:stft_triangular_filters}
\end{figure}

Výsledkem STFT je rozložení signálu na jednoduché frekvenční složky (sinusoidy) s konstantně vzdálenými frekvencemi. Jinými slovy frekvenční osa spektrogramu vytvořeném metodou STFT je lineární. Jak jsme již nastínili v úvodu, povaha hudebních intervalů a harmonických struktur tónů spočívá v tom, že téměř všechny periodické signály se v hudební skladbě vyskytují ve vzájemných poměrech (v případě intervalů v poměrech $2^{\frac{n}{12}}$ a v případě harmonických frekvencí v celočíselných). K tónu, jehož základní frekvence je rovna $440\,\rm Hz$, patří také harmonické složky s frekvencemi $880\,\rm Hz$, $1320\,\rm Hz$, \dots, tedy absolutní vzdálenosti na spektrogramu STFT mezi frekvencemi harmonických složek jsou závislé na výšce základní frekvence. Toto způsobuje obtíže při analýze signálů, jelikož všechny uvažované frekvenční rozdíly jsou relativní. Častým druhem zpracování STFT je proto převod frekvenční osy na logaritmickou, na té pak platí pro vzdálenost libovolné fundamentální frekvence $f_0$ a její libovolné $h$-té harmonické frekvence $f_h = f_0 \cdot h$:

$$ \log{f_h} - \log{f_0} = \log{(h \cdot f_0)} - \log{f_0} = \log{h} + \log{f_0} - \log{f_0} = \log{h} = \rm const.$$

Tedy absolutní vzdálenosti uvnitř harmonické struktury tónů na spektrogramu s logaritmickou osou frekvence zůstávají konstantní nezávisle na výšce fundamentální frekvence. Tento přepočet se obvykle provádí pomocí banky filtrů s trojúhelníkovou odezvou, pro transformovaný signál $X(f, \tau)$ s $N$ frekvenčními složkami spočteme nový spektrogram s logaritmickou osou následovně:

$$ X_{\mathrm{log}}(\omega, \tau) = \sum^{\infty}_{f=-\infty}{g_\omega(f)X(f, \tau)}$$

Přičemž $g_\omega(f)$ značí odezvu trojúhelníkového filtru pro výstupní frekvenční složku $\omega$.

\subsubsection{Multirezoluční transformace}\label{sec:multirezolucni}

Na libovolnou metodu převodu diskretizovaného signálu na frekvenční doménu se vztahuje Gaborův limit, který popisuje závislost přesnosti lokalizace signálu ve frekvenční a časové doméně \citep{Gabor1945}. Volbou délky vstupního okna transformace zpřesňujeme buď frekvenční nebo časové rozlišení výsledné spektrální reprezentace. Zvolíme-li krátké vstupní okno, zvyšujeme časové rozlišení (krátké okno lépe zachycuje rychlé změny průběhu signálu), avšak ztrácíme přesnost na frekvenční ose, opačný vztah platí pro volbu delšího okna.

Tato limitace je markatní zejména pokud STFT používáme pro hudební data. , u vyšších tónů jsou proto vzdálenosti mezi frekvencemi signálů větší než u nižších tónů. Frekvenční rozlišení STFT je však konstantní na celém výstupním frekvenčním rozsahu a volba velikosti okna transformace zajistí dobrý poměr frekvenčního a časového rozlišení jen pro část rozsahu. Ve výsledku je pak buď pro vyšší frekvence okno příliš velké (zbytečně detailní frekvenční rozlišení na úkor časového rozlišení) a nebo naopak pro nižší frekvence je okno nedostačující (rozlišení frekvence nemusí být pro basy ani na úrovni půltónů).

Z tohoto důvodu existují vedle STFT i další metody, jejichž cílem je nabídnout lepší kompromis frekvenčně-časového rozlišení v kontextu melodických dat. \cite{Goto1999} používají MRFFT (Multi-Resolution Fast Fourier Transform), principem je opakovaný downsampling signálu (převzorkování na nižší vzorkovací frekvenci) a aplikace Fourierovy transformace na každý vzniklý signál; s každou iterací spektrum obsahuje čím dál podrobnější informace o nižších frekvencích, protože vyšší frekvence se při downsamplingu ztratí. \cite{Brown1990} popsala metodu Constant-Q Transform (CQT), která spočívá v použití proměnné délky okna Fourierovy transformace pro výstupní frekvenční pásma, která rovnoměrně pokrývají logaritmickou osu frekvence. \cite{Cancela2010} kombinuje CQT a Chirp Z-transform, jedná se o zobecnění diskrétní Fourierovy transformace, ve které je signál rozložen na tzv. lineární čerpy --- sinusoidy s proměnnou frekvencí. Díky tomu transformace dokáže s lepším rozlišením zachytit signály, které rychle mění výšku tónu, v hudbě například vibrato. \cite{Paiva2004} napodobují mechanismy lidského sluchu pomocí banky pásmových filtrů (Cochleagram) s logaritmicky rozmístěnými mezními frekvencemi a sumy autokorelací na jednotlivých frekvenčních pásmech signálu (Summary correlogram).

I přes uvedené důvody se \cite{Salamon2014} a \cite{Dressler2016} domnívají, že metoda zpracování signálu příliš neovlivňuje výslednou přesnost algoritmů pro přepis melodie. Tvrzení dokládají jednak celkovým srovnáním výsledků metod ze všech ročníků soutěže MIREX a jednak neochvějnou převahou využití krátkodobé Fourierovy transformace, jakožto efektivní a dostačující metody pro spektrální analýzu.

\subsubsection{Postprocessing spektrogramu}

Po převodu signálu na frekvenční reprezentaci následuje u většiny metod některý druh úpravy celého spektrogramu, předcházející samotnému výpočtu \emph{funkce salience}. Výsledkem tohoto kroku může být potlačení šumu a nemelodických částí signálu, zpřesnění informace o výšce znějících frekvencí nebo normalizace či jiná úprava amplitud spektrogramu.

Nejčastější úpravou je nalezení lokálních maxim (vrcholů). Potlačením nemaximálních oblastí se zbavíme velkého množství nemelodických složek signálu, přitom informaci o těch melodických neztratíme. Výhodou práce s množinou vrcholů je, že jejich frekvenci lze na základě spektra dále zpřesnit pomocí parabolické interpolace (\cite{Rao2010}) a nebo využitím úhlové frekvence (\cite{Salamon2012a}, \cite{Dressler2009}). 

Jiným druhem úpravy jsou různé způsoby normalizace, ať jednoduché aplikace logaritmu na jednotlivé hodnoty spektrogramu (\cite{Cancela2008}, \cite{Bittner2017}) nebo složitější strategie normalizace, které se aplikují na celá výstupní okna krátkodobé Fourierovy transformace (spectral whitening, \cite{Ryynanen2008}), či které používají pohyblivé průměry nebo jinou metodu, beroucí v potaz širší zvukový kontext. Cílem normalizace je zvýraznění slabších harmonických frekvencí a potlačení celopásmových zvuků (například perkusí). Principiálně podobným krokem je aplikace pásmového filtru (\cite{Goto1999}) pro zvýraznění frekvencí obsahující melodii. Případně využití psychoakustických filtrů modelující lidské vnímání hlasitosti (\cite{Salamon2012a}, \cite{Ikemiya2016}). Ze signálu lze také oddělit melodické nástroje a perkusivní doprovod pomocí metod separace signálů (source separation). Používanými metodami jsou například Harmonic/Percussive Sound Separation (HPSS) (\cite{Tachibana2010}) nebo Robust principal component analysis (RPCA) (\cite{Ikemiya2016}).

\subsection{Funkce salience}

Salience tónu vyjadřuje míru důležitosti či nápadnosti ke svému hudebnímu okolí. Nejvíce ji ovlivňuje hlasitost v poměru ke zbylým znějícím hlasům, vliv má ale také řada dalších charakteristik hraní. \cite{Dressler2016} mezi příklady uvádí například frekvenční modulaci, jako je vibrato nebo glissando, zejména oproti frekvenčně stálému hudebnímu doprovodu (piano, kytara). Velký vliv má samozřejmě také i barva hlasu. Lidský zpěv nebo obecně zvuky se silnějšími vyššími alikvótními frekvencemi snadněji upoutají pozornost. V případě vícehlasu mají obecně posluchači potíže rozeznat výšky tónů uvnitř souzvuku. Pokud má posluchač přiřadit pociťovanou výšku tónu akordu, obvykle volí nejvyšší či nejnižší ze znějících frekvencí.

Výstupem funkce salience je ohodnocení každé výšky tónu v každém časovém okamžiku nahrávky, které co nejlépe odpovídá v relativních poměrech výše popsané zvukové salienci. Jelikož neexistují žádné studie, které by se zabývaly měřením a kvantifikací toho, co člověk považuje za salientní v hudbě, nelze posoudit, jak dobře výsledky obvyklých způsobů výpočtu funkce salience korelují s mírou, ze které vychází. \cite{Bittner2018a} se však domnívá, že odhad bude velmi hrubý, většina postupů totiž do výpočtu zahrnuje pouze hlasitost hlasu, a tedy vynechává řadu jiných důležitých faktorů, které salienci ovlivňují.

Přístupy k výpočtu by se daly zařadit do tří kategorií - sčítání harmonických frekvencí, odhad parametrů modelujících vstup, a metody strojového učení. 

\subsubsection{Sčítání harmonických frekvencí}

Metody založené na sčítání harmonických frekvencí jsou principiálně nejjednodušší skupinou. Vychází z práce \cite{Hermes1988}, jejich podstatou je využití harmonické struktury zvuku tónů. Ohodnocení frekvence $f_\tau$ získáme váženou sumou amplitud všech jejích harmonických frekvencí $h \cdot f_\tau$. Pro spektrogram $X(f, \tau)$ signálu $x$, funkci vah $g(f_\tau, h)$ a $N_h$ počet zahrnutých harmonických frekvencí v sumě:

    $$S_x(f_\tau, \tau) = \sum_{h=1}^{N_h}{g(f_\tau, h)\abs{X(h \cdot f, \tau)}}$$

Pro ilustraci uvažme jednohlasý harmonický signál s fundamentální frekvencí $f^\star$. Hodnoty spektrogramu $X(f, \tau)$ tedy budou vyšší kolem frekvencí $H_{f^\star} = \{1\cdot f^\star, 2\cdot f^\star, 3\cdot f^\star, \dots\}$ a jinde nulové. Funkce $S_x(f, \tau)$ bude tedy pro $f \not\in H_{f^\star}$ nulová a v $f^\star$ bude nabývat globálního maxima. Příklad výstupu této metody pro polyfonní nahrávku je na obrázku \ref{obr:sig_spec_sal}.

\cite{Dressler2011} tuto metodu vylepšuje zpracováváním dvojic vrcholů vstupního spektrogramu, její výsledný salienční spektrogram obsahuje méně kandidátů na fundamentální frekvenci. \cite{Cancela2008} se pokouší zmenšovat chybné hodnoty salienční funkce pomocí vyhodnocení subharmonických frekvencí.

\subsubsection{Statistické modelování signálu}

\begin{figure}[h]\centering
    \includegraphics[scale=0.8]{../img/tone_model}
\caption{Ilustrace modelu tónu spolu se signálem, převzato z \cite{Marolt2004}}\label{obr:tone_model}
\end{figure}

Jiným přístupem k počítání funkce salience, který používá \cite{Goto1999}, je modelování okna spektrogramu váženým součtem harmonických struktur (modelů tónů). Snažíme se vrcholy ve spektru rozdělit mezi různě silně znějící tóny tak, aby v součtu co nejlépe odpovídaly měřeným intenzitám. Přístup se jinými slovy snaží zjistit, jaké tóny musely v danou chvíli znít, aby vzniklo dané spektrum. 

Vstupem metody je okno normalizovaného spektrogramu $p_X^{(t)}(x)$ v čase $t$. Okno se pokusíme modelovat jako hustotu pravděpodobnosti $p(x; \theta^{(t)})$ vzniklou váženou směsí modelů všech možných tónů melodie v definovaném rozsahu frekvencí v intervalu $[F_l, F_h]$. Hustotu pravděpodobnosti jednoho z tónů s fundamentální frekvencí $F$ označíme jako $p(x|F)$ (obrázek \ref{obr:tone_model} ukazuje jeden z možných modelů tónu), a jako $w^{(t)}(F)$ označíme váhu, kterou model tónu $p(x|F)$ přispívá do celkové smíšené hustoty pravděpodobnosti. Pak $p(x; \theta^{(t)})$ definujeme jako:

$$p(x; \theta^{(t)}) = \int_{F_l}^{F_h}{w^{(t)}(F)p(x|F) \mathrm{d}F}$$
$$\theta^{(t)} = \{\, w^{(t)}(F) \mid F_l < F < F_h \,\}$$

Cílem pak je nalezení takových parametrů $\theta^{(t)}$, aby model $p(x; \theta^{(t)})$ dobře popisoval pozorovaní $p_X^{(t)}(x)$. K tomu \cite{Goto1999} využívá Expectation-Maximization (EM) algoritmus. Výsledné parametry $\theta^{(t)}$ jsou pak hodnoty salienční funkce.

% \textcolor{red}{TODO: popsat Durrieu, jakožto jiný přístup k modelování signálu}

\subsubsection{Metody strojového učení}

K výpočtu funkce salience můžeme využít také metody strojového učení. První práci využívající těchto metod představují \cite{Poliner}, kteří úlohu formulují jako klasifikační. Vstupní okno signálu jejich metoda klasifikuje do jednotlivých tříd tónů melodie, výstup je tedy kvantizován na rozlišení jednoho půltónu. Pro tento účel využívají metodu podpůrných vektorů (SVM) přičemž vstupními příznaky je část spektrogramu signálu, konkrétně vektor s 256 složkami, který získávají pomocí krátkodobé Fourierovy transformace podvzorkovaného signálu. Metoda dosahovala průměrných výsledků v soutěžích MIREX 2005 a 2006. Strojové učení se v oboru příliš neuchytilo, z důvodu nedostupnosti dat, ale možná také i kvůli limitacím, které \cite{Poliner} ve své práci prezentují.

V roce 2016 na tyto nedostatky odpovídá práce \cite{Kum2016}. Augmentací dat a zaměřením se na odhad výšky zpěvu místo melodie úspěšně překovávají problém nedostatku dat. Místo klasifikátoru SVM používají hluboké neuronové sítě (3 skryté plně propojené vrstvy) a problém kvantizace na úroveň půltónu řeší natrénováním tří nezávislých sítí s různě jemným rozlišením výstupu, které pak spojí v jeden výstup. V jejich případě platí, že jemná síť má sice lepší výstupní rozlišení, celkově má ale horší přesnost, naopak je tomu u sítě s hrubým rozlišením. Proto jejich výsledky kombinují a dosahují tehdejších state-of-the-art výsledků na vokálních datech. Podobný postup v tomtéž roce podniká i tým \cite{Rigaud2016}, podobně jako v práci \cite{Kum2016} se tým omezuje na odhad výšky zpěvu, používá augmentaci dat, přechází na hlubokou neuronovu síť (2 skryté plně propojené vrstvy) a používá jemnější rozlišení výsledného vektoru pravděpodobnostního rozdělení znějící výšky. Rozdílem je ale předzpracování dat pomocí dekompozice na harmonické a perkusivní složky pomocí HPSS. Práce \cite{Balke2017} používá neuronovou síť s jednou skrytou vrstvou pro anotaci jazzových sól ze spektrogramu s logaritmickou osou frekvence. 

\cite{Bittner2017} představuje první pokus o použití hlubokých sítí na úlohu extrakce melodie bez zaměření na zpěv. Úlohu formuluje jako odstranění šumu obrázku, cílem je ze vstupního spektrogramu pomocí hluboké sítě s konvolučními vrstvami vytvořit salienční funkci. Vstupní a výstupní data mají tedy stejné měřítko, výstup však v ideálním případě obsahuje pouze informace o fundamentálních frekvencích znějící melodie. Velmi přínosný je také popis vstupní spektrální reprezentace HCQT, která spočívá ve výpočtu několika CQT spektrogramů (kanálů), jejichž počáteční frekvence jsou vzdálené v harmonických poměrech, tudíž (protože CQT spektrogram používá logaritmickou osu frekvence) všechny související harmonické složky na celém spektru jsou na těchto spektrogramech zarovnané nad sebou v ose kanálů. Tento koncept je hlouběji popsán v kapitole Experimenty v sekci HCNN, kde tento nápad aplikujeme nejen na vstupní spektrogram, ale nově také na propojení celé neuronové sítě. Dalším důležitým přínosem práce je úprava reprezentace cílové salienční funkce pomocí rozostření hranic cílové výšky tónu. Zatímco předchozí metody trénovali sítě jako diskrétní klasifikaci s jednou správnou výstupní třídou, v této práci je cíl trénování gaussián se střední hodnotou uprostřed výšky tónu.

Práce \cite{Bittner2017} dokázala překonat state-of-the-art metody pouhým výpočtem salienční funkce. Metoda tedy úplně přeskakuje vyhlazování odhadů v čase a používá pouze nejzákladnější metody pro detekci melodie pomocí práhování. Zároveň však její salienční funkce pro odhad jednoho okna délky $\approx 11\,\rm ms$ zpracovává přibližně $150\,\rm ms$ vstupního okna, tedy v porovnání s existujícími metodami její salienční funkce zpracovává velmi široký kontext, proto není vyhlazování jejich výsledků nezbytné.

\cite{DBasaranSEssid2018} na této práci buduje a jednak prozkoumává možnosti použití jiné vstupní spektrální reprezentace (založené na práci \cite{Durrieu2011a}) a jednak do architektury sítě zabudovává rekurentní neuronové sítě, které zajišťují zmiňované vyhlazování odhadů. Jeho metoda překonává výsledky \cite{Bittner2017}, nevýhodou jeho přístupu je však hrubý výstup s rozlišením na půltóny.

V rámci soutěže MIREX také přibývají od roku 2016 nové metody založené na hlubokých sítích, ty se však buď stále zaměřují pouze na extrakci zpěvu (například \cite{Su2018}) případně k nim nelze dohledat související článek (v roce 2016 metody účastníka Zhe-Cheng Fan a v roce 2018 metoda od týmu Sanguen Kum, Juhan Nam).

\subsection{Hledání melodie}

Po výpočtu funkce salience $S_x(f_\tau, \tau)$ máme k dispozici odhady fundamentálních frekvencí v signálu, z těchto ohodnocení pak musíme vybrat výslednou konturu melodie. V závislosti na způsobu výpočtu funkce salience tyto ohodnocení frekvencí více či méně odpovídají jejich důležitosti v signálu. Triviálním řešením zpracování těchto hodnot by bylo vybrat frekvence s maximální saliencí pro každé časové okno $\hat{f}_\tau = \argmax_{f_\tau}{S_x(f_\tau, \tau)}$, tento jednoduchý přístup však mnoho metod nevolí, protože jeho výstup má u složitějších skladeb tendenci \uv{přeskakovat} mezi doprovodem a melodií.

Obecně se dají přístupy rozdělit na pravidlové metody a statistické metody, dále se pak metody liší v tom, zda na základě salienční funkce vytváří abstraktnější popisy obsahu - ať už na úrovni jednotlivých celistvých konturů, tónů nebo celých frází.

\cite{Goto1999} pro sledování melodie používá množinu \uv{agentů}, pohybujících se v čase po výstupu salienční funkce a na základě předem definovaných pravidel jejich pohyb zaručuje kontinuitu výstupní fundamentální frekvence. Podobné sledování kontur v čase na základě salienční funkce využívá i \cite{Dressler2009}. Jinou pravidlovou metodou je opakované nalezení globálního maxima, jeho iterativní prodlužování v obou směrech časové osy a následné vymazání této nalezené kontury z výstupu salienční funkce, čímž dovolíme nalezení nového globálního maxima (\cite{Cancela2008}, \cite{Salamon2012a}). Z extrahovaných kontur následně můžeme vybrat ty, které splňují kritéria pro melodické kontury.

Jiným přístupem k hledání melodie je použití statistických metod, jako je například modelování pomocí skrytých Markovových modelů. Na salienční funkci tyto metody pohlíží jako na sérii pozorování a pomocí hledání nejpravděpodobnější cesty skrz stavy modelu s vhodně nastavenými či z dat získanými pravděpodobnostmi přechodu získávají vyhlazenou konturu melodie. Tyto modely mohou být velmi komplexní a můžou zahrnovat modely průběhu not \citep{Ryynanen2008} nebo naopak velmi minimalistické, zahrnující pouze stavy pro jednotlivé tóny (\cite{Yeh2012})

\subsubsection{Přítomnost melodie (voicing)}

Důležitou součástí algoritmů pro extrakci melodie je detekce melodie v signálu. Většina metod tento krok provádí na konci vyhodnocování pomocí pevně nastaveného či dynamického práhování, jiné metody detekci melodie řeší filtrováním melodických kontur (\cite{Salamon2012a}). V případě statistických metod je stav neznějící melodie často přímo zabudován do statistického modelu \citep{Ryynanen2008}. Některé metody také používají klasifikační metody strojového učení \citep{Rigaud2016}.

\section{Srovnání existujících metod}

\begin{figure}[h]\centering
\includegraphics[width=\textwidth,height=\textheight,keepaspectratio]{../img/mirex_results_grey}
\caption{Výsledky metod v soutěži MIREX v letech 2015-2018 s vybranými metodami ze starších ročníků.}
\label{obr:mirex_results}
\end{figure}
Pro celkové kvantitativní srovnání metod jsme zpracovali výsledky všech ročníků soutěže MIREX. Tuto soutěž a používané datasety blíže popisujeme v kapitole \nameref{cha:evaluace}. V této sekci prezentujeme shrnutí výsledků metod, pro které existují výsledky na všech evaluačních datasetech soutěže MIREX, vybíráme tedy převážně z metod od roku 2015, kdy byl vydán doposud nejnovější evaluační dataset ORCHSET. Díky práci \cite{Bosch2014}, který starší metody spustil na svém datasetu ORCHSET, můžeme ke srovnání přidat také výsledky metod \cite{Dressler2009}, \cite{Salamon2012a} a \cite{Durrieu2010}. Celkové srovnání nalezneme v tabulce \ref{obr:mirex_results}. Na základě těchto výsledků vybíráme metodu \cite{Salamon2012a}, jejíž implementace je volně dostupná a spolu s prací \cite{Dressler2009} dosahuje v průměru na datasetech MIREX nejlepších výsledků. V této práci ji lze považovat také jako zástupce metod, které nejsou založeny na strojovém učení.\footnote{Pro metodu \cite{Dressler2009} implementace zveřejněná není.}

\begin{figure}[h]\centering
\includegraphics[scale=0.5]{../img/mirex_results_cumulative_grey}
\caption{Stagnující vývoj metod pro extrakci melodie.}
\label{obr:mirex_results_cumulative}
\end{figure}

Na základě grafu \ref{obr:mirex_results} také vidíme, že největší variabilitu mají výsledky na datasetu ORCHSET, zde mají metody velký prostor pro zlepšení, naopak u datasetů INDIAN08 a variant MIREX09 je zřejmá jistá hranice kterou je pro metody obtížné překonat. \cite{Salamon2014} ve svém přehledovém článku dochází k závěru, že vývoj metod extrakce melodie začal od roku 2009 stagnovat, na obrázku \ref{obr:mirex_results_cumulative} znázorňujeme maximální dosaženou celkovou přesnost metod na jednotlivých datasetech od počátku soutěže MIREX. Bohužel musíme konstatovat, že v rámci soutěže MIREX stagnace pokračuje doposud, přitom výzkum metod stále pokračuje a zejména díky strojovému učení se obor posouvá. V rámci MIREXu však nebyly vyhodnoceny nové stěžejní metody \cite{Bittner2017} a \cite{DBasaranSEssid2018} a zájem o soutěž v této kategorii postupně upadá (v roce 2017 nesoutěžily žádné týmy, v roce 2018 pouze dva). Důvodem může být právě nedostatečný prostor pro zlepšení kvůli nedostatku nových, zajímavých dat.


\subsection{Replikace výsledků}

Pro srovnání metod představovaných v této práci spouštíme metody \cite{Salamon2012a}, \cite{Bittner2017} a \cite{DBasaranSEssid2018} na testovacích množinách. Všechny tři metody mají volně dostupnou implementaci, první ve formě VAMP plug-inu,\footnote{\url{https://www.upf.edu/web/mtg/melodia}} zbylé jsou implementovány v jazyce Python a používají standardní knihovny určené pro hluboké učení.\footnote{\url{https://github.com/rabitt/ismir2017-deepsalience/}}\footnote{\url{https://github.com/dogacbasaran/ismir2018_dominant_melody_estimation}} Výsledky těchto metod uvádíme v kapitole Výsledky.

Poznamenáme, že implementace algoritmu \cite{Salamon2012a} existují dvě, druhá v rámci knihovny Essentia,\footnote{https://essentia.upf.edu/documentation/} tato implementace však v porovnání s implementací VAMP podávala výrazně horší výsledky napříč datasety. V kapitole Výsledky proto používáme implementaci VAMP.

Pokusili jsme se také o replikaci výsledků \cite{Bosch2014}, bohužel se nám ale kvůli nekompatibilitě mezi verzemi knihoven nepodařilo tuto 5 let starou metodu spustit.

% Pro srovnání jsme se také pokusili spustit metodu \cite{Durrieu2010}, délka běhu algoritmu je však při zachování výchozího nastavení parametrů neúnosně dlouhá, 23 minut dat (ORCHSET) nám trvalo zpracovat dva dny. Tudíž její spuštění na větší korpusy (MedleyDB, WJazzD) je mimo možnosti autora práce.

% \textcolor{red}{TODO: tabulka metod s popisem architektury}
\chapter{Datasety}\label{cha:datasety}

Nedostupnost dostatečného množství dat pro automatickou transkripci melodie představuje zejména pro metody strojového učení značný problém. Zatímco pro vzdáleně příbuznou úlohu automatického přepisu mluveného slova existuje tisíce hodin nahrávek (například dataset LibriSpeech, který vznikl na základě audioknih), největší dataset s přepsanou melodickou linkou MedleyDB má celkovou délku pod šest hodin. Do roku 2014, kdy MedleyDB vznikl, existovaly datasety, které byly buď rozmanité, ale příliš krátké (ADC04, MIREX05, INDIAN08) nebo naopak celkově větší, ale žánrově a hudebně homogenní (MIREX09, MIR1K, RWC). V roce 2015 byl vydán dataset Orchset, který obsahuje 23 minut výňatků z orchestrálních skladeb různých období. Za dataset pro extrakci melodie se také dá považovat Weimar Jazz Database, který je sice primárně zaměřený na využití v muzikologii, nicméně obsahuje přes 450 přepsaných jazzových sól. Novinkou z roku 2017 je vydání datasetu MDB-melody-synth, který byl automaticky vygenerován základě vstupní vícestopé hudby (převzaté z MedleyDB), existuje tedy naděje, že současný korpus pro přepis melodie by se mohl v budoucnu rozšířit o velkou část automaticky přesyntetizovaných, veřejně dostupných vícestopých nahrávek.

Co se týče blízké úlohy transkripce hudby, velikost největších datasetů se pohybuje v řádu desítek hodin, tudíž jde stále o omezené kolekce. Mezi největší se řadí MusicNet (orchestrální, 34 hodin), MAPS (klavír, 18 hodin), MDB-mf0-synth (multižánrový, 4,7 hodin), GuitarSet (kytara, 3 hodiny) a URMP (komorní orchestr, 1,3 hodiny). I když jde o úlohu, která je lépe definovaná (na rozdíl od extrakce melodie zde nehraje roli subjektivita volby hlavního hlasu), s použitím polyfonních nástrojů vyvstává problém náročné ruční anotace.

Vytváření nových datasetů je obecně velmi pracné a nákladné. Obvyklý postup totiž zahrnuje buď kompletní ruční přepis nahrávky nebo alespoň ruční opravu výstupu automatického přepisu jednohlasých nahrávek, přičemž tuto práci odvedou kvalitně pouze zaškolení hudebníci. Každá vzniklá anotace se také musí překontrolovat, a to nejlépe jiným hudebníkem. Dalším problémem je vůbec identifikace melodie - jelikož je určení hlavní melodické linie subjektivní, musí se na výsledné anotaci shodnout co nejvíce posluchačů. Ve výsledku se proto do datasetů buď vybírají takové nahrávky, které nejsou sporné, nebo na každé anotaci pracuje celý tým, který melodii společně určí. S tím souvisí také zavedení a pečlivé dodržování anotační politky u komplexnějších skladeb (například orchestrálních), kde může melodii nést více hlasů zároveň současně či střídajíc se. Také množství výchozích dat pro vznik datasetů není velké. Jednak musí být skladby šiřitelné, pokud má být dataset volně dostupný a jednak by k nim měly být dostupné \emph{audio stopy} (nahrávky samostatných hlasů), ze kterých je smíchán finální mix, jelikož ruční anotace finálního mixu je mnohem náročnější než anotace oddělených stop.

Existence dostatenčně velkých datasetů je obecně vzato zásadním předpokladem pro využití metod strojového učení pomocí hlubokých neuronových sítí, zejména pak pro netriviální úlohy, jakou je například přepis melodie, jelikož dovoluje zvětšení celkové kapacity modelu, aniž by docházelo k přeučení. Také pro evaluaci metod, například i v soutěži MIREX, jsou potřeba takové datasety, které dobře reprezentují reálná data, přitom dataset MedleyDB vznikl mimo jiné z důvodu, že stávající datasety nestačily ani pro účel evaluace. 

V následující sekci uvádíme přehled veřejně dostupných dat a jejich společnou strukturu, po této sekci následuje podrobnější popis jednotlivých datasetů.

% Možností řešení nastíněného probému nedostatku dat je více. Přímým řešením by byl návrh metody, která by celý proces vzniku datasetů výrazně ulehčila. O to se snaží článek \cite{Salamon2017} a princip této metody popisuje kapitola \ref{sec:mdb_synth}. 

% Východisek z nastíněného probému nedostatku dat je více. Jeden z nejnadějnějších směrů představuje \cite{Salamon2017}, 


\section{Struktura dostupných dat a jejich přehled}

\begin{table}[h!]

\scalebox{0.68}{%
\centering
    \begin{tabular}{lllllllll}
    \toprule
                      {} & MedleyDB & Orchset & ADC04  & \shortstack[l]{MIREX05\\train}  & MDB-synth & WJAZZD & MIR-1K & RWC \\
    \midrule
        Audio            & Ano        & Ano       & Ano        & Ano        & Ano         & Ne\tablefootnote{Autoři audio poskytují neveřejně pro výzkumné účely}  & Ano & Ano\tablefootnote{Přístup k datasetu je zpoplatněn}      \\
        F0 melodie       & Ano        & Ne      & Ano        & Ano        & Ano         & Ano      & Ano & Ano     \\
        MIDI melodie      & Ne       & Ano       & Ne       & Ne       & Ne        & Ano      & Ne & Ne    \\
        Audio stopy      & Ano \tablefootnote{Část stop obsahuje přeslech ostatních nástrojů, informace o přeslechu je současí metadat každé skladby.}     & Ne      & Ne       & Ne       & Ano         & Ne     & Ano\tablefootnote{Oddělený zpěv a karaoke doprovod} & Ne \\
        Multi-F0         & Ne\tablefootnote{Je dostupný přepis všech znějících melodií, viz Definice 3 v sekci MedleyDB.}        & Ne      & Ne       & Ne       & Ano         & Ne     & Ne  & Ne   \\
        MIDI             & Ne       & Ne      & Ne       & Ne       & Ne        & Ne     & Ne  & Ano   \\
        Priorita stop    & Ano        & Ne      & Ne       & Ne       & Ano         & Ne     & Ne  & Ne   \\
        \shortstack[l]{Informace\\o instrumentaci}    & Ano        & Ano      & Ne       & Ne       & Ano         & Ano     & Ne  & Částečné  \\
        Celková délka    & $7.3\,\rm h$\tablefootnote{$5.59\,\rm h$ s anotací melodie} & $23.4\,\rm m$   & $6.1\,\rm m$     & $6.5\,\rm m$     & $3.19\,\rm h$      & $8.85\,\rm h$   & $2.22\,\rm h$ & ---   \\
        \shortstack[l]{Poměr znějící\\melodie} & 60.9\%   & 93.69\% & 85.7\%   & 63.1\%   & 50.4\%    & 62.8\% &  --- & ---  \\
        Počet nahrávek    & 122\tablefootnote{108 s anotací melodie}   & 64      & 20       & 13       & 65        & 299    & 1000  & 315  \\
        Webová stránka   & \tablefootnote{\url{https://medleydb.weebly.com/}} & \tablefootnote{\url{https://www.upf.edu/web/mtg/orchset}}      & \tablefootnote{\url{http://ismir2004.ismir.net/melody_contest/results.html}}       & \tablefootnote{\url{https://labrosa.ee.columbia.edu/projects/melody/}}       & \tablefootnote{\url{http://synthdatasets.weebly.com/mdb-melody-synth.html}}        & \tablefootnote{\url{https://jazzomat.hfm-weimar.de/}}    & \tablefootnote{\url{https://sites.google.com/site/unvoicedsoundseparation/mir-1k}} & \tablefootnote{\url{https://staff.aist.go.jp/m.goto/RWC-MDB/}} \\
        Žánr    & \shortstack[l]{mnoho-\\žánrový} & klasika & \shortstack[l]{pop,jazz,\\opera,midi} & \shortstack[l]{pop,\\midi} & \shortstack[l]{mnoho-\\žánrový} & jazz & karaoke & \shortstack[l]{pop, jazz\\klasika}  \\
        \shortstack[l]{Účel v této\\práci} & \shortstack[l]{Trénování\\Validace\\Testování} & Testování & Testování  & Testování  & Testování & Testování & Žádný & Žádný \\
    \bottomrule
    \end{tabular}
}%

\caption{Souhrnná tabulka se základními informacemi o veřejně dostupných datasetech.}\label{tab:dataset_summary}
\end{table}
Dataset, který chceme použít pro řešení úlohy extrakci melodie, musí obsahovat soubory se zvukem a k nim příslušící anotace melodie. Standardním formátem zvukových souborů je jedno- nebo vícekanálový formát WAVE, se vzorkovací frekvencí $44\,100\,\rm Hz$. Anotace obsahuje spojitou informaci o znějící fundamentální frekvenci melodie v každém časovém okamžiku nahrávky. Výjimkou je dataset ORCHSET, který neobsahuje přesné anotace fundamentální frekvence melodie, ale pouze frekvence konkrétních not. Tedy frekvence nejsou spojité, nýbrž jsou omezené na přesnost jednoho půltónu (V tabulce \ref{tab:dataset_summary} je tato informace zohledněna řádkem MIDI melodie). Důležitou poznámkou je, že zde nejde o diskretizaci původní spojité křivky, ale opravdu jde o anotaci not, tedy pokud melodii nese nástroj hrající vibrato a svou výškou se dostane nad rozsah jednoho půltónu, v anotaci tato skutečnost není zaznamenána. 

Datasety však mohou obsahovat více informací či audio souborů. Užitečné jsou například přiložené audio stopy, ze kterých je vytvořena výsledná píseň (mix), informace o všech znějících výškách (Multi-F0) nebo notách (MIDI), o melodické prioritě jednotlivých audio stop nebo o instrumentaci skladby.

V tabulce \ref{tab:dataset_summary} nalezneme přehledné shrnutí obsahu všech dostupných datasetů.

% \textcolor{red}{TODO: Motivační obrázek pianoroll a zarovnaného audia}


% pěkný podobný seznam datasetů: https://arxiv.org/pdf/1612.08727.pdf

\section{MedleyDB}

MedleyDB je žánrově rozmanitý dataset obsahující 122 nahrávek, k 108 z nich je dostupná anotace melodie. Kromě té dataset obsahuje také metadata o všech písní s informacemi o žánru a instrumentaci. S celkovou délkou 7.3 hodiny jde o nejdelší volně dostupný dataset, který obsahuje více žánrů hudby. O rozmanitosti svědčí i to, že se v datasetu vyskytuje řada nástrojů mimoevropského původu, a že jen přibližně polovina písní obsahuje zpěv. Na rozdíl od ostatních datasetů jsou nahrávky ve většině případů celé písně, tedy nejde pouze o krátké výňatky, a ke každé jsou poskytnuty audiostopy, ze kterých je vytvořen výsledný mix.
Na základě diskuze, kterou shrnujeme v kapitole o definici melodie, autoři datasetu \cite{Bittner2014} poskytují tři verze anotací, na základě různě obecných definic:

\begin{enumerate}
    \item Základní frekvence nejvýraznějšího melodického hlasu, jehož zdroj zůstává po dobu nahrávky neměnný. \footnote{Tato definice je shodná pro evaluační datasety používané v soutěži MIREX, s výjimkou Orchsetu}
    \item Základní frekvence nejvýraznějšího melodického hlasu, jehož zdroje se mohou měnit.
    \item Základní frekvence všech melodických hlasů, potenciálně pocházejících z více zdrojů.
\end{enumerate}

Ačkoli třetí definice umožňuje, aby v anotaci znělo více melodických linek zároveň, v datasetu se nejedná o kompletní přepis nahrávek (použitelný pro úlohu multi-f0 estimation, tedy pro úplný přepis všech fundamentálních frekvencí znějících tónů), ten autoři neposkytují.

Dataset vznikl obvyklou cestou ruční anotace. Ze shromážděného vícestopého materiálu byly vybrány stopy s potenciálním výskytem melodie, stopy s přeslechem byly předzpracovány pomocí algoritmu pro oddělení hlasu a doprovodu (source-separation) s ručně doladěnými parametry pro každou jednotlivou stopu, následně byla na monofonní stopy spuštěna metoda pYIN pro odhad výšky v monofonních datech (pitch tracker) a výsledné automaticky získané anotace opravilo a vzájemně zkontrolovalo pět anotátorů s hudebním vzděláním. 

\section{Orchset}

Dataset vytvořený týmem \cite{Bosch2016} orientovaný na orchestrální repertoár pocházející z různých historických období včetně 20. století. Obsahuje 64 výňatků délky od 10 do 32 sekund. Výňatky byly vybírány tak, aby obsahovaly zřejmou melodii, dataset tedy obsahuje v porovnání málo pasáží bez melodie (6\% z celkové délky). Vzhledem k komplexitě uvažovaných žánrů autoři vycházejí z kombinace rozšířené definice melodie podle \cite{Bittner2014} a definice \cite{Poliner2007}. Melodii ve výňatcích proto zpravidla nese více hudebních nástrojů (nebo celých sekcí), které se v průběhu střídají, případně mohou části hrát společně v rozdílných oktávách (nebo jiných intervalech, tvoříce tak harmonický doprovod). 

Pro zjištění melodie se v takto vrstveném materiálu autoři uchylují k úplnému základu definice melodie (\cite{Poliner2007}) a nechávají si skupinou čtyř posluchačů výňatky přezpívávat. Tato hrubá data pak autoři sumarizují a odebírají z datasetu ty výňatky, na jejichž melodii se posluchači neshodli. Přezpívané tóny bylo nutné ručně opravit, aby načasováním přesně seděly na výňatek. Lidský hlas také samozřejmě nemá rozsah plného orchestru, proto bylo dalším krokem transponovat anotace tak, aby zněly ve správných oktávách. Zde se opět může vyskytnout problém subjektivity, pokud melodii hrají dva různé nástroje, pouze v jiných oktávách, pak je sporné, který nástroj označit jako hlavní, a v některých případech taková otázka ani nedává příliš smysl. Částečným řešením je zvolit libovolnou anotační politiku a tu konzistentně dodržovat (žádná společná v komunitě MIR neexistuje), v případě Orchsetu byla snaha minimalizovat skoky v melodické kontuře, což zároveň respektuje obecné pozorování, že v melodii se vyskytují mnohem častěji malé skoky mezi tóny (nejčastěji prima a malá/velká sekunda) než větší. Tedy například pokud pasáži hrané ve dvou různých oktávách předcházela pasáž hraná v jedné, anotace obou pasáží lze transponovat do společné oktávy tak, abychom na rozhraní těchto pasáží minimalizovali skok v anotaci.

Dataset obsahuje pouze hrubé anotace tónů melodie, nikoli přesnou základní frekvenci nástroje, který v danou chvíli melodii hraje. Článek o tomto rozhodnutí příliš nediskutuje, vychází ale opět logicky z volby dat. U orchestrálních dat je tento abstraktnější pojem melodie mnohem méně sporný. Pokud hraje melodii sekce nástrojů v unisonu, přesná základní frekvence není dobře definovaná, jelikož se základní frekvence znějících hlasů vzájemně překrývají.

\section{MIREX datasety}

Datasety MIREX05 train a MIR-1K byly vydány jako trénovací data v rámci soutěži MIREX. Jde o malé množství dat, MIREX05 train se skládá z několika anotovaných populárních skladeb a několika syntetizovaných písní z MIDI souborů, MIR-1K obsahuje 1000 úryvků zpěvu s karaoke doprovodem. První dataset používáme jako testovací, druhý vzhledem k dostupnosti jiných, rozmanitějších testovacích dat nepoužíváme. 

Dataset ADC2004, použitý ve stejnojmenné soutěži, která předcházela vzniku MIREXu, byl po konci soutěže zveřejněn včetně testovací množiny, stále je však využíván jako jeden z testovacích datasetů v soutěži MIREX. Celý dataset proto také používáme jako testovací. Obsahuje 20 výňatků ze žánrů popu, jazzu a opery a dále pak 4 syntetické skladby. 

\section{Weimar Jazz Database}

Weimar Jazz Database (práce německého týmu \cite{Pfleiderer}) obsahuje přes 450 transkripcí jazzových sól ze všech období vývoje jazzu. Data původně zamýšlená pro muzikologické studie využívající statistické metody ale lze využít i pro potřeby extrakce melodie, jelikož uvažované nahrávky spadají zřejmě pod nejrestriktivnější definici melodie (definici používanou v soutěži MIREX) - melodii nese jistě právě jeden, sólový nástroj, a po celou dobu výňatku je jistě nejvýraznější. Výběr sólových nástrojů se omezuje pouze na jednohlasé, jelikož ruční anotace vícehlasých je příliš obtížná. Hlavním problémem při využívání je restriktivní licence, která platí na nahrávky, tudíž zdrojové audio, na základě kterého anotace vznikaly, není veřejně přístupné. 

Dataset v této práci používáme pouze pro testování metod, nikoliv trénování.

% Jelikož pro data neexistují jednotlivé stopy, ruční anotace probíhala přímo z finální nahrávky, což je obtížný úkol - 

\section{MDB-synth}\label{sec:mdb_synth}

Hlavním přínosem práce \cite{Salamon2017} je navržení způsobu anotace základní frekvence monofonních audiostop takovým způsobem, že výsledná dvojice zvukové stopy a anotace nevyžaduje další manuální kontrolu. Anotace monofonní stopy probíhá ve dvou krocích: nejprve získáme pomocí libovolné metody přepisu jednohlasu křivku základní frekvence. Poté na základě této křivky, která může obsahovat chybně anotované úseky, syntetizujeme novou stopu, která zachovává barvu nahrávky, ale výšku tónu určuje právě tato automaticky získaná anotace. Díky tomu je pak přesnost anotace pro tuto novou, syntetickou nahrávku stoprocentní, přitom (v ideálním případě) neztrácí charakteristiky původní nahrávky.

Pro vytváření datasetu je toto významné zjednodušení, protože tím algoritmus odstraňuje časově nejnáročnější část práce --- ruční kontrolu anotací audiostop. Pokud by se ukázalo, že syntéza významně neubírá na kvalitě dat, použitím navrhované metody by mohlo vzniknout velké množství nových dat (napříkad repozitář Open Multitrack Testbed obsahuje stovky vícestopých nahrávek, které by bylo možné využít). Autoři v článku provádí kvantitativní analýzu pomocí srovnání state-of-the-art algoritmů pro extrakci melodie a prokazují, že výsledky těchto metod na syntetických datech se významně neliší od výsledků na původních, tím je podle autorů potvrzená možnost použití dat jak pro trénování tak pro evaluaci nových metod.

Metoda má ale bohužel svá omezení. Mezi ty zásadní patří, že se dá aplikovat pouze na stopy, které obsahují monofonní signál, vstupní data tedy nesmí obsahovat přeslech a nahrávaný nástroj může hrát pouze jednohlas. V důsledku tedy nelze zpracovat například klavír či kytara, které hrají zpravidla vícehlas. Pro generování datasetu určeného pro přepis melodie tato limitace není zásadní, jelikož melodii často hraje jeden hlas a doprovod může být vícehlasý. Problémem je  spíše generování datasetů pro úlohu kompletního přepisu (multi-f0 estimation).

Bohužel také k článku není zveřejněná kompletní refereční implementace algoritmu, tudíž algoritmus nelze snadno spustit na nových datech. Ve výsledku je tudíž největším praktickým přínosem nová sada syntetických dat pro úlohy přepisu melodie, přepisu basy, přepisu jednohlasu a kompletního přepisu. Každý z datasetů určených pro zmíněné úlohy obsahuje destíky nahrávek. Vícestopá data použitá pro syntézu byla převzata z MedleyDB, tudíž nové datasety nerozšiřují celkový hudební záběr, pouze zpřesňují již existující.

Dataset v této práci používáme pouze pro testování metod, nikoliv trénování.

% \textcolor{red}{TODO obrázek? Porovnání spektrogramů syntetické a původní nahrávky}

% Z kvalitativního pohledu je na výstupních syntetických nahrávkách poznat, že jsou syntetické. Autoři sice prokazují, že současné metody na těchto datech dosahují stejných výsledků, nicméně v článku chybí diskuse o tom, zda-li v datech algoritmus nevytváří nové umělé artefakty, které by mohly zneužít metody strojového učení pro spolehlivější výsledky (které by však negeneralizovaly na reálná data). Při pohledu na spektrogram je například zřejmé, že syntetická nahrávka obsahuje mnohem více výrazných alikvótních frekvencí

\section{Dataset RWC}

Dataset RWC (práce \cite{Goto2002}) je první vzniklá rozsáhlá kolekce dat určená pro úlohy Music Information Retrieval, mezi které v tomto případě patří i extrakce melodie. Dataset obsahuje 100 popových, 50 orchestrálních a 50 jazzových skladeb. Přístup k datasetu RWC je však zpoplatněn, proto ho v této práci nepoužíváme.

\chapter{Evaluace metod}

\section{MIREX}

Soutěž MIREX (Music Information Retrieval Evaluation eXchange) probíhá již od roku 2005 a v MIR komunitě zastává hlavní postavení jakožto každoroční událost pro nezávislé, objektivní srovnání state-of-the-art metod a algoritmů pro řešení širokého spektra úloh souvisejících se zpracováním hudebních dat. Mezi tyto úlohy patří například \textit{rozpoznání žánru}, \textit{odhad tempa}, \textit{odhad akordů}, \textit{identifikace coveru} a samozřejmě také \textit{extrakce melodie}.

Na rozdíl od jiných úloh, kde debata o zvolení nejvhodnějších objektivních metrik pro porovnávání stále probíhá, metriky pro extrakci melodie se ustanovily již v prvním ročníku (na základě dřívějších zkušeností) a zůstaly neměnné dodnes \cite{Raffel2014}. Naopak data použitá pro testování se postupně kumulují a dnes soutěž probíhá již s řadou datasetů (ADC04, MIREX05, MIREX08, MIREX09, ORCHSET), které blíže popisuje kapitola o dostupných datech.

\cite{Downie2010}

\section{Trénovací, validační a testovací množina}

Z dostupných dat, které pro úlohu máme k dispozici, musíme vyhradit množiny pro trénování, validaci a testování, aby byly metody porovnatelné jak mezi sebou, tak se stávajícími state-of-the-art metodami. Pro trénování se jeví jako nejvhodnější dataset MedleyDB, jednak pro svou délku a jednak pro žánrovou rozmanitost, proto je použit pro většinu popsaných experimentů. Rozdělení na tři části vychází z práce \cite{Bittner2017} a \cite{DBasaranSEssid2018}, aby byly metriky přímo porovnatelné s výsledky v uvedených článcích. Další výhodou použití stejného \textit{splitu} je možnost replikace výsledků, za použití popisované architektury, a tím pádem minimalizování možnosti nějaké velké implementační chyby v kódu. Pokud by se totiž výsledky nepodařilo replikovat se stejnými daty i architekturou, musela by být chyba jinde - tedy s největší určitostí v vyvinutém frameworku.

Dalším užitečným zdrojem dat je dataset \textit{MDB-melody-synth}, který je přesyntetizován z vícestopých nahrávek \textit{MedleyDB}, proto se nabízí použít stejné rozdělení dat, jaké se používá pro \textit{MedleyDB}, ze stejných důvodů uvedených v předchozím odstavci. Jelikož dataset neobsahuje veškerá data, ale pouze jejich podmnožinu, i v experimentech používaný \textit{split} obsahuje pouze podmnožinu z původního \textit{splitu} datasetu \textit{MedleyDB}. 

Posledním velkým datasetem, používaným pro trénování, je \textit{Weimar Jazz Database}. Zde žádný doporučený postup ani výběr rozdělení dataestu v relevatní literatuře neexistuje, proto jsem dataset rozdělil podle metody \cite{Bittner2017} na tři části (v celkové délce nahrávek na části v poměrech 63\%, 14\% a 23\%). Skladby jsou rozděleny do částí podle interpretů tak, aby se každý interpret vyskytoval právě v jedné části datasetu. Toto omezení na podmnožiny \cite{Bittner2017} nediskutuje, lze však doložit (práce \cite{Sturm2013}), že pro úlohu \textit{rozpoznání žánru} metody založené na strojovém učení vykazují po trénování a validaci na datech bez tohoto filtru výrazně lepší výsledky než stejné metody spuštěné na roztříděných datech, takové zlepšení výkonu je ale jistě umělým důsledkem špatné volby trénovací množiny. 

Ostatní datasety (ADC04, MIREX05, ORCHSET) jsou v práci použity pouze jako testovací data, díky tomu lze korektně výsledky přímo srovnávat s žebříčky úlohy Melody Extraction v soutěži MIREX.

% ------

% - MatthewEntwistle_FairerHopes
% obsahuje harfu, ale trénovací data ji neobsahují, chce to ale víc prozkoumat, jelikož trénovací data neobsahují víc nástrojů, tak zjistit přesně přesnost anotace pro tyto nástroje

\section{Kvalitativní příklady}

Pro lepší porozumění hranic testovaných metod je vhodné studovat také výsledky na kvalitativních ukázkách. Modely byly při práci vyhodnocovány na několikaminutových množinách výňatků z validačních a testovacích dat. Metodika výběru spočívala v poslechu nahrávek a ručním hledáním zajímavých hudebních jevů a také v seřazení nahrávek podle úspěšnosti přepisu stávajícími metodami a výběrem výňatků právě z těchto nejproblematičtějších příkladů.

Omezení plynoucí z potřeby zkrátit výňatky na minimum,


\section{Metriky}

Celkovou kvalitu metody pro extrakci melodie určuje její schopnost určit výšku tónu hrající melodie (\textit{odhad výšky melodie}) a také rozpoznat části skladby, které melodii neobsahují (\textit{detekce melodie}). Jelikož jsou tyto podúlohy na sobě nezávislé, standardní sada metrik zahrnuje jak celkové vyhodnocení přesnosti, tak dílčí vyhodnocení pro \textit{odhad výšky} a \textit{detekci melodie}. 

% -------

% - můžu zmínit to, že je toto rozdělení důležité pro Orchset, který je z většiny voiced, a tedy overall accuracy může být zavádějící u algoritmů s přísným voicing detection.
% - celkové skóre na datasetu je průměr všech písní

\subsection{Formát výstupu}

Obvyklý formát výstupu algoritmů je CSV soubor se dvěma sloupci. První sloupec obsahuje pravidelné časové značky, druhý sloupec pak odhad základní frekvence melodie. Některé algoritmy uvádí i odhady výšky základní frekvence mimo detekovanou melodii (může jít například o doprovod, který zní i po hlavním melodickém hlasu). Aby tyto odhady byly odlišené od odhadů hlavní melodie, jsou uvedeny v záporných hodnotách. Díky tomu pak lze nezávisle vyhodnotit přesnost \textit{odhadu výšky} a \textit{detekce melodie}. Odhad výšky se vyhodnocuje podle absolutní hodnoty frekvence ve všech časových oknech, ke kterým existuje anotace, detekce melodie pak na všech hodnotách vyšších než 0. 

\subsection{Definice metrik}

Většina metrik je definována na základě porovnávání jednotlivých anotačních oken - tedy typicky srovnáním odhadovaných a pravdivých výšek melodie po konstantních časových skocích. Datasety používané pro vyhodnocování v soutěži MIREX používají časový skok délky 10 ms. V definicích budu vycházet ze značení v práci \cite{Salamon2014}. 

    Označme vektor odhadovaných základních frekvencí $\mathbf{f}$ a cílový vektor $\mathbf{f^*}$, složka $f_\tau$ je buď rovna hodnotě $f_0$ melodie nebo $0$, pokud v daném čase melodie nezní. Obdobně zaveďme vektor indikátorů $\mathbf{v}$, jehož prvek na pozici $\tau$ je roven $v_\tau=1$, pokud je v daném časovém okamžiku detekována melodie a $v_\tau = 0$ v opačném případě. Podobným způsobem zavedeme i vektor cílových indikátorů melodického hlasu $\mathbf{v^*}$ a také vektor indikátorů absence melodie $\bar{v}_\tau = 1 - v_\tau$. 

\subsubsection{"Úplnost detekce" = Voicing Recall rate}

Poměr počtu časových oken, které byly správně označené jakožto obsahující melodii, a počtu časových oken doopravdy obsahujících melodii podle anotace.

    $$\mathrm{VR}(\mathbf{v}, \mathbf{v^*}) = \frac{\sum_\tau{v_\tau v^*_\tau}}{\sum_\tau{v^*_\tau}}$$

% ------

% The proportion of frames labeled as melody frames in the ground truth that are estimated as melody frames by the algorithm.

\subsubsection{"Nesprávné detekce" = Voicing False Alarm rate}

Poměr počtu časových oken, které byly nesprávně označené jako melodické, k počtu doopravdy nemelodických oken.

    $$\mathrm{FA}(\mathbf{v}, \mathbf{v^*}) = \frac{\sum_\tau{v_\tau \bar{v}^*_\tau}}{\sum_\tau{\bar{v}^*_\tau}}$$

% -------

% The proportion of frames labeled as non-melody in the ground truth that are mis- takenly estimated as melody frames by the algorithm.

\subsubsection{"Přesnost odhadu tónu" = Raw Pitch Accuracy}

Poměr správně odhadnutých tónů k celkovému počtu melodických oken. Výška správně určeného tónu se může lišit až o jeden půltón.


    $$\mathrm{RPA}(\mathbf{f}, \mathbf{f^*}) = \frac{\sum_\tau{v^*_\tau v_\tau \mathcal{T}[\mathcal{M}(f_\tau) - \mathcal{M}(f^*_\tau)}] }{\sum_\tau{v^*_\tau}}$$

kde $\mathcal{T}$ je prahová funkce

    \begin{equation*}
        \mathcal{T}[a] = \begin{cases}
                1 & \mathrm{pro} \lvert |a| \le 0.5 \\
                0 & \text{jinak}
                
            \end{cases}
    \end{equation*}

a $\mathcal{M}$ je funkce zobrazující frekvenci $f$ na reálné číslo počtu půltónů od nějakého referenčního tónu $f_{\mathrm{ref}}$ (například od 440 Hz, tedy komorního A4).

    $$\mathcal{M}(f) = 12 \log_2(\frac{f}{f_{\mathrm{ref}}})$$


% -------

% The proportion of melody frames in the ground truth for which f_τ is considered correct

% Raw Pitch Accuracy: The proportion of melody frames in the ground truth for which fτ is considered correct
% (i.e. within half a semitone of the ground truth f∗τ ).

\subsubsection{"Přesnost odhadu tónu nezávisle na oktávě" = Raw Chroma Accuracy}

Počítá se podobně jako \textit{Přesnost odhadu tónu}, výstupní a cílové tóny jsou však mapovány na společnou oktávu. Metrika tedy ignoruje chyby odhadu způsobené špatným určením oktávy tónu.

    $$\mathrm{RCA}(\mathbf{f}, \mathbf{f^*}) = \frac{\sum_\tau{v^*_\tau v_\tau \mathcal{T}[\langle \mathcal{M}(f_\tau) - \mathcal{M}(f^*_\tau)} \rangle_{12}] }{\sum_\tau{v^*_\tau}}$$

Nezávislost na oktávě zajistíme pomocí zobrazení rozdílu cílového a výstupního tónu na společnou oktávu.

    $$\langle a \rangle_{12} = a - 12 \lfloor \frac{a}{12} + 0.5 \rfloor  $$

% -------

% As raw pitch accuracy, except that both the estimated and ground truth f0 sequences are mapped onto a single octave.


\subsubsection{"Celková přesnost" = Overall Accuracy}

Celková přesnost měří výkon algoritmu jak v odhadu melodie tak v detekci melodie. Počítá se jako podíl správně odhadnutých oken a celkového počtu oken.

    $$\mathrm{OA}(\mathbf{f}, \mathbf{f^*}) = \frac{\sum_\tau{v^*_\tau v_\tau \mathcal{T}[\mathcal{M}(f_\tau) - \mathcal{M}(f^*_\tau)}] + \bar{v}^*_\tau \bar{v}_\tau }{L}$$

% ---------

% this measure combines the perfor- mance of the pitch estimation and voicing detection tasks to give an overall performance score for the system.

\subsubsection{Poznámka k definicím metrik}

Definice RPA, RCA a OA zde uvedené se mírně liší od výchozích v práci \cite{Salamon2014}, jejich přímá implementace podle vzorce totiž vede kvůli nedostatečně dobře zadefinovanému vektoru frekvencí $\mathbf{f}$ k chybě, která byla přítomna i v nejpoužívanější, veřejné implementaci MIR metrik \textit{mir\_eval}. Tato chyba se týká zejména metriky RCA, která v původní definici chybně zahrnovala jako správné tóny ty, které algoritmus odhadl jako nulové (tedy neznějící) a zároveň jejich pravdivá hodnota byla po zobrazení na jednu společnou oktávu blízká nule (tedy původní tón byl blízký nějakému násobku referenční frekvence). Kvůli zobrazení na společnou oktávu se stanou "neznělé nulové odhady" a tóny blízké referenčním frekvencím nerozlišitelné a byly nesprávně považované za korektní.

V praxi chyba této metriky na datasetu MedleyDB mohla dosahovat až sedmi procentních bodů, na repozitáři hostovaném na serveru Github jsme již spolu s autory chybu odstranili \footnote{odkaz na Github issue: \url{https://github.com/craffel/mir_eval/issues/311}}. Opravný patch bude zahrnut do další verze balíku.

\subsection{Další metriky}

Protože princip vnitřního fungování neuronových sítí často není zřejmý, je užitečné mít co nejvíce různých indikátorů, abychom měli při porovnávání jednotlivých modelů alespoň podrobnou informaci, v jakých ohledech se síť zlepšuje nebo zhoršuje. Pro tento účel jsem při práci implementoval další metriky, které při hledání architektur sítí pomáhaly.

\subsubsection{Chroma Overall Accuracy}

Počítá se obdobně jako Overall Accuracy, ale tóny jsou mapovány na společnou oktávu.

\subsubsection{Raw Harmonic Accuracy}

Metrika počítá odhadovaný tón jako správný, pokud se trefil do některé z harmonických frekvencí tónu. Protože je harmonických frekvencí teoreticky nekonečné množství, parametrem metriky je do jakého celočíselného násobku se ještě odhad počítá.

    $$\mathrm{RHA}(\mathbf{f}, \mathbf{f^*}, n) = \frac{\sum_{k=1}^n \sum_\tau{v^*_\tau v_\tau \mathcal{T}[\mathcal{M}(f_\tau) - \mathcal{M}(k f^*_\tau)} ] }{\sum_\tau{v^*_\tau}}$$

\subsubsection{Matice záměn not}

Pro podrobnější souhrnný přehled četností chyb se pro klasifikační úlohy používá matice záměn. Sloupce označují správné noty, řádky odhadované. Buňka na pozici $(x,y)$ má pak hodnotu podle četnosti odhadu noty $y$ místo správné noty $x$.

% \subsubsection{Histogram vzdáleností odhadu}

% Histogram hodnot rozdílu $\mathbf{f} - \mathbf{f^*}$, 

% - confusion matrix
% - estimation distance histogram
% - pitch accuracy per note

% \subsection{Limitace základních metrik}
% - limitace jsou předvedeny v onsets+frames
%     - nakonec nejsou, tam kritizují jenom 
% - například je otázka, jestli jsou všechny framy stejně důležité - zejména u vybrnkávání, piana, perkusí je otázka, kdy ještě anotovat, tedy jsou tam sporné konce. Na small_valid je to hodně vidět na té harfě
% - nijak se nepenalizuje nekontinualita výstupů, je rozdíl mezi 50\% accuracy, kde je zbytek unvoiced a 50\% accuracy, kde odhady strašně skáčou

% - Bosch metrics \cite{Bosch2016}
%     - Weighted Raw Chroma accuracy - počítá vzdálenost v oktávách
%     - Octave Jumps - vyjadřuje skokovitost o oktávy v po sobě následujících framech v rámci správných chroma odhadů
%     - Chroma continuity - 

% \section{Kvalitativní}

% - popsat můj small_validation

% - ilustrační příklady !!!
% 	- orchestrální i neorchestrální
% 		- metody z related work fungují na neorch.
% 	- jeden hlas
% 	- melodie nahoře (zkusit vybrat extrém ~ np.max(annotations))
% 	- melodie dole
% 	- melodie uprostřed

% 	- vlastnosti melodie
% 		- stabilní dlouhý tóny (a kolem doprovod)
% 		- něco proměnlivého
% 	- potichu/nahlas

\chapter{Experimenty}

Práce obsahuje souhrnné výsledky experimentů zejména nad datasetem MedleyDB, aby modely byly dobře porovnatelné se state-of-the-art výsledky a výhoda prezentovaných metod netkvěla pouze v použití více dat. U vybraných experimentů došlo k přetrénování na větší trénovací množině, aby bylo možné posoudit vliv množství dat na výsledný výkon. 

V první části se zabývám zejména odhadováním *výšky tónů*. U úspěšných architektur pak implementuji i *detekci melodie*.

\section{Architektura CREPE}

První sada experimentů se zakládá na architektuře popsané v článku od \cite{Kim2018} použité pro *monopitch tracking*. Přestože se nejedná o úlohu extrakce melodie, cílem monopitch trackingu je určit konturu základní frekvence melodického nástroje v monofonní nahrávce, která se skládá ze součtu čistého signálu a šumu v pozadí. Pokud rozšíříme pojem šumu v pozadí tak, aby zahrnoval i melodický doprovod, pak dostáváme formální definici signálu zpracovávaného algoritmy pro přepis melodie \cite{Salamon2014}.

Jinými slovy - *monopitch tracking* je speciálním případem extrakce melodie a tudíž přinejmenším stojí za zkoušku pokusit se tuto architekturu pro extrakci využít. Mimo to monofonní stopy často obsahují přeslech ostatních nástrojů, pokud nahrávka vznikala při společném hraní, tudíž by model trénovaný na výsledných mixech mohl být robustní vůči tomuto druhu rušení. 

Architektura CREPE se sestává ze šesti konvolučních a pooling vrstev, pro regularizaci používá batch normalization a dropout po každé konvoluční vrstvě, jako aktivační funkce používá ReLU. Po konvolucích následuje výstupní plně propojená vrstva se sigmoid aktivací. Vstupem modelu je okno o velikosti 1024 samplů, audio je převzorkováno na 16 kHz. Před první konvolucí je vstup normalizován tak, aby každé jednotlivé okno se vzorky mělo střední hodnotu 0 a směrodatnou odchylku 1. Přesná podoba modelu je naznačena na obrázku.

Výsledný vektor o 640 složkách aproximuje pravděpodobnostní rozdělení výšky základní frekvence uprostřed vstupního okna, přičemž tento vektor pokrývá rozsah od noty $C_{-1}$ po $G_{9}$, mezi dvěma sousedními predikovanými tóny je vzdálenost 20 centů. Výšky tónů v centech označíme $\cent_1, \cent_2, \dots, \cent_{640}$. Rozsah tedy bezpečně pokrývá obvyklé hudební nástroje a na jednu notu připadá 5 složek (tónů) výsledného vektoru.

    $$\cent(f) = 1200 \log_2{\frac{f}{f_{\mathrm{ref}}}}$$

Pro trénování modelu potřebujeme také cílové diskrétní pravděpodobnostní rozdělení základní frekvence tónu. Jako cílovou pravděpodobnostní funkci použijeme normální rozdělení se střední hodnotou v bodě cílové základní frekvence $\cent(f_{\mathrm{ref}})$ a se směrodatnou odchylkou 25 centů. Toto rozdělení dikretizujeme, aby měl cílový vektor stejné dimenze jako odhadovaný.

    $$y_i = \frac{1}{\sqrt{2 \pi \sigma^2}}\exp{(-\frac{(\cent_i - \cent_{\mathrm{ref}})^2}{2 \sigma^2})}$$

Převod z výstupního vektoru na výšky not provedeme pomocí střední hodnoty výstupního vektoru. Jelikož by ale výšku tónu ovlivňoval i další melodický šum, který se na výstupním vektoru také objevuje, spočítáme střední hodnotu pouze z okolí maxima výstupu.

    $$ \left. \hat{\cent} = \sum_{\scaleto{i, \lvert \cent_i - \cent_m \rvert < 50}{8pt}} {\hat{y}_i \cent_i} \middle/ \sum_{\scaleto{i, \lvert \cent_i - \cent_m \rvert < 50}{8pt}} \hat{y}_i \right., m = \mathrm{argmax}_i(\hat{y}_i)$$

Optimalizovaná loss funkce modelu $\mathcal{L}(\mathbf{y}, \mathbf{\hat{y}})$ se počítá jako binární vzájemná korelace mezi vektorem cílových pravděpodobností $y$ a výstupním vektorem $\hat{y}$.

    $$\mathcal{L}(\mathbf{y}, \mathbf{\hat{y}}) = \sum_{i = 1}^{640}{(-y_i\log\hat{y}_i - (1-y_i)\log(1-\hat{y_i}))}$$

Optimalizace probíhá pomocí algoritmu Adam \citep{Kingma2014} s learning rate 0.0002.

% TODO: Přidat obrázek modelu (draw.io)


% rozepsat:
% - obhajoba raw signálu

% diskuze:
% - převzorkování na 16kHz
% - normalizace vstupu
% - formulace jako klasifikační úloha, nikoli regresní
% - je lepší odhadovat opravdové pravděpodobnostní rozdělení a nebo jejich škálované? (přijde mi, že kvůli sigmoid aktivaci bude jednodušší 1.0 = Truth, protože ty vstupní logity do sigmoid aktivace můžou být crazyshit velký)
% - crepe model - např. nedává vůbec smysl velikost kernelu 64 v posledních vrstvách, zbytečně se tam přidávají nuly jako padding


\subsection{Replikace výsledků CREPE}


Pro ověření správnosti implementace architektury *monopitch trackeru CREPE* spustíme model na syntetických, monofonních datech používaných v článku \cite{Salamon2017}. Na rozdíl od článku \cite{Kim2018} jsem model netrénoval na všech datech pomocí postupu *5 fold cross validation*, jiné zásadní rozdíly mezi implementacemi jsem však na základě článku a veřejně dostupného kódu neidentifikoval.

Po jedné epoše trénování model dosáhl vyšší přesnosti, než je uváděná v literatuře, tento rozdíl přičítám zejména zmiňované odlišné evaluační strategii.

    \begin{tabular}{llrr}
    \toprule
    Metrika & Práh & Průměrná hodnota & Hodnota \cite{Kim2018} \\
    \midrule
    RCA & 50 centů & 0.988 & 0.970 \\
    RPA  & 50 centů & 0.986 & 0.967 \\
    RPA  & 25 centů & 0.975 & 0.953 \\
    RPA  & 10 centů & 0.937 & 0.909 \\
    \bottomrule
    \end{tabular}

Při replikaci experimentu jsem narazil na důležitost správného promíchání dat. Framework Tensorflow použitý pro trénování promíchává data vždy pomocí bufferu pevné velikosti pro dvojice vstupů a cílových výstupů. V praxi je však potřeba buď nastavit buffer na velikost větší než je celková velikost datasetu, a nebo implementovat vlastní míchání přes všechna dostupná data. Při nedostatečně promíchaných datech totiž trénovací dávky (batch) nejsou reprezentativní pro celý dataset, ale pouze pro jeho podmnožinu, což se negativně projevuje kolísající validační přesností modelu.

\subsection{CREPE pro extrakci melodie}

Jako první experiment nad melodickými daty spustíme nezměněnou architekturu CREPE, v následujících experimentech se tuto baseline pokusíme překonat. Abychom urychlili trénování následujících experimentů, přesnost určíme pro sítě s různou kapacitou, pokud se výsledky při různých kapacitách příliš neliší, můžeme experimenty provádět s architekturou s nižší kapacitou. Kapacity upravíme pomocí multiplikátoru počtu filtrů u všech konvolučních vrstev, počty filtrů jsou uvedeny v tabulce.


    \begin{tabular}{lrrrrrrr}
    \toprule
    Vrstva     &  1.   &  2.  &  3.  &  4.  &  5.   &  6.  &  Celkový počet parametrů \\
    \midrule
    CREPE 4x   &  128  &  16  &  16  &  16  &  32   &  64  &  558240 \\
    CREPE 8x   &  256  &  32  &  32  &  32  &  64   &  128 &  1771200 \\
    CREPE 16x  &  512  &  64  &  64  &  64  &  128  &  256 &  6163200 \\
    \bottomrule
    \end{tabular}


    \begin{tabular}{lrr}
    \toprule
    Model      &  RPA    &  RCA \\
    \midrule
    CREPE 4x   &  0.634  &  0.753 \\
    CREPE 8x   &  0.661  &  0.766 \\
    CREPE 16x  &  0.666  &  0.771 \\
    Salamon    &  0.547  &  0.608 \\
    Bittner    &  0.735  &  0.791 \\
    Basaran    &  0.737  &  0.803 \\
    \bottomrule
    \end{tabular}

Z výsledků na validačních datech po 200k iteracích (přibližně 6 epoch) je zřejmé, že překonání state-of-the-art metod založených na pravidlovém zpracování zvuku \citep{salamon2012musical} není obtížné. Zároveň také vidíme, že se výsledek modelů CREPE 8x a CREPE 16x liší řádově o desetiny procentních bodů a přitom model s větší kapacitou se trénuje o 35% delší dobu. Proto pro další experimenty zvolíme architektury s multiplikátorem 8x a případně přetrénujeme s vyšší kapacitou pouze nadějné konfigurace.


\subsection{Vliv rozlišení diskretizace výšky noty}

Otestujeme nastavení granularity výstupního vektoru. V článku \cite{Kim2018} se totiž důvod volby pěti frekvencí na notu nediskutuje. Intuitivně by však mělo vyšší rozlišení spíše pomáhat, důvodem je, že nástroje a zejména lidský hlas se často při hraní odchylují od přesných frekvencí hraných not a vyšší rozlišení tyto odchylky může lépe zachytit.

    \begin{tabular}{llrr}
    \toprule
    Kapacita & Diskretizace &  RPA &  RCA \\
    \midrule
     4x &        hrubá      & 0.606 & 0.708 \\
     4x &        jemná      & 0.634 & 0.753 \\
     8x &        hrubá      & 0.614 & 0.724 \\
     8x &        jemná      & 0.661 & 0.766 \\
    16x &        hrubá      & 0.612 & 0.711 \\
    16x &        jemná      & 0.666 & 0.771 \\
    \bottomrule
    \end{tabular}

% TODO: Přidat graf

Jak je vidět z tabulky a grafů, jemná granularita výstupu jednoznačně zlepšuje přesnost sítě. Abychom potvrdili hypotézu, že vyšší rozlišení pomáhá zmenšit počet chyb o půltón, můžeme vytvořit histogram vzdáleností cílového a odhadovaného tónu, v tomto histogramu by pak měl být vidět pokles v příslušných třídách.

% TODO: Přidat histogramy

Podle histogramu se počet chyb o půltón mezi zkoumanými modely liší téměř o polovinu, zlepšení tohoto druhu chyb je tedy podstatné.

\subsection{Vliv rozptylu cílové pravděpodobnostní distribuce výšky noty}

Podle \cite{Bittner2017} pomáhá cílová distribuce s vyšším rozptylem snížit penalizaci sítě za téměř korektní odhady výšek tónů. Mimo to u dostupných dat často nejsou anotace naprosto perfektní, jisté rozostření hranice anotace tudíž pomáhá i v případě nepřesné cílové anotace, síť pak není tolik penalizována za svou případnou správnou odpověď. 

V článku se však nediskutuje nastavení směrodatné odchylky na 20 centů, \cite{Kim2018} používá odchylku 25 centů a není na první pohled zřejmé, jaká je optimální hodnota. Příliš vysoký rozptyl způsobí, že síť bude tolerovat více chyb o půltón, příliš nízký rozptyl naopak penalizuje i téměř správné odhady. Intuitivně se nejlepší nastavení pravděpodobně bude pohybovat kolem používaných 25 centů, jelikož to je hranice chybné klasifikace, na druhou stranu optimální hodnota jistě bude závislá na nastavení rozlišení výstupního vektoru, jelikož nižší rozlišení bude jistě vyžadovat vyšší hodnotu rozptylu (v extrémním případě rozptylu blížícího se k nule a cílové frekvence mimo kvantizační hladiny by vzniklý cílový vektor nemusel obsahovat žádné ostré maximum).

Poznamenám také technický detail, který je důležitý při samotné implementaci. Přestože jsem cílový výstup sítě zadefinoval jako diskrétní pravděpodobnostní rozdělení, při trénování je tento vektor hodnot pronásoben koeficientem tak, aby $\max(\mathbf{y}) = 1.0$ a tedy součet prvků vektoru není roven jedné (a o pravděpodobnostní rozdělení se doopravdy nejedná). Důvodem je použití aktivační funkce *sigmoid* u výstupní vrstvy, která nezaručuje výstup korektního rozdělení. Díky tomu se na výstupu může objevit různé množství stejně pravděpodobných kandidátů na melodii.

Testovaná síť má vstupní okno široké 4096 vzorků, používá multiplikátor kapacity 16x a vstup zpracovává 6 různě širokými konvolučními vrstvami (viz experiment *Vliv násobného rozlišení první konvoluční vrstvy*).

    \begin{tabular}{lrr}
    \toprule
    Směrod. &  Raw Pitch Accuracy &  Raw Chroma Accuracy \\
    \midrule
    0.000   &               0.657 &                0.759 \\
    0.088   &               0.672 &                0.775 \\
    0.177   &               0.689 &                0.784 \\
    0.354   &               0.669 &                0.773 \\
    0.707   &               0.654 &                0.757 \\
    \bottomrule
    \end{tabular}

Z experimentů vyplývá, že optimální směrodatná odchylka se pohybuje kolem hodnoty $0.177$, tedy níže než v porovnávaných pracích. 

% ------
% - cílová distribuce doopravdy není distribuce
% - ty zvláštní testované směrod. odchylky jsou kvůli mé chybné implementaci rozostřování
% - zde můžu přidat obrázek, jak vypadají anotace
%     mám to rozpracované na: http://jirkabalhar.cz:6088/notebooks/bakalarka/algoritmy/ismir2017-deepsalience/deepsalience/out/io_comparison.ipynb#

\subsection{Vliv šířky vstupního okna}

Architektura CREPE byla navržena pro monopitch tracking, dá se předpokládat, že jelikož je v monofonních nahrávkách oproti polyfonním daleko méně (melodického) šumu, není pro určení výšky tónu potřeba větší kontext než použitých 1024 vzorků (při vzorkovací frekvenci 16kHz toto odpovídá 64 milisekundám audia). To ale nemusí platit pro složitější signály, kde by síť mohla z delšího kontextu těžit. Otestujeme tedy vliv většího vstupního okna na výslednou přesnost.

    \begin{tabular}{lrr}
    \toprule
    Šířka vstupního okna &  Raw Pitch Accuracy &  Raw Chroma Accuracy \\
    \midrule
    512 (32 ms)          &               0.634 &                0.748 \\
    1024 (64 ms)         &               0.645 &                0.763 \\
    2048 (128 ms)        &               0.648 &                0.760 \\
    4096 (256 ms)        &               0.650 &                0.762 \\
    8192 (512 ms)        &               0.675 &                0.775 \\
    \bottomrule
    \end{tabular}

% ------

% TODO: možná by to chtělo taky přetrénovat

% - širší okno se také hodí pro onsety a offsety

\subsection{Vliv násobného rozlišení první konvoluční vrstvy}

Podle \cite{Kim2018} se přesnost CREPE snižuje s výškou tónu. Autoři si tuto skutečnost vysvětlují neschopností modelu generalizovat na barvy a výšky tónů neobsažených v trénovací množině, generalizaci by ale mohla pomoci také úprava modelu. Protože k rozpoznání vyšších frekvencí stačí méně vzorků než pro rozpoznání nižších, mohli bychom se pokusit upravit první konvoluční vrstvu sítě, která tento úkol zastává, a rozdělit ji na množiny různě širokých konvolucí, jejichž kanály následně sloučíme zpět do jednotné vrstvy. To by mělo mít za následek, že rozpoznávání vysokých tónů budou zastávat užší konvoluce a jejich kernel bude jednodušší než široké kernely s vysokou mírou redundance.

První vrstvu s kernelem s 256 filtry (tj. počet filtrů první vrstvy s multiplikátorem 8x, viz první experiment) jsem rozdělil na vícero různě širokých kernelů s menším počtem filtrů, tak aby kapacita sítě zůstala přibližně stejná a sítě byly porovnatelné. 


    \begin{tabular}{lrrrrrrrrr}
    \toprule
    Počet/šířka kernelů & 512 & 256 & 128 & 64 & 32 & 16 & 8  & 4  & Celkový počet parametrů  \\
    \midrule
    1                   & 256 &     &     &    &    &    &    &    & 2098880 \\
    2                   & 128 & 128 &     &    &    &    &    &    & 2066112 \\
    3                   & 85  & 85  & 85  &    &    &    &    &    & 2041918 \\
    4                   & 64  & 64  & 64  & 64 &    &    &    &    & 2029248 \\
    5                   & 51  & 51  & 51  & 51 & 51 &    &    &    & 2016350 \\
    6                   & 42  & 42  & 42  & 42 & 42 & 42 &    &    & 2001944 \\
    7                   & 36  & 36  & 36  & 36 & 36 & 36 & 36 &    & 1996184 \\
    8                   & 32  & 32  & 32  & 32 & 32 & 32 & 32 & 32 & 2000448 \\
    \bottomrule
    \end{tabular}

Experiment jsem provedl na síti se vstupním oknem 978 vzorků, multiplikátorem kapacity 8, 

    \begin{tabular}{lrr}
    \toprule
    Počet konvolučních vrstev &  Raw Pitch Accuracy &  Raw Chroma Accuracy \\
    \midrule
    1                         &               0.629 &                0.734 \\
    2                         &               0.628 &                0.732 \\
    3                         &               0.632 &                0.734 \\
    4                         &               0.636 &                0.739 \\
    5                         &               0.643 &                0.740 \\
    6                         &               0.638 &                0.737 \\
    7                         &               0.636 &                0.736 \\
    8                         &               0.640 &                0.737 \\
    \bottomrule
    \end{tabular}

Zlepšení výsledků se pohybuje v řádu desetin procentních bodů, tedy není příliš vysoké. Zlepšení je nejvíce patrné v případě pěti různě širokých konvolučních vrstev, kde dosahuje $1.3$ procentního bodu. Analýzou výsledků přesnosti podle výšky noty se mi nepodařilo prokázat hypotézu, že by konvoluce s více rozlišeními pomáhala u odhadu not vyšších frekvencí. Její přínos je drobný a projevuje se na většině frekvenčních pásem.


\section{Wavenet}

Generativní model WaveNet popsaný týmem \cite{Oord2016} je architektura navržená pro generování zvukového signálu, autoři však síť testovali i pro převod mluvené řeči na text (dataset TIMIT) a dosáhli výsledků srovnatelných se state-of-the-art. Síť se však pro *Music Information Retrieval* od svého zveřejnění příliš neuchytila. Její použití se v oblasti hudby se omezuje na generativní úlohy (\cite{Hawthorne2018a}, \cite{Yang2017}, \cite{Engel2017} a další), případně *source-separation* \citep{Stoller2018}. Jediný publikovaný pokus s použitím architektury WaveNet pro automatický přepis podnikli \cite{Martak2018} nad datasetem MusicNet. Jejich model však netestovali na standardních evaluačních datasetech ze soutěže MIREX, tudíž není zřejmé, jakých výsledků v porovnání s existujícími metodami autoři dosáhli.

Architektura spočívá v důmyslném vrstvení dilatovaných konvolucí. Díky exponenciálně rostoucím dilatacím se také exponenciálně zvětšuje receptivní pole jednotlivých konvolučních vrstev. Díky této vlastnosti pak například stačí pro pokrytí 1024 vzorků vstupu pouze 9 vrstev s šířkou kernelu 2 a dilatacemi 1,2,4,8 ... 512. Pokud bychom stejného receptivního pole chtěli dosáhnout pomocí obvyklých konvolucí počet potřebných vrstev by byl lineární vzhledem k šířce pole. Vrstvení konvolucí je porovnáno na schématu. 

% TODO: přidat schéma konvolucí

\subsection{Baseline na základě \cite{Martak2018}}

Pro srovnání spustíme architekturu popsanou ve zmíněném článku pro úlohu extrakce melodie. Jelikož byla architektura zamýšlena pro dataset MusicNet, který obsahuje celý přepis skladeb do MIDI not, výstupem jsou diskrétní noty. Jak jsme zjistili v předchozím experimentu na architektuře CREPE, hrubá diskretizace výrazně zhoršuje přesnost výsledků, upravíme proto architekturu tak, aby měla výstupní distribuce jemnější rozlišení.

\chapter{Výsledky}\label{cha:vysledky}

V této kapitole shrnujeme výsledky vybraných architektur z kapitoly \nameref{cha:experimenty} na testovacích datech a provádíme kvantitativní a kvalitativní srovnání se state-of-the-art systémy pro extrakci melodie, představené v kapitole \nameref{cha:souvisejici}.

\section{Výběr testovaných modelů}

Pro srovnání jsme vybrali nejlepší natrénované modely z kapitoly \nameref{cha:experimenty} od každé testované architektury. Každou vybranou architekturu krátce popíšeme a v tabulce uvedeme nalezené nastavení hyperparametrů.

Všechny testované modely mají společnou reprezentaci cílového výstupu při trénování. Rozlišení diskretizace výšky noty je nastaveno na 5 hodnot na půltón, rozptyl distribuce výšky noty je nastaven na 0.18, na základě experimentů v sekci \nameref{sec:crepe}.

Modely byly trénovány pomocí grafické karty NVIDIA GeForce GTX 1070.

\subsection{Architektura CREPE}

Zvolený model dosáhl na validačních datech přesnosti odhadu výšky 0.682 a přesnosti odhadu výšky nezávisle na oktávě 0.783. Počet trénovatelných parametrů tohoto modelu je $2\,016\,350$. Proběhlo $360\,000$ iterací trénování. Trénování probíhalo přes dvě hodiny. Parametry vybrané architektury uvádíme v tabulce \ref{tab:crepe_hyperparams}.

\begin{table}[p]
\centering
\begin{tabular}{llll}
\toprule
Prohledávané parametry               &          & Ostatní parametry &         \\
Parametr                             & Hodnota  & Parametr          & Hodnota \\
\midrule
Multiplikační koef. kapacity         & 8        & Velikost dávky    & 32      \\
Šířka vstupního okna                 & 2048     & Iterace trénování & 360000         \\
Násobné rozlišení první vrstvy       & 6 vrstev & Learning rate     & 0.001   \\
\bottomrule
\end{tabular}
\caption{Nastavení architektury a hyperparametrů pro testovanou architekturu CREPE.}\label{tab:crepe_hyperparams}
\end{table}


\subsection{Architektura WaveNet}

Zvolený model dosáhl na validačních datech přesnosti odhadu výšky 0.673 a přesnosti odhadu výšky nezávisle na oktávě 0.768. Počet trénovatelných parametrů tohoto modelu je $5\,206\,524$. Proběhlo $360\,000$ iterací trénování. Trénování probíhalo téměř pět hodin. Parametry vybrané architektury uvádíme v tabulce \ref{tab:wavenet_hyperparams}.

\begin{table}[p]
\centering
\begin{tabular}{llll}
\toprule
Prohledávané parametry               &          & Ostatní parametry &         \\
Parametr                             & Hodnota  & Parametr          & Hodnota \\
\midrule
Velikost první konvoluce & 0      & Iterace trénování     & 100000 \\
Počet filtrů  & 16     & Velikost dávky    & 8      \\
Počet bloků            & 2     & Learning rate & 0.001       \\
Maximální dilatace         & 1024   &                &        \\
Transformace skip propojení & concat &                &        \\
\bottomrule
\end{tabular}
\caption{Nastavení architektury a hyperparametrů pro testovanou architekturu WaveNet.}\label{tab:wavenet_hyperparams}
\end{table}

\subsection{Architektura HCNN}

Pro úspěšnost modelů na validačních datech jsme se rozhodli porovnat dvě různé architektury HCNN. Architekturu s minimálním kontextem $5.8\,\rm ms$, dále nazývanou HCNN noctx, a architekturu, která uvažuje širší kontext, dále jen HCNN.

\subsubsection{HCNN noctx}

Zvolený model dosáhl na validačních datech přesnosti odhadu výšky 0.751 a přesnosti odhadu výšky nezávisle na oktávě 0.813. Počet trénovatelných parametrů tohoto modelu je $27\,857$. Proběhlo $100\,000$ iterací trénování. Trénování probíhalo 22 minut. Parametry vybrané architektury uvádíme v tabulce \ref{tab:hcnnnoctx_hyperparams}.

\begin{table}[p]
\centering
\begin{tabular}{llll}
\toprule
Prohledávané parametry               &          & Ostatní parametry &         \\
Parametr                             & Hodnota  & Parametr          & Hodnota \\
\midrule
Parametr \texttt{hop\_size} & 256       & Iterace trénování                       & 100000 \\
Kontext                & noctx         & Velikost dávky                      & 16     \\
Vstupní reprezentace   & HCQT       & Learning rate                   & 0.001  \\
Filtry, Bloky                 & 16, 8        & Dropout                          & 0.3    \\
Harmonická trans. & $\pm 12$, $\pm 19$ &  &       \\
\bottomrule
\end{tabular}
\caption{Nastavení architektury a hyperparametrů pro testovanou architekturu HCNN noctx.}\label{tab:hcnnnoctx_hyperparams}
\end{table}
\begin{table}[p]
\centering
\begin{tabular}{p{4.5cm}p{3cm}ll}
\toprule
Prohledávané parametry               &          & Ostatní parametry &         \\
Parametr                             & Hodnota  & Parametr          & Hodnota \\
\midrule
Parametr \texttt{hop\_size} & 256     & Iterace trénování                       & 100000 \\
Kontext            & 3\_last\_layers \_wavenet & Velikost dávky                      & 8      \\
Vstupní reprezentace    & Vícekan. HCQT & Learning rate                   & 0.001  \\
Filtry, Bloky & 16, 4      & Dropout                          & 0.3    \\
Harmonická trans. & $\pm 12$&   &       \\
Pst. augmentace       & 0.75    &                                  &        \\
\bottomrule
\end{tabular}
\caption{Nastavení architektury a hyperparametrů pro testovanou architekturu HCNN.}\label{tab:hcnn_hyperparams}
\end{table}

\subsubsection{HCNN}

Zvolený model dosáhl na validačních datech přesnosti odhadu výšky 0.755 a přesnosti odhadu výšky nezávisle na oktávě 0.828. Počet trénovatelných parametrů tohoto modelu je $23\,153$. Proběhlo $100\,000$ iterací trénování. Trénování probíhalo 75 minut. Parametry vybrané architektury uvádíme v tabulce \ref{tab:hcnn_hyperparams}.

\subsection{Detekce melodie}

Pro detekci melodie z vypočítané funkce salience jsme použili techniku práhování (thresholding). Pro odhad přítomnosti melodie nalezneme maximální hodnotu funkce salience v daném okamžiku --- pokud tato hodnota přesahuje jistý práh, daný časový okamžik se uvažuje jako obsahující melodii. Konkrétní nastavení práhu jsme určili na základě validačních dat pro každou metodu zvlášť. Tato metoda práhování je použita například také v práci \cite{Bittner2017}.

\section{Kvantitativní srovnání}

% , která je představuje evaluační páteř oboru a je každoročně pořádána pro nezávislé srovnávání metod úloh Music Information Retrieval

V tabulkách \ref{tab:vysledky_OA}, \ref{tab:vysledky_RPA}, \ref{tab:vysledky_RCA}, \ref{tab:vysledky_VR} a \ref{tab:vysledky_VFA} prezentujeme výsledky nově představovaných metod v porovnání s velmi silnými baseline metodami pro extrakci melodie. Jak uvádíme v kapitole \nameref{cha:souvisejici}, práce \cite{Salamon2012a} dosahuje spolu s prací \cite{Dressler2009} v průměru nejlepších výsledků v soutěži MIREX. Práce \cite{Bittner2017} a \cite{DBasaranSEssid2018} představují metody, které dosahují nejlepších výsledků na prozatím nejrozmanitějším datasetu MedleyDB. 

Pro srovnání metod používáme datasety ADC04, MIREX05train a ORCHSET, které jsme v práci vyhradili pouze pro testování. Také používáme testovací množiny datasetů MedleyDB a MDB-melody-synth, převzaté z prací \cite{Bittner2017} a \cite{DBasaranSEssid2018}, pro dataset WJazzD používáme vlastní testovací množinu. Metodika výběru příkladů do testovací množiny WJazzD je popsána v kapitole \nameref{cha:datasety}, všechny množiny jsou výčtem popsány v elektronické příloze.

V následujících tabulkách uvádíme standardní metriky, používané pro srovnání metod pro extrakci melodie, jde o metriky celkové přesnosti (OA), přesnosti odhadu výšky (RPA), přesnosti odhadu výšky nezávisle na oktávě (RCA), úplnosti detekce melodie (VR) a nesprávné detekce (VFA). Připomínáme, že metrika celkové přesnosti (OA) zahrnuje vyhodnocení odhadu výšky i vyhodnocení detekce melodie, zbylé metriky měří úspěšnost pouze jedné z obou podúloh. K vyhodnocování používáme knihovnu \texttt{mir\_eval}, která poskytuje transparentní a standardizovaný způsob výpočtu metrik pro úlohy oboru Music Information Retrieval.

\begin{table}[p]
\centering
\scalebox{1.0}{%
\begin{tabular}{lrrrrrr}
\toprule
Metoda & ADC04 & \shortstack[r]{MDB-m-s\\ test} & \shortstack[r]{MIREX05\\train.} & \shortstack[r]{MDB\\test} & \shortstack[r]{ORCH-\\SET} & \shortstack[r]{WJazzD\\test} \\
\midrule
    Salamon &         0.714 &           0.527 &          0.715 &         0.519 &        0.235 &        0.667 \\
    Bittner &         0.716 &           0.633 &          0.702 &         0.611 &        0.407 &        0.692 \\
    Basaran &         0.669 &   \textbf{0.689}&          0.734 &         0.640 &\textbf{0.483}&        0.700 \\
\arrayrulecolor{black!30}\midrule
      CREPE &         0.590 &           0.562 &          0.652 &         0.502 &        0.248 &        0.671 \\
    WaveNet &         0.681 &           0.528 &          0.649 &         0.503 &        0.256 &        0.648 \\
 HCNN noctx & \textbf{0.737}&           0.626 &          0.723 &         0.635 &        0.439 &        0.715 \\
       HCNN &         0.726 &           0.661 &  \textbf{0.755}& \textbf{0.652}&        0.459 &\textbf{0.725} \\
\arrayrulecolor{black}\bottomrule
\end{tabular}
}%
\caption{Výsledky celkové přesnosti (Overall Accuracy). Vyznačené výsledky jsou pro daný dataset nejvyšší z porovnávaných v rámci daného datasetu.}\label{tab:vysledky_OA}
\end{table}

\begin{table}[p]
\centering
\scalebox{1.0}{%
\begin{tabular}{lrrrrrr}
\toprule
Metoda & ADC04 & \shortstack[r]{MDB-m-s\\ test} & \shortstack[r]{MIREX05\\train.} & \shortstack[r]{MDB\\test} & \shortstack[r]{ORCH-\\SET} & \shortstack[r]{WJazzD\\test} \\
\midrule
    Salamon &          0.767 &           0.514 &          0.761 &         0.526 &         0.281 &        0.693 \\
    Bittner &          0.814 &           0.606 &          0.807 &         0.670 &         0.519 &        0.774 \\
    Basaran &          0.793 &   \textbf{0.733}&          0.798 &         0.706 & \textbf{0.635}&        0.767 \\
\arrayrulecolor{black!30}\midrule
      CREPE &          0.794 &           0.550 &          0.779 &         0.616 &         0.408 &        0.782 \\
    WaveNet &          0.796 &           0.528 &          0.792 &         0.595 &         0.345 &        0.759 \\
 HCNN noctx &          0.827 &           0.647 &          0.833 &         0.701 &         0.511 &        0.805 \\
       HCNN &  \textbf{0.841}&           0.654 &  \textbf{0.851}& \textbf{0.715}&         0.535 &\textbf{0.806}\\
\arrayrulecolor{black}\bottomrule
\end{tabular}
}%
\caption{Výsledky přesnosti odhadu výšky (Raw Pitch Accuracy). Vyznačené výsledky jsou pro daný dataset nejvyšší z porovnávaných v rámci daného datasetu.}\label{tab:vysledky_RPA}
\end{table}

\begin{table}[p]
\centering
\scalebox{1.0}{%
\begin{tabular}{lrrrrrr}
\toprule
Metoda & ADC04 & \shortstack[r]{MDB-m-s\\ test} & \shortstack[r]{MIREX05\\train.} & \shortstack[r]{MDB\\test} & \shortstack[r]{ORCH-\\SET} & \shortstack[r]{WJazzD\\test} \\
\midrule
    Salamon &        0.807 &        0.639 &        0.805 &         0.659 &         0.568 &             0.757 \\
    Bittner &        0.855 &        0.666 &        0.824 &         0.735 &         0.694 &             0.785 \\
    Basaran &        0.820 &\textbf{0.766}&        0.807 &         0.757 & \textbf{0.776}&             0.776 \\
\arrayrulecolor{black!30}\midrule
      CREPE &        0.851 &        0.617 &        0.810 &         0.714 &         0.607 &             0.808 \\
    WaveNet &        0.843 &        0.597 &        0.828 &         0.703 &         0.564 &             0.793 \\
 HCNN noctx &        0.862 &        0.699 &        0.845 &         0.767 &         0.683 &     \textbf{0.821}\\
       HCNN &\textbf{0.880}&        0.716 &\textbf{0.863}& \textbf{0.781}&         0.732 &             0.820 \\
\arrayrulecolor{black}\bottomrule
\end{tabular}
}%
\caption{Výsledky přesnosti odhadu výšky nezávisle na oktávě (Raw Chroma Accuracy). Vyznačené výsledky jsou pro daný dataset nejvyšší z porovnávaných v rámci daného datasetu.}\label{tab:vysledky_RCA}
\end{table}


\begin{table}[h]
\centering
\scalebox{1.0}{%
\begin{tabular}{lrrrrrr}
\toprule
Metoda & ADC04 & \shortstack[r]{MDB-m-s\\ test} & \shortstack[r]{MIREX05\\train.} & \shortstack[r]{MDB\\test} & \shortstack[r]{ORCH-\\SET} & \shortstack[r]{WJazzD\\test} \\
\midrule
\arrayrulecolor{black!30}\midrule
    Salamon &          0.774 &       \bf{0.729}&      \bf{0.841}&         0.705 &       0.603 &       0.794 \\
    Bittner &          0.796 &           0.638 &          0.796 &         0.675 &       0.614 &       0.846 \\
    Basaran &          0.732 &           0.704 &          0.713 &         0.676 &       0.605 &       0.841 \\
      CREPE &          0.584 &           0.431 &          0.576 &         0.449 &       0.326 &       0.680 \\
    WaveNet &          0.765 &           0.595 &          0.747 &         0.618 &       0.494 &       0.784 \\
 HCNN noctx &      \bf{0.806}&           0.684 &          0.824 &         0.728 &   \bf{0.729}&   \bf{0.880}\\
       HCNN &          0.794 &           0.692 &          0.836 &     \bf{0.741}&       0.721 &       0.872 \\
\arrayrulecolor{black}\bottomrule
\end{tabular}
}%
\caption{Výsledky úplnosti detekce (Voicing Recall).}\label{tab:vysledky_VR}
\end{table}
\begin{table}[h]
\centering
\scalebox{1.0}{%
\begin{tabular}{lrrrrrr}
\toprule
Metoda & ADC04 & \shortstack[r]{MDB-m-s\\ test} & \shortstack[r]{MIREX05\\train.} & \shortstack[r]{MDB\\test} & \shortstack[r]{ORCH-\\SET} & \shortstack[r]{WJazzD\\test} \\
\midrule
\arrayrulecolor{black!30}\midrule
    Salamon &      \bf{0.103}&           0.394 &          0.263 &         0.300 &        0.385 &       0.271 \\
    Bittner &          0.278 &           0.273 &          0.308 &         0.306 &        0.490 &       0.333 \\
    Basaran &          0.188 &           0.271 &      \bf{0.160}&         0.290 &        0.407 &       0.274 \\
      CREPE &          0.178 &       \bf{0.252}&          0.171 &     \bf{0.243}&    \bf{0.235}&   \bf{0.213}\\
    WaveNet &          0.311 &           0.383 &          0.387 &         0.397 &        0.426 &       0.370 \\
 HCNN noctx &          0.246 &           0.312 &          0.336 &         0.333 &        0.535 &       0.339 \\
       HCNN &          0.222 &           0.258 &          0.278 &         0.310 &        0.511 &       0.300 \\
\arrayrulecolor{black}\bottomrule
\end{tabular}
}%
\caption{Výsledky nesprávné detekce (Voicing False Alarm). Nižší hodnota je lepší.}\label{tab:vysledky_VFA}
\end{table}


\subsection{Popis výsledků}

Metody HCNN a HCNN noctx překonávají srovnávané algoritmy v metrikách celkové přesnosti (OA), přesnosti odhadu výšky (RPA) a přesnosti odhadu výšky nezávisle na oktávě (RCA) na datasetech ADC04, MIREX05train a WJazzD. Metoda HCNN pak překonává všechny srovnávané přístupy i na datasetu MedleyDB. Na obrázku \ref{obr:final_medleydb} porovnáváme rozdělení dosažených výsledků na datasetu MedleyDB, na uvedeném krabicovém grafu lze navíc porovnat rozptyl výsledků. V metrice RCA metody HCNN dosahují menší variability na sadě testovacích příkladů. Na zbylých datasetech MDB-melody-synth a ORCHSET překonává architektura HCNN ve všech uvažovaných metrikách pouze práce \cite{Salamon2012a} a \cite{Bittner2017}.

Co se týče zbylých architektur CREPE a WaveNet, v metrikách přesnosti odhadu výšky (RPA) a přesnosti odhadu výšky nezávisle na oktávě (RCA) na téměř všech testovacích datasetech překonávají metodu \cite{Salamon2012a}, která není založena na strojovém učení. Výsledky v porovnání s HCNN, \cite{Bittner2017} a \cite{DBasaranSEssid2018} jsou však až na výjimky nižší.

Podle očekávání na základě výsledků ze soutěže MIREX se na datasetu ORCHSET výsledky algoritmů liší nejvíce, v některých případech až o desítky procent. Jak popisujeme v kapitole \nameref{cha:datasety}, dataset je složen z orchestrálních nahrávek a kvůli vysokému stupni polyfonie a rozmanitým kombinacím barev nástrojů se jedná pro metody extrakce melodie o velmi náročný materiál. Naopak nejblíže, zejména v metrikách RPA a RCA, jsou si výsledky na datasetech ADC04, MIREX05train a WJazzD. Výňatky v těchto datasetech často obsahují velmi zřetelnou melodii a v porovnání s datasetem MedleyDB je jejich hudební obsah žánrově homogenní.

Vzhledem k dosaženým výsledkům, jednoduchosti sítí a rychlému trénování považujeme návrh architektury HCNN jako nadějný podklad pro navazující práci. Abychom mohli uvedené výsledky interpretovat, provedeme nejprve také kvalitativní srovnání algoritmů. 

% \textcolor{red}{dopsat}
% - basaran má vyhlazování, bittnerová ne, tu překovávám vždy (porovnat velikosti oken)

\begin{figure}[h]\centering
\includegraphics[width=\textwidth,height=\textheight,keepaspectratio]{../img/final_medleydb}
\caption{Výsledky nejúspěšnějších metod na datasetu MedleyDB}
\label{obr:final_medleydb}
\end{figure}

\section{Kvalitativní srovnání}

Na základě kvantitativního vyhodnocení vybíráme metody Bittnerové a Basarana pro podrobnější srovnání na jednotlivých příkladech. Z testovaných architektur pak vybíráme obě varianty HCNN. V následujících srovnáních se soustředíme na odhad výšky, proto v obrázcích zobrazujeme pouze části odhadů, ve kterých podle referenční anotace melodie zní. Obrázky jsou bez tohoto zjednodušení příliš komplikované a v práci detekci melodie řešíme pouze okrajově. Metodika výběru kvalitativních příkladů spočívala v hledání skladeb, ve kterých se odhady jednotlivých algoritmů vzájemně nejvíce lišily s nadějí, že právě tyto příklady budou nejlépe ilustrovat limity porovnávaných metod. Vybíráme ale také příklady, které jsou napříč metodami pro odhad melodie obtížné a také ukázku snadno analyzovatelného vstupu. Legenda barev použitých ve všech následujících obrázcích je vysvětlena na obrázku \ref{obr:legenda}
 Pro srovnání vybíráme příklady, ve kterých se výstupy sítí nejvíce lišily.

\begin{figure}[h]\centering
\includegraphics[scale=0.75]{../img/legenda}
\caption{Legenda pro následující kvalitativní srovnání.}
\label{obr:legenda}
\end{figure}

\begin{figure}[h]\centering
\includegraphics[width=\textwidth,height=\textheight,keepaspectratio]{../img/vysledky/mirex05_train01}
\caption{Příklad s vysokou úspěšností přepisu \texttt{train01} z datasetu MIREX05\-train, se kterým je čtenář seznámen z úvodu práce.}
\label{obr:mirex05_train01}
\end{figure}

Na obrázku \ref{obr:mirex05_train01} můžeme vidět výsledky metod spuštěných na popové nahrávce, ve které melodii nese hlas zpěvačky. Napříč metodami je přepis této nahrávky, kterou uvádíme v úvodu práce, velmi spolehlivý, přesnost odhadu výšky se pohybuje mezi 0.86 a 0.89. Protože v následujících srovnáních ukazujeme zejména chyby přepisu a metody porovnáváme na obtížných příkladech, nechceme, aby si po přečtení sekce čtenář odnesl, že existující metody přepisu nefungují. Proto uvádíme tento příklad (obr. \ref{obr:mirex05_train01}) jako pozitivní ukázku toho, že na obvyklých vstupních datech všechny porovnávané metody fungují velmi dobře.


% \begin{table}[h]
% \centering

%     \begin{tabular}{ll}
%     \toprule
%     Metrika (Metoda) & train10 \\
%     \midrule
%           RPA (HCNN) &   0.917 \\
%           RCA (HCNN) &   0.934 \\
%       RPA (Bittner) &   0.863 \\
%       RCA (Bittner) &   0.868 \\
%       RPA (Basaran) &   0.512 \\
%       RCA (Basaran) &   0.572 \\
%     \bottomrule
%     \end{tabular}

% \caption{Přesnost metod na testovacím souboru \texttt{train10} z datasetu MIREX05.}\label{tab:mirex05_train10}
% \end{table}
\begin{figure}[h]\centering
\includegraphics[width=\textwidth,height=\textheight,keepaspectratio]{../img/vysledky/orchset_Musorgski-Ravel-PicturesExhibition-ex6}
\caption{Výstup metod na testovacím souboru \texttt{Musorg\allowbreak{}ski\-Ravel\-Pictures\allowbreak{}Exhibition\-ex6} z datasetu ORCHSET.}
\label{obr:orchset_Musorgski-Ravel-PicturesExhibition-ex6}
\end{figure}

\begin{figure}[h]\centering
\includegraphics[scale=0.4]{../img/vysledky/orchset_Musorgski-Ravel-PicturesExhibition-ex6_salience}
\caption{Výstupní salience metody HCNN na testovacím souboru \texttt{Musorg\allowbreak{}ski\-Ravel\-Pictures\allowbreak{}Exhibition\-ex6} z datasetu ORCHSET.}
\label{obr:orchset_Musorgski-Ravel-PicturesExhibition-ex6_salience}
\end{figure}

Největší slabinou sítě HCNN noctx se stala podle očekávání časová kontinuita odhadů. Protože tato síť pro odhad výšek uvažuje vždy pouze $5.8\,\rm ms$ vstupu a po vytvoření funkce salience na odhady tónů neaplikujeme žádné způsoby vyhlazování, odhady jednotlivých časových oken na sebe nenavazují. To se nejeví jako zásadní problém v případech, kdy ve skladbě melodii nenese více hlasů v souzvuku (viz výstup \ref{obr:mirex05_train01}, \ref{obr:mirex05_train10}). U některých orchestrálních skladeb však vzniká problém, například pokud melodii nese zároveň sekce smyčců a dechů v různých oktávách. Jak můžeme vidět na výstupu algoritmů \ref{obr:orchset_Musorgski-Ravel-PicturesExhibition-ex6}, HCNN noctx pak \uv{přeskakuje} mezi oktávami. Problém je také dobře vidět na výstupní funkci salience \ref{obr:orchset_Musorgski-Ravel-PicturesExhibition-ex6_salience}, na které vidíme tři totožné kontury posunuté o oktávu. Mírné zlepšení tohoto problému vidíme na výstupech metod HCNN a \cite{Bittner2017}, které sice také nijak výsledek salienční funkce nezpracovávají, na druhou stranu pro její výpočet uvažují delší okna délky $162\,\rm ms$ v případě HCNN a $150\,\rm ms$ v případě metody Bittner. Na obrázku \ref{obr:orchset_Musorgski-Ravel-PicturesExhibition-ex6} vidíme, že množství odhadů těchto metod, je často chybný pouze kvůli nesprávně určené oktávě --- díky většímu kontextu může metoda vybrat v čase navazující odhady a proto tyto výstupy obsahují méně velmi krátkých chybných úseků. Metoda týmu \cite{DBasaranSEssid2018} odhad výšky tónů vyhlazuje pomocí rekurentní architektury GRU, jejich výstup proto obsahuje nejméně skoků, jelikož metoda uvažuje celý kontext skladby, nikoli jen okno omezené délky. Použití rekurentní sítě díky tomu dovoluje zachytit ještě dlouhodobější závislosti a výstupní kontura pak často obsahuje nejmenší množství velmi krátkých, chybných skoků mimo hlavní melodii, které se na obrázku \ref{obr:orchset_Musorgski-Ravel-PicturesExhibition-ex6} hojně vyskytují u metod bez vyhlazování. Na obrázku \ref{obr:orchset_Musorgski-Ravel-PicturesExhibition-ex6} proto vidíme, že metoda \cite{DBasaranSEssid2018} se drží při odhadu jedné oktávy a přesto, že je tato oktáva zvolena špatně, výsledný přepis je koherentní.

% Také je celkový průběh výsledné kontury u této metody častěji . Další výraznou slabinu jsme z analýzy jednotlivých výstupů sítí neregistrovali. U výstupů, ve kterých se sítě nejvíce lišily, byl tento rozdíl nejčastěji způsoben právě těmito diskontinuitami, případně pak zachycením jiného než hlavního nástroje v pozadí. Největší rozdíly ve výsledcích jsou proto na datasetu ORCHSET, ve kterém je výskyt polyfonie nejčastější. 


% \begin{table}[h]
% \centering

%     \begin{tabular}{ll}
%     \toprule
%     Metrika (Metoda) & Musorgski-Ravel-PicturesExhibition-ex6 \\
%     \midrule
%           RPA (HCNN) &                                  0.125 \\
%           RCA (HCNN) &                                  0.725 \\
%       RPA (Bittner) &                                  0.397 \\
%       RCA (Bittner) &                                  0.826 \\
%       RPA (Basaran) &                                  0.040 \\
%       RCA (Basaran) &                                  0.914 \\
%     \bottomrule
%     \end{tabular}

% \caption{Přesnost metod na testovacím souboru \texttt{Musorgski-Ravel-PicturesExhibition-ex6} z datasetu ORCHSET.}\label{tab:orchset_Musorgski-Ravel-PicturesExhibition-ex6}
% \end{table}

\begin{figure}[h]\centering
\includegraphics[width=\textwidth,height=\textheight,keepaspectratio]{../img/vysledky/wjazzd_CannonballAdderley_SoWh}
\caption{Detail přepisu metod na testovacím souboru \texttt{Cannonball\allowbreak{}Adderley\allowbreak{}\_\allowbreak{}So\allowbreak{}What} z datasetu WJazzD.}
\label{obr:wjazzd_CannonballAdderley_SoWhat_detail}
\end{figure}


Basaranova metoda pro tuto koherenci výsledných kontur však obětovala frekvenční přesnost znějících výšek tónů, frekvenční rozlišení této metody je totiž na úrovni jednoho půltónu. Jak jsme již prezentovali v kapitole \nameref{cha:experimenty}, výstup metod kvantizovaný na půltóny obsahuje množství chyb navíc, jelikož často selhává v zachycení frekvenčních modulací. Na obrázku \ref{obr:wjazzd_CannonballAdderley_SoWhat_detail} vidíme další limitaci takového výstupu --- pokud je obsah skladby laděný podle jiného referenčního tónu než jaký byl použit pro trénování sítě, znějící tóny vycházejí výškou \uv{mezi} výstupní složky. Z tabulky \ref{tab:wjazzd_CannonballAdderley_SoWhat} je pak zřejmé, že kvůli této kvantizaci síť nedosahuje srovnatelných výsledků, přestože na jiných, žánrově shodných datech, které jsou laděny na správný referenční tón, podává kompetitivní výsledky. Limitace se proto týká zejména jazzových nahrávek pocházejících z období před rokem 1955, před zavedením referenčního tónu A4=440Hz ve standardu ISO16.
\begin{table}[h]
\centering
\begin{tabular}{ll}
\toprule
 Metrika (Metoda) & CannonballAdderley\_SoWhat \\
\midrule
       RPA (HCNN) &                     0.850 \\
 RPA (HCNN noctx) &                     0.848 \\
    RPA (Bittner) &                     0.828 \\
    RPA (Basaran) &                     0.653 \\
\bottomrule
\end{tabular}


% \begin{figure}[h]\centering
% \includegraphics[width=\textwidth,height=\textheight,keepaspectratio]{../img/vysledky/wjazzd_CannonballAdderley_SoWhat}
% \caption{Výstup metod na testovacím souboru \texttt{CannonballAdderley\_SoWhat} z datasetu WJazzD.}
% \label{obr:wjazzd_CannonballAdderley_SoWhat}
% \end{figure}

\caption{Přesnost metod na testovacím souboru \texttt{Cannonball\allowbreak{}Adderley\allowbreak{}\_So\allowbreak{}What} z datasetu WJazzD.}\label{tab:wjazzd_CannonballAdderley_SoWhat}
\end{table}


\begin{figure}[p]\centering
\includegraphics[width=\textwidth,height=\textheight,keepaspectratio]{../img/vysledky/mirex05_train10}
\caption{Výstup metod na testovacím souboru \texttt{train10} z datasetu MIREX05.}
\label{obr:mirex05_train10}
\end{figure}

\begin{figure}[p]\centering
\includegraphics[width=\textwidth,height=\textheight,keepaspectratio]{../img/vysledky/basaran_salience_comparison}
\caption{Srovnání vstupní frekvenčně-časové reprezentace $\bm{\mathrm{H}}^{F_0}$ Basaranovy metody testovacích souborů \texttt{train01} a \texttt{train10} z datasetu MIREX05.}
\label{obr:basaran_salience_comparison}
\end{figure}
Dalším problémem Basaranovy metody je zhoršená schopnost extrakce na syntetických datech. Příklady \texttt{midi2REF}, \texttt{midi3REF}, \texttt{train10REF} z datasetů ADC04 a MIREX05 jsou syntetizovány na základě MIDI pomocí základních zvukových fontů, nahrávky proto zní velmi uměle. Jak vidíme na obrázku \ref{obr:mirex05_train10}, zatímco metody Bittnerové a HCNN si s touto syntetickou barvou hlasu dokáží poradit, výstup Basaranovy metody obsahuje šum a skoky k doprovázejícím nástrojům. Příčinou může být jiná vstupní reprezentace signálu, která je založena na práci \cite{Durrieu2010} a spočívá na modelování hlavního hlasu pomocí zdroje a filtrů. Na obrázku \ref{obr:basaran_salience_comparison} srovnáváme tuto reprezentaci pro vstupní signál s lidským zpěvem (nahoře, \texttt{train01}) a pro signál se syntetickou flétnou (dole, \texttt{train10}). Je zřejmé, že zatímco lidský zpěv tato reprezntace dokáže zachytit, syntetický hlas na reprezentaci téměř zachycen není. 

\begin{figure}[h]\centering
\includegraphics[width=\textwidth,height=\textheight,keepaspectratio]{../img/vysledky/mdb_MatthewEntwistle_FairerHopes}
\caption{Výstup metod na testovacím souboru \texttt{Matthew\allowbreak{}Entwistle\allowbreak{}\_Fairer\allowbreak{}Hopes} z datasetu MedleyDB.}
\label{obr:mdb_MatthewEntwistle_FairerHopes}
\end{figure}
Pokud srovnáme metody HCNN a metodu Bittner, rozdílem v predikcích jsou zejména jiné priority, které přiřazují barvám hlasů. Lze tudíž nalézt mnoho příkladů, kde HCNN zároveň přepisuje melodii a nesprávně místy přeskakuje k nástrojům v doprovodu, zatímco metoda Bittnerové na stejném příkladu tuto chybu nedělá, podobně však existují i opačné příklady. Příkladem, ve kterém se tyto metody nejvíce rozchází, je soubor \texttt{MatthewEntwistle\_FairerHopes} z kolekce MedleyDB, ve kterém melodii hraje harfa. Zvuk harfy se však nevyskytuje v množině trénovacích dat, rozdílem proto je, že zatímco metoda HCNN zvládá alespoň částečně generalizovat i na tuto dosud neslyšenou barvu hlasu, metoda Bittnerové tyto tóny úplně ignoruje a přepisuje doprovod pod harfou (viz obrázek \ref{obr:mdb_MatthewEntwistle_FairerHopes}).

\section{Interpretace výsledků}

Zásadní výhodou metody Basaran oproti HCNN a Bittner je zpracování výsledků funkce salience pomocí rekuretní sítě. Tento rozdíl spolu s jinou výchozí časově-frekvenční reprezentací signálu jeho metodu zvýhodňuje zejména na datasetu ORCHSET, kde jeho metoda v metrice RPA dosahuje o deset procentních bodů lepších výsledků. Pro metodu Basaran je na tomto datasetu také výhodné, že jeho referenční anotace mají půltónové frekvenční rozlišení, tedy stejné, jako výstup této metody. Tudíž jeho metoda při použití tohoto hrubého rozlišení nijak netratí. Na zbylých datasetech se však nižší frekvenční rozlišení projevuje více a metodu pravděpodobně spíše znevýhodňuje. Přesto jsou jeho predikce, zejména pak na složitějších vstupních datech, často koherentnější, obsahují méně šumu. 

Výsledky HCNN a Bittner jsou si oproti výsledkům metody Basaran mnohem podobnější, ačkoliv je mezi nimi větší procentuální rozdíl. Při kvalitativním vyhodnocování jsme nenalezli příliš mnoho příkladů, na kterém by se výstupy metod výrazně lišily. Mezi sítěmi je však řada podobností, zejména stejná vstupní reprezentace, přibližně stejně velký zpracovávaný kontext, přeskočení vyhlazování výstupu ale i celková struktura sítě. Kvantitativní rozdíly na všech datasetech tedy přičítáme spíše lépe naučeným barvám nástrojů a jejich priorit v celkovém mixu. Podobnost výsledků ilustrujeme také výpočtem korelace, zatímco výsledky metriky RPA pro metody Bittner a HCNN mezi sebou mají korelaci 0.932, mezi Basaran a HCNN vychází nižší korelace 0.736.

\section{Online demo}

Pro kvalitativní srovnání výsledků všech metod představovaných v práci je zpřístupněno jednoduché online demo na url \url{http://jirkabalhar.cz:6090/}. 

% \begin{table}[h!]
% \centering

%   \begin{tabular}{ll}
%   \toprule
%   Metrika (Metoda) & MatthewEntwistle\_FairerHopes \\
%   \midrule
%         RPA (HCNN) &                        0.451 \\
%         RCA (HCNN) &                        0.626 \\
%     RPA (Bittner) &                        0.118 \\
%     RCA (Bittner) &                        0.423 \\
%     RPA (Basaran) &                        0.544 \\
%     RCA (Basaran) &                        0.661 \\
%   \bottomrule
%   \end{tabular}

% \caption{Přesnost metod na testovacím souboru \texttt{MatthewEntwistle\_FairerHopes} z datasetu MedleyDB.}\label{tab:mdb_MatthewEntwistle_FairerHopes}
% \end{table}




% \begin{figure}[h]\centering
% \includegraphics[width=\textwidth,height=\textheight,keepaspectratio]{../img/vysledky/mdb_MusicDelta_Pachelbel}
% \caption{Výstup metod na testovacím souboru \texttt{MusicDelta\_Pachelbel} z datasetu MedleyDB.}
% \label{obr:mdb_MusicDelta_Pachelbel}
% \end{figure}

% \begin{table}[h!]
% \centering

% \begin{tabular}{ll}
% \toprule
% Metrika (Metoda) & MusicDelta\_Pachelbel \\
% \midrule
%       RPA (HCNN) &                0.472 \\
%       RCA (HCNN) &                0.510 \\
%    RPA (Bittner) &                0.461 \\
%    RCA (Bittner) &                0.493 \\
%    RPA (Basaran) &                0.435 \\
%    RCA (Basaran) &                0.491 \\
% \bottomrule
% \end{tabular}

% \caption{Přesnost metod na testovacím souboru \texttt{MusicDelta\_Pachelbel} z datasetu MedleyDB.}\label{tab:mdb_MusicDelta_Pachelbel}
% \end{table}


\chapter*{Závěr}
\addcontentsline{toc}{chapter}{Závěr}

% V práci prezentujeme tři nové metody výpočtu funkce salience s důrazem na odhad výšky tónů v nahrávkách. Ze tří navrhovaných dosáhla architektura HCNN výsledků, které na většině veřejně dostupných datasetech překonávají state-of-the-art metody extrakce melodie. 

%%% Seznam použité literatury
\include{literatura}

%%% Obrázky v bakalářské práci
%%% (pokud jich je malé množství, obvykle není třeba seznam uvádět)
\listoffigures

%%% Tabulky v bakalářské práci (opět nemusí být nutné uvádět)
%%% U matematických prací může být lepší přemístit seznam tabulek na začátek práce.
\listoftables

%%% Použité zkratky v bakalářské práci (opět nemusí být nutné uvádět)
%%% U matematických prací může být lepší přemístit seznam zkratek na začátek práce.
\chapwithtoc{Seznam použitých zkratek}

\begin{tabular}{ll}
MIDI & Musical Instrument Digital Interface \\
MIR & Music Information Retrieval \\
MIREX & Music Information Retrieval Evaluation eXchange \\
ISMIR & The International Society of Music Information Retrieval \\
OA & Overall Accuracy \\
RPA & Raw Pitch Accuracy \\
RCA & Raw Chroma Accuracy \\
VR & Voicing Recall \\
VFA & Voicing False Alarm \\
HCQT & Harmonic Constant-Q Transform \\
CQT & Constant-Q Transform \\
STFT & Short Time Fourier Transform \\
FFT & Fast Fourier Transform \\
HCNN & Harmonic Convolutional Neural Network \\
\end{tabular}

%%% Přílohy k bakalářské práci, existují-li. Každá příloha musí být alespoň jednou
%%% odkazována z vlastního textu práce. Přílohy se číslují.
%%%
%%% Do tištěné verze se spíše hodí přílohy, které lze číst a prohlížet (dodatečné
%%% tabulky a grafy, různé textové doplňky, ukázky výstupů z počítačových programů,
%%% apod.). Do elektronické verze se hodí přílohy, které budou spíše používány
%%% v elektronické podobě než čteny (zdrojové kódy programů, datové soubory,
%%% interaktivní grafy apod.). Elektronické přílohy se nahrávají do SISu a lze
%%% je také do práce vložit na CD/DVD. Povolené formáty souborů specifikuje
%%% opatření rektora č. 23/2016.
\chapwithtoc{Přílohy}

% \appendix
% \chapter{Přílohy}

\section{Architektura HCNN, Vliv úpravy architektury ovlivňující receptivní pole modelu}\label{appendix:hcnn_ctx}

\begin{table}[h]
\centering
\begin{tabular}{llrr}

\toprule
Konfigurace bloků & Úprava architektury &   RPA &   RCA \\
\midrule
8 filtrů, 4 konv. bloky & noctx & 0.722 & 0.786 \\
{} & deep\_ctx\_3 & 0.728 & 0.796 \\
{} & first\_layers\_ctx & 0.728 & 0.795 \\
{} & last\_layers\_ctx & 0.737 & 0.800 \\
{} & 1\_last\_layer\_dilated & 0.733 & 0.797 \\
{} & 2\_last\_layers\_dilated & 0.733 & 0.796 \\
{} & 3\_last\_layers\_dilated & 0.731 & 0.796 \\
{} & 2\_last\_layers\_wavenet & 0.733 & 0.798 \\
{} & 3\_last\_layers\_wavenet & 0.745 & 0.804 \\
\arrayrulecolor{black!30}\midrule
16 filtrů, 4 konv. bloky & noctx & 0.737 & 0.802 \\
{} & deep\_ctx\_3 & 0.744 & 0.806 \\
{} & first\_layers\_ctx & 0.744 & 0.804 \\
{} & last\_layers\_ctx & 0.757 & 0.817 \\
{} & 1\_last\_layer\_dilated & 0.743 & 0.802 \\
{} & 2\_last\_layers\_dilated & 0.749 & 0.814 \\
{} & 3\_last\_layers\_dilated & 0.752 & 0.813 \\
{} & 2\_last\_layers\_wavenet & 0.754 & 0.813 \\
{} & 3\_last\_layers\_wavenet & 0.761 & 0.819 \\
\arrayrulecolor{black!30}\midrule
8 filtrů, 8 konv. bloků & noctx & 0.732 & 0.801 \\
{} & deep\_ctx\_3 & 0.735 & 0.796 \\
{} & first\_layers\_ctx & 0.747 & 0.814 \\
{} & last\_layers\_ctx & 0.749 & 0.811 \\
{} & 1\_last\_layer\_dilated & 0.745 & 0.806 \\
{} & 2\_last\_layers\_dilated & 0.751 & 0.814 \\
{} & 3\_last\_layers\_dilated & 0.749 & 0.806 \\
{} & 2\_last\_layers\_wavenet & 0.754 & 0.814 \\
{} & 3\_last\_layers\_wavenet & 0.751 & 0.803 \\
\arrayrulecolor{black!30}\midrule
16 filtrů, 8 konv. bloků & noctx & 0.744 & 0.809 \\
{} & deep\_ctx\_3 & 0.741 & 0.815 \\
{} & first\_layers\_ctx & 0.748 & 0.817 \\
{} & last\_layers\_ctx & 0.749 & 0.819 \\
{} & 1\_last\_layer\_dilated & 0.753 & 0.817 \\
{} & 2\_last\_layers\_dilated & 0.758 & 0.819 \\
{} & 3\_last\_layers\_dilated & 0.757 & 0.817 \\
{} & 2\_last\_layers\_wavenet & 0.759 & 0.819 \\
{} & 3\_last\_layers\_wavenet & 0.749 & 0.809 \\
\arrayrulecolor{black}\bottomrule
    \end{tabular}
\caption{Architektura HCNN, Vliv úpravy architektury ovlivňující receptivní pole modelu.}\label{tab:spectrogram_ctx_archs}
\end{table}

\section{Výsledky detekce melodie testovaných metod}\label{appendix:voicing}

\begin{table}[h]
\centering
\scalebox{1.0}{%
\begin{tabular}{lrrrrrr}
\toprule
Metoda & ADC04 & \shortstack[r]{MDB-m-s\\ test} & \shortstack[r]{MIREX05\\train.} & \shortstack[r]{MDB\\test} & \shortstack[r]{ORCH-\\SET} & \shortstack[r]{WJazzD\\test} \\
\midrule
\arrayrulecolor{black!30}\midrule
    Salamon &          0.774 &       \bf{0.729}&      \bf{0.841}&         0.705 &       0.603 &       0.794 \\
    Bittner &          0.796 &           0.638 &          0.796 &         0.675 &       0.614 &       0.846 \\
    Basaran &          0.732 &           0.704 &          0.713 &         0.676 &       0.605 &       0.841 \\
      CREPE &          0.584 &           0.431 &          0.576 &         0.449 &       0.326 &       0.680 \\
    WaveNet &          0.765 &           0.595 &          0.747 &         0.618 &       0.494 &       0.784 \\
 HCNN noctx &      \bf{0.806}&           0.684 &          0.824 &         0.728 &   \bf{0.729}&   \bf{0.880}\\
       HCNN &          0.794 &           0.692 &          0.836 &     \bf{0.741}&       0.721 &       0.872 \\
\arrayrulecolor{black}\bottomrule
\end{tabular}
}%
\caption{Výsledky úplnosti detekce (Voicing Recall).}\label{tab:vysledky_VR}
\end{table}
\begin{table}[h]
\centering
\scalebox{1.0}{%
\begin{tabular}{lrrrrrr}
\toprule
Metoda & ADC04 & \shortstack[r]{MDB-m-s\\ test} & \shortstack[r]{MIREX05\\train.} & \shortstack[r]{MDB\\test} & \shortstack[r]{ORCH-\\SET} & \shortstack[r]{WJazzD\\test} \\
\midrule
\arrayrulecolor{black!30}\midrule
    Salamon &      \bf{0.103}&           0.394 &          0.263 &         0.300 &        0.385 &       0.271 \\
    Bittner &          0.278 &           0.273 &          0.308 &         0.306 &        0.490 &       0.333 \\
    Basaran &          0.188 &           0.271 &      \bf{0.160}&         0.290 &        0.407 &       0.274 \\
      CREPE &          0.178 &       \bf{0.252}&          0.171 &     \bf{0.243}&    \bf{0.235}&   \bf{0.213}\\
    WaveNet &          0.311 &           0.383 &          0.387 &         0.397 &        0.426 &       0.370 \\
 HCNN noctx &          0.246 &           0.312 &          0.336 &         0.333 &        0.535 &       0.339 \\
       HCNN &          0.222 &           0.258 &          0.278 &         0.310 &        0.511 &       0.300 \\
\arrayrulecolor{black}\bottomrule
\end{tabular}
}%
\caption{Výsledky nesprávné detekce (Voicing False Alarm). Nižší hodnota je lepší.}\label{tab:vysledky_VFA}
\end{table}



\openright
\end{document}
