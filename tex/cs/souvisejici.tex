\chapter{Související práce}\label{cha:souvisejici}

Pokusy o vytvoření automatické metody pro kompletní transkripci hudby se podle \cite{Poliner2007} objevují již od sedmdesátých let, z důvodu značné obtížnosti této úlohy, která strmě roste s každým dalším přidaným hlasem ve zkoumaném signálu, však dodnes jedná o otevřený problém. Z tohoto důvodu se od devadesátých let objevují práce, které se pokouší alespoň o částečný automatický popis některých muzikálních aspektů skladeb.

Jednou z prvních je práce týmu \cite{Goto1999}, který se záměrně omezuje na identifikaci jedné, nejhlasitější, spojité křivky fundamentní frekvence hlasu (F0) v omezeném frekvenčním rozsahu. Vzniklé transkripce pak sice nejsou kompletní, na druhou stranu je jejich získání výpočetně nenáročné a přitom poskytují sémanticky bohatý popis nahrávek, který je poměrně často shodný s melodií. Ustanovením úlohy pojmenované jako "Predominant-F0 Estimation" (PreFEst), byly položeny základy pro vznik navazujících prací a soutěží zabývající se automatickým přepisem melodie.

Největší rozkvět v oboru začal od roku 2004. Uspořádáním první soutěže pro porovnání systémů pro automatický popis hudby v rámci konference ISMIR (ISMIR 2004 Audio Description Contest), se ustanovily priority, formalizovaly podmínky evaluace a byly sestaveny první kolekce dat pro testování algoritmů (\cite{Downie2010}). Soutěž se v následujícím roku přerodila do samostatné každoroční události, v níž soutěží stále více týmů v rostoucím počtu úloh. 

\textcolor{red}{TODO: dopsat signpost}
% tady by to chtělo nějaký odstaveček jako „signpost“. Metoda psaní ve třech fázích:
% - napřed vám řeknu, co vám budu říkat
% - pak vám to teda řeknu
% - a nakonec vám řeknu, co jsem vám říkal


\section{Průzkum existujících metod}

Jen do soutěže MIREX se od roku 2005 přihlásilo 45 týmů s 62 různými metodami pro extrakci melodie, s různou mírou přesnosti přepisu. Mezi přístupy k tomuto problému tedy existuje veliká rozmanitost, jejíž kompletní popis přesahuje rámec této práce. Zaměříme se proto na společné rysy a celkové trendy v oboru. 

Shrnující práce od \cite{Poliner2007} a \cite{Salamon2014} se při charakterizaci systémů pro transkripci odkazují na příbuznou úlohu odhadu fundamentální frekvence monofonní nahrávky. Algoritmy pro monofonní tracking na základě vstupního signálu $x_{mono}(t)$ počítají \emph{funkci salience} $S_{x_{mono}}(f_\tau, \tau)$ pro každý krátký časový okamžik (okno) $\tau$ a frekvenci $f_\tau$. Výsledkem této funkce je relativní ohodnocení (příp. pravděpodobnost) jednotlivých frekvencí obsažených ve vstupním signálu, které značí, zda-li je daná frekvence fundamentální frekvencí znějícího hlasu. 

Pro zvýšení robustnosti vůči šumu, přeslechu, dozvuku a jiným vlivům, které zhoršují kvalitu odhadu salience, se využívá také spojitosti fundamentální frekvence. Pro zajištění kontinuity extrahovaných frekvenčních kontur se zohledňuje také faktor temporálních závislostí $C(\mathbf{f})$, jejímž vstupem je kandidátní kontura $\mathbf{f}$ a výstupem je ohodnocení této celé kontury na její spojitost. Například tato funkce může penalizovat odhady, ve kterých výstupní F0 často přeskakuje o oktávu, což je u skutečného signálu nepravděpodobné a naopak jde o častou chybu při výpočtu salienční funkce signálů se silnými sudými harmonickými frekvencemi.

Výstupem monofonního trackingu je posloupnost frekvencí s maximální saliencí a spojitostí, tedy posloupnost frekvencí, které jsou nejlépe ohodnocenými kandidáty na fundamentální frekvenci a zároveň tato celá sekvence má také vysoké ohodnocení spojitosti.

    $$\hat{\mathbf{f}} = \argmax_{\mathbf{f}}{[\sum_{\tau}{S_{x_{mono}}(f_\tau, \tau)} + C(\mathbf{f})]}$$

Přejdeme-li k úloze extrakce melodie, obecně se vstupní polyfonní signál $x(t) = x_m(t) + x_d(t)$ skládá ze směsi melodického hlasu $x_m(t)$ a hudebního doprovodu $x_d(t)$, cílem metod pro extrakci je z pohledu přepisu fundamentální frekvence zvýšení robustnosti algoritmu vůči mnohem výraznějšímu druhu šumu - hudebnímu doprovodu $x_d(t)$. Výstupem našeho systému tedy bude posloupnost odhadů frekvence v každém časovém okně vstupního signálu, reprezentovaná vektorem $\hat{\mathbf{f}}$:

    $$\hat{\mathbf{f}} = \argmax_{\mathbf{f}}{[\sum_{\tau}{S'_x(f_\tau, \tau)} + C'(\mathbf{f})]}$$

kde $f_\tau$ je frekvence na pozici $\tau$ ve vektoru $\mathbf{f}$. $S'_x(f_\tau, \tau)$ je upravená funkce salience, která při výpočtu zohledňuje vliv doprovodu a složka $C'(\mathbf{f})$ představuje ohodnocení celého průběhu melodie.

Spolu s odhadem frekvencí by také měl systém na výstupu určit úseky, ve kterých v nahrávce melodie zní a kdy nikoli. K výstupu tedy patří také vektor $\hat{\mathbf{v}}$, se stejným počtem složek jako $\hat{\mathbf{f}}$, který indikuje přítomnost melodie v každém časovém okně $\tau$.

Většina existujících metod sdílí podobnou základní strukturu při řešení extrakce, která se zakládá na popsané formalizaci. Prvním krokem je transformace zvuku do frekvenční domény a následný odhad znějících výšek tónů v polyfonním signálu (výpočet funkce salience), druhým krokem je pak zpracování těchto odhadů a výběr melodie (tedy zpřesnění výsledné $\hat{\mathbf{f}}$ pomocí $C'(\mathbf{f})$). Přístupy k řešení těchto dvou kroků již s konkrétními příklady nastíníme v dalších sekcích.

% \subsection{Odhad výšek tónů}

\subsection{Spektrální analýza}

Zvuk hraného tónu na melodickém nástroji je z fyzikálního pohledu periodická změna tlaku vzduchu. Perioda tohoto signálu se nazývá fundamentální frekvence (označujeme F0) a zpravidla je tento signál složen ze součtu řady sinusoid, jejichž frekvence jsou celočíselným násobkem fundamentální frekvence. V čase měnící se amplitudy těchto \emph{harmonických frekvencí} udávají hlasitost a barvu hlasu, výška první harmonické frekvence (tj. výška fundamentální frekvence) pak ve většině případů odpovídá posluchačem vnímané výšce tónu. 

\textcolor{red}{TODO obrázek: signál -> spektrum -> salience}

Prvním krokem metod pracujících s hudebním signálem je proto provedení spektrální analýzy, jde o převod zvuku do frekvenční reprezentace, která odhaluje tyto harmonické struktury tónů a umožňuje s nimi dále pracovat. 

\subsubsection{Krátkodobá Fourierova transformace}

Přístupů ke spektrální analýze je více, přímočará a podle \cite{Dressler2016} nejčastěji používaná metoda je \emph{krátkodobá Fourierova transformace} (STFT). Jejím principem je rozdělení vstupního signálu na množinu překrývajících se oken konstantní délky a výpočet Fourierovy transformace těchto krátkých zvukových úseků. Komplexní výsledek transformace umocníme a získáme tzv. výkonové spektrum signálu, které obsahuje informaci o poměrech energie frekvencí, ze kterých se signál v okně skládá. 

    $$X(f, \tau) = \int_{-W/2}^{W/2}{w(t)y(\tau + t)e^{-j2\pi f t} \mathrm{d}t}$$

\textcolor{red}{TODO vysvětlit vzoreček, minimálně písmena}

\subsubsection{Multirezoluční transformace}

Na libovolnou metodu převodu diskretizovaného signálu na frekvenční doménu se vztahuje Gaborův limit, který popisuje závislost přesnosti lokalizace signálu ve frekvenční a časové doméně \citep{Gabor1945}. Volbou délky vstupního okna transformace zpřesňujeme buď frekvenční nebo časové rozlišení výsledné spektrální reprezentace. Zvolíme-li krátké vstupní okno, zvyšujeme časové rozlišení (krátké okno lépe zachycuje rychlé změny průběhu signálu), avšak ztrácíme přesnost na frekvenční ose, opačný vztah platí pro volbu delšího okna.

Tato limitace je markatní zejména pokud STFT používáme pro hudební data. Z povahy hudebních intervalů a harmonických struktur tónů platí, že téměř všechny periodické signály se v hudební skladbě vyskytují ve vzájmených poměrech (v případě intervalů v poměrech $2^{\frac{n}{12}}$ a v případě harmonických frekvencí v celočíselných), u vyšších tónů jsou proto vzdálenosti mezi frekvencemi signálů větší než u nižších tónů. Frekvenční rozlišení STFT je však konstantní na celém výstupním frekvenčním rozsahu a volba velikosti okna transformace zajistí dobrý poměr frekvenčního a časového rozlišení jen pro část rozsahu. Ve výsledku je pak buď pro vyšší frekvence okno příliš velké (zbytečně detailní frekvenční rozlišení na úkor časového rozlišení) a nebo naopak pro nižší frekvence je okno nedostačující (rozlišení frekvence nemusí být pro basy ani na úrovni půltónů).

Z tohoto důvodu existují vedle STFT i další metody, jejichž cílem je nabídnout lepší kompromis frekvenčně-časového rozlišení v kontextu melodických dat. \cite{Goto1999} používají MRFFT (Multi-Resolution Fast Fourier Transform), principem je opakovaný downsampling signálu (převzorkování na nižší vzorkovací frekvenci) a aplikace Fourierovy transformace na každý vzniklý signál; s každou iterací spektrum obsahuje čím dál podrobnější informace o nižších frekvencích, protože vyšší frekvence se při downsamplingu ztratí. \cite{Brown1990} popsala metodu Constant-Q Transform (CQT), která spočívá v použití proměnné délky okna Fourierovy transformace pro výstupní frekvenční pásma, která rovnoměrně pokrývají logaritmickou osu frekvence. \cite{Cancela2010} kombinuje CQT a Chirp Z-transform, jedná se o zobecnění diskrétní Fourierovy transformace, ve které je signál rozložen na tzv. lineární čerpy --- sinusoidy s proměnnou frekvencí. Díky tomu transformace dokáže s lepším rozlišením zachytit signály, které rychle mění výšku tónu, v hudbě například vibrato. \cite{Paiva2004} napodobují mechanismy lidského sluchu pomocí banky pásmových filtrů (Cochleagram) s logaritmicky rozmístěnými mezními frekvencemi a sumy autokorelací na jednotlivých frekvenčních pásmech signálu (Summary correlogram).

I přes uvedené důvody se \cite{Salamon2014} a \cite{Dressler2016} domnívají, že metoda zpracování signálu příliš neovlivňuje výslednou přesnost algoritmů pro přepis melodie. Tvrzení dokládají jednak celkovým srovnáním výsledků metod ze všech ročníků soutěže MIREX a jednak neochvějnou převahou využití krátkodobé Fourierovy transformace, jakožto efektivní a dostačující metody pro spektrální analýzu.

\subsubsection{Postprocessing spektrogramu}

Po převodu signálu na frekvenční reprezentaci následuje u většiny metod některý druh úpravy celého spektrogramu, předcházející samotnému výpočtu \emph{funkce salience}. Výsledkem tohoto kroku může být potlačení šumu a nemelodických částí signálu, zpřesnění informace o výšce znějících frekvencí nebo normalizace či jiná úprava amplitud spektrogramu.

Nejčastější úpravou je nalezení lokálních maxim (vrcholů). Potlačením nemaximálních oblastí se zbavíme velkého množství nemelodických složek signálu, přitom informaci o těch melodických neztratíme. Výhodou práce s množinou vrcholů je, že jejich frekvenci lze na základě spektra dále zpřesnit pomocí parabolické interpolace (\cite{Rao2010}) a nebo využitím úhlové frekvence (\cite{Salamon2012a}, \cite{Dressler2009}). 

Jiným druhem úpravy jsou různé způsoby normalizace, ať jednoduché aplikace logaritmu na jednotlivé hodnoty spektrogramu (\cite{Cancela2008}, \cite{Bittner2017}) nebo složitější strategie normalizace, které se aplikují na celá výstupní okna krátkodobé Fourierovy transformace (spectral whitening, \cite{Ryynanen2008}), či které používají pohyblivé průměry nebo jinou metodu, beroucí v potaz širší zvukový kontext. Cílem normalizace je zvýraznění slabších harmonických frekvencí a potlačení celopásmových zvuků (například perkusí). Principiálně podobným krokem je aplikace pásmového filtru (\cite{Goto1999}) pro zvýraznění frekvencí obsahující melodii. Případně využití psychoakustických filtrů modelující lidské vnímání hlasitosti (\cite{Salamon2012a}, \cite{Ikemiya2016}). Ze signálu lze také oddělit melodické nástroje a perkusivní doprovod pomocí metod separace signálů (source separation). Používanými metodami jsou například Harmonic/Percussive Sound Separation (HPSS) (\cite{Tachibana2010}) nebo Robust principal component analysis (RPCA) (\cite{Ikemiya2016}).

\subsection{Funkce salience}

Salience tónu vyjadřuje míru důležitosti či nápadnosti ke svému hudebnímu okolí. Nejvíce ji ovlivňuje hlasitost v poměru ke zbylým znějícím hlasům, vliv má ale také řada dalších charakteristik hraní. \cite{Dressler2016} mezi příklady uvádí například frekvenční modulaci, jako je vibrato nebo glissando, zejména oproti frekvenčně stálému hudebnímu doprovodu (piano, kytara). Velký vliv má samozřejmě také i barva hlasu. Lidský zpěv nebo obecně zvuky se silnějšími vyššími alikvótními frekvencemi snadněji upoutají pozornost. V případě vícehlasu mají obecně posluchači potíže rozeznat výšky tónů uvnitř souzvuku. Pokud má posluchač přiřadit pociťovanou výšku tónu akordu, obvykle volí nejvyšší či nejnižší ze znějících frekvencí.

Výstupem funkce salience je ohodnocení každé výšky tónu v každém časovém okamžiku nahrávky, které co nejlépe odpovídá v relativních poměrech výše popsané zvukové salienci. Jelikož neexistují žádné studie, které by se zabývaly měřením a kvantifikací salience v hudbě, nelze posoudit, jak dobře výsledky obvyklých způsobů výpočtu funkce salience korelují s mírou, ze které vychází. \cite{Bittner2018a} se však domnívá, že odhad bude velmi hrubý, většina postupů totiž do výpočtu zahrnuje pouze hlasitost hlasu, a tedy vynechává řadu jiných důležitých faktorů, které salienci ovlivňují.

Přístupy k výpočtu by se daly zařadit do tří kategorií - sčítání harmonických frekvencí, odhad parametrů modelujících vstup a metody strojového učení. 

\subsubsection{Sčítání harmonických frekvencí}

Metody založené na sčítání harmonických frekvencí jsou principiálně nejjednodušší skupinou. Vychází z práce \cite{Hermes1988}, jejich podstatou je využití harmonické struktury zvuku tónů. Ohodnocení frekvence $f_\tau$ získáme váženou sumou amplitud všech jejích harmonických frekvencí $h \cdot f_\tau$. Pro spektrogram $X(f, \tau)$ signálu $x$, funkci vah $g(f_\tau, h)$ a $N_h$ počet zahrnutých harmonických frekvencí v sumě:

    $$S_x(f_\tau, \tau) = \sum_{h=1}^{N_h}{g(f_\tau, h)\abs{X(h \cdot f, \tau)}}$$

Pro ilustraci uvažme jednohlasý harmonický signál s fundamentální frekvencí $f^\star$. Hodnoty spektrogramu $X(f, \tau)$ tedy budou vyšší kolem frekvencí $H_{f^\star} = \{1\cdot f^\star, 2\cdot f^\star, 3\cdot f^\star, \dots\}$ a jinde nulové. Funkce $S_x(f, \tau)$ bude tedy pro $f \not\in H_{f^\star}$ nulová a v $f^\star$ bude nabývat globálního maxima. Příklad výstupu této metody pro polyfonní nahrávku je na obrázku \ref{obr:salience}.

\cite{Dressler2011} tuto metodu vylepšuje zpracováváním dvojic vrcholů vstupního spektrogramu, její výsledný salienční spektrogram obsahuje méně kandidátů na fundamentální frekvenci. \cite{Cancela2008} se pokouší zmenšovat chybné hodnoty salienční funkce pomocí vyhodnocení subharmonických frekvencí.

\subsubsection{Statistické modelování signálu}

\textcolor{red}{TODO: obrázek salience pomocí modelů tónů}

Jiným přístupem k počítání \emph{funkce salience}, který používá \cite{Goto1999}, je modelování okna spektrogramu váženým součtem harmonických struktur (modelů tónů). Snažíme se vrcholy ve spektru rozdělit mezi různě silně znějící tóny tak, aby v součtu co nejlépe odpovídaly měřeným intenzitám. Přístup se jinými slovy snaží zjistit, jaké tóny musely v danou chvíli znít, aby vzniklo dané spektrum. 

Vstupem metody je okno normalizovaného spektrogramu $p_X^{(t)}(x)$ v čase $t$. Okno se pokusíme modelovat jako hustotu pravděpodobnosti $p(x; \theta^{(t)})$ vzniklou váženou směsí modelů všech možných tónů melodie v definovaném rozsahu frekvencí v intervalu $[F_l, F_h]$. Hustotu pravděpodobnosti jednoho z tónů s fundamentální frekvencí $F$ označíme jako $p(x|F)$, a jako $w^{(t)}(F)$ označíme váhu, kterou model tónu $p(x|F)$ přispívá do celkové smíšené hustoty pravděpodobnosti. Pak $p(x; \theta^{(t)})$ definujeme jako:

$$p(x; \theta^{(t)}) = \int_{F_l}^{F_h}{w^{(t)}(F)p(x|F) \mathrm{d}F}$$
$$\theta^{(t)} = \{\, w^{(t)}(F) \mid F_l < F < F_h \,\}$$

Cílem pak je nalezení takových parametrů $\theta^{(t)}$, aby model $p(x; \theta^{(t)})$ dobře popisoval pozorovaní $p_X^{(t)}(x)$. K tomu \cite{Goto1999} využívá Expectation-Maximization (EM) algoritmus. Výsledné parametry $\theta^{(t)}$ jsou pak hodnoty salienční funkce.

\textcolor{red}{TODO: popsat Durrieu, jakožto jiný přístup k modelování signálu}

\subsubsection{Metody strojového učení}

\textcolor{red}{TODO}
% musim doprojít všechny existujícíc metdy
% poliner
% - nejsou nejlépe srovnatelné s ostatními salience funkcemi, protože často mají pro výpočet salience větší kontext než jedno FFT okno
% - multicolumn deep neural networks \cite{Kum2016}



\subsection{Hledání melodie}

\textcolor{red}{TODO: tady bude popis tone trackingu různých metod, asi nemusí být tak rozsáhlý, jelikož moje práce se tím nejspíš nestihne zabývat}
\textcolor{red}{TODO: píšu to unavenej,kdyžtak přepsat}

Po výpočtu funkce salience $S_x(f_\tau, \tau)$ máme k dispozici odhady fundamentálních frekvencí v signálu, z těchto ohodnocení pak musíme vybrat výslednou konturu melodie. Triviálním řešením by bylo vybrat frekvence s maximální saliencí pro každé časové okno $\hat{f}_\tau = \argmax_{f_\tau}{S_x(f_\tau, \tau)}$, tento jednoduchý přístup však nevolí mnoho metod. 

% -         - získání melodie
%         - množina pravidel pro výběr z nalezených not
%             - důraz na kontinuitu výšky a síly not v melodii
%             - vymazání velkých skoků, krátkých v délce
%             - preference pro nějaký frekvenční rozsah
%             - výběr nejvyšších nebo nejnižších frekvencí při souzvuku
%         - tracking agents
%             - soupeřící hypotézy, výtězí ta, která nejlépe splňuje stanovené podmínky
%         - HMM
%                     - tracking
%                         - získání melodie z kandidátů v salience function
%                         - zde se algoritmy nejvíce různí - clustering, heuristic-based agents, HMMs, dynamic programming, ...
% PŘEDTIM PROJÍT DRESSLERKU


\subsubsection{Přítomnost melodie (voicing)}
% - některé metody nemusí (když shlukují a odstraňují, tak tam přirozeně vzniknou prázná místa)
% - common approach is to use a fixed or dynamic per-frame salience-based threshold

\section{Srovnání existujících metod}

Pro celkové kvantitativní srovnání metod jsme zpracovali výsledky všech ročníků soutěže MIREX a dostupné, state-of-the-art metody spustili na dalších datech, která do vyhodnocování v soutěži MIREX nejsou zahrnuta. Výsledkem je podrobné srovnání většiny relevantních, existujících metod pro extrakci melodie.

\textcolor{red}{TODO: tabulka metod s popisem architektury}
\textcolor{red}{TODO: tabulka metod s výsledky v mirexu a na mých datech}

%  jde zejména o zohlednění frekvenčních charakteristik melodických signálů a metody se liší například ve volbě druhu spektrální transformace signálu

%  společnou základní strukturu algoritmů pro extrakci melodie.  