\chapter{Související práce}

\section{Definice melodie}


Jelikož z muzikologického hlediska žádná jasná a obecně přijímaná definice melodie neexistuje a ve výsledku melodie zůstává pro každého posluchače ryze subjektivním pojmem, pro extrakci melodie si výzkumné týmy volí spíše pragmaticky takové definice, se kterými se nejlépe pracuje. Příkladem může být jedna z prvních prací zabývající se extrakcí melodie. Výstupem v práci \cite{Goto1999} je kontura fundamentální frekvence sestávající se z nejsilnějších tónů hrajících v omezeném frekvenčním rozsahu. Tato definice je poměrně úzká, tóny melodie se totiž jistě mohou vyskytovat i mimo autory specifikovaný rozsah a nemusí být vždy v poměru s doprovodem nejsilnější složkou signálu. Z technického hlediska však dovolila autorům implementaci algoritmu běžícího v reálném čase, který poskytoval sémanticky bohatý popis vstupních nahrávek. Navazující články pracují s volnějšími definicemi, které lépe reflektují podstatu melodie. Mimo to se používaná definice proměňuje díky novým datasetům, jejichž autoři tvoří protipól k ryze technickým a objektivním cílům algoritmických metod. Zatímco pro tvorbu algoritmů je praktické zvolit co nejkonkrétnější cíl, při tvorbě datasetu se naopak projevuje lidská subjektivita autorů anotací. 

Kompromisem mezi subjektivní a praktickou definicí se na dlouhou dobu stala "extrakce základní frekvence hlavního melodického hlasu". Ačkoli melodii v reálném hudebním materiálu obvykle nese více hlasů, které se v hraní střídají (například píseň se zpěvem a kytarovým sólem), v letech 2005 - 2015 se v soutěži MIREX provádí evaluace pouze nad krátkými výňatky, kde tato definice není omezující. Tento pohled však otevírá také jiné přístupy, například extrakci melodie pomocí modelování hudebního záznamu jako součtu signálu jednoho hlasu a doprovodu \citep{Durrieu2010}, \citep{Bosch2016b} nebo přímo omezení se na separaci lidského zpěvu a doprovodu \citep{Ikemiya2016}. Nově se objevují práce, které "hlavní" melodický hlas neinterpretují nutně jako "nejsilnější". Skladatelé a hráči používají množství různých postupů, které melodii zvýrazňují - krom dynamiky ji ovlivňuje například také barva hlasu, vibrato nebo délka not. \cite{Salamon2012a} využívá těchto rysů pro výběr mezi kandidáty na melodickou konturu.

Posunem v rámci MIR komunity bylo zveřejnění nových datasetů MedleyDB \citep{Bittner2014} a ORCHSET \citep{Bosch2016}, oba přináší nová data, ve kterých již melodii nenese pouze jeden hlas po celou dobu skladby. V porovnání s do té doby dostupnými daty jde o mnohem rozmanitější kolekce. V případě MedleyDB jde o první volně dostupný dataset, ve kterém se objevují celé skladby, nikoli pouze výňatky a autoři předkládají rovnou tři verze anotací:

\begin{enumerate}
    \item Základní frekvence nejvýraznějšího melodického hlasu, jehož zdroj zůstává po dobu nahrávky neměnný.
    \item Základní frekvence nejvýraznějšího melodického hlasu, jehož zdroje se mohou měnit.
    \item Základní frekvence všech melodických hlasů, potenciálně pocházejících z více zdrojů.
\end{enumerate}

První formulace je v souladu s doposud používanou definicí. Zbylé dvě se snaží posouvat možné cíle budoucích metod a předložit komunitě nové výzvy, podle \cite{Salamon2014} totiž výzkum začal v letech 2009-2012 stagnovat. Zatímco anotace s jednou melodickou linkou (1. a 2. definice) se v navazujících pracích často používají, zatím žádný článek se nepokusil představit metodu, jejímž cílem by bylo extrahovat více melodických linek (3. definice).

\cite{Bosch2016} při práci na datasetu ORCHSET vychází z článku \cite{Poliner2007}, který definuje melodii jako "jednohlasou sekvenci tónů, kterou bude posluchač nejspíše reprodukovat, pokud jej požádáme o zapískání či zabroukání příslušné skladby". Přestože nejde o objektivní definici, v praxi se posluchači často na jedné konkrétní sekvenci tónů shodnou, a to jak u populární hudby, kde melodii často nese lidský zpěv, tak u orchestrálních skladeb. Ačkoli se definice neujala pro metody extrakce, \cite{Bosch2016} ji využili pro anotaci výňatků z orchestrálních skladeb, u kterých by předchozí zmíněné definice selhávaly, jelikož pojem melodie je u orchestrální hudby mnohdy komplikovanější než u jiných žánrů. Anotace tak spočívala v přezpívání orchestrálních výňatků skupinou posluchačů a následném srovnání a zpracování těchto nahrávek.