\chapter*{Úvod}
\addcontentsline{toc}{chapter}{Úvod}

Spolu s harmonií a rytmem představuje melodie základní kámen většiny známé hudby. V průběhu vývoje od folklórních zpěvů přes orchestrální skladby po soudobou elektroniku si melodie téměř vždy zachovávala své dominantní postavení nositele esence jednotlivých písní. Melodie je to hlavní, co si člověk po poslechu skladby odnáší a nejsnadněji vybaví, a její důležitost je zejména v našem kulturním kontextu natolik jednoznačná, že je občas těžké si bez ní vůbec hudbu představit. 

Tato práce se zabývá metodami přepisu melodických kontur ze zvukové nahrávky. Jde o jednu z nejdůležitějších a zároveň nejtěžších úloh z oboru \textit{Music Information Retrieval}, jejíž rozsah využití v této doméně pokrývá významnou část aktivně řešených, otevřených problémů. Spolehlivý přepis melodie by usnadnil vyhledávání v hudebních datech, ať už na základě notového zápisu (\textit{Symbolic Melodic Similarity}), pomocí nekvalitní nahrávky z rádia (\textit{Audio Fingerprinting}), pomocí broukání (\textit{Query by Singing/Humming}) nebo dokonce pomocí coveru hledané písně (\textit{Audio Cover Song Identification}). Mimo vyhledávání by byl algoritmus užitečný pro další zpracování zvukového signálu, ať už pro manipulaci a úpravu melodického hlasu (například software Melodyne), nebo naopak jeho odstranění a vytvoření karaoke doprovodu (\textit{Informed Source Separation}). V neposlední řadě by extrakce melodie pomohla při kategorizaci hudebních dat, například podle žánru (\cite{Salamon2012}) nebo podle zpěváka (\textit{Singer Characterization}). A konečně široké spektrum využití by se nalezlo i v muzikologii (případně etnomuzikologii) pro statistickou i kvalitativní studii hudebních motivů a postupů (V jazzu například \cite{Pfleiderer}).

Příkladem použití ale může být i pomoc při transkripci. Představíme-li si začínajícího hráče na saxofon, který si chce do not přepsat svoje oblíbené jazzové sólo, aby se ho mohl naučit, výstup algoritmu pro přepis melodie mu dá užitečnou informaci o tom, jaký tón zní v jakou chvíli. Z této reprezentace už pak hráči zbývá nalezené tóny projít a zapsat je do notové osnovy ve správných délkách.