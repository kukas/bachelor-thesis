\chapter*{Úvod}
\addcontentsline{toc}{chapter}{Úvod}

Spolu s harmonií a rytmem představuje melodie základní stavební kámen většiny existující hudby. V průběhu vývoje od folklórních zpěvů přes orchestrální skladby po soudobou elektroniku si melodie téměř vždy zachovávala své dominantní postavení nositele esence jednotlivých písní. Melodie je to hlavní, co si člověk po poslechu skladby odnáší a nejsnadněji vybaví, a její důležitost je zejména v našem kulturním kontextu natolik jednoznačná, že je občas těžké si bez ní vůbec hudbu představit. 

Tato práce se zabývá metodami přepisu melodických kontur ze zvukové nahrávky. Jde o jednu z nejdůležitějších a zároveň nejtěžších úloh z oboru \textit{Music Information Retrieval}, jejíž rozsah využití v této doméně pokrývá významnou část aktivně řešených, otevřených problémů. Spolehlivý přepis melodie by usnadnil vyhledávání v hudebních datech, ať už na základě notového zápisu (\textit{Symbolic Melodic Similarity}), pomocí nekvalitní nahrávky z rádia (\textit{Audio Fingerprinting}), pomocí broukání (\textit{Query by Singing/Humming}) nebo dokonce pomocí coveru hledané písně (\textit{Audio Cover Song Identification}). Mimo vyhledávání by byl algoritmus užitečný pro další zpracování zvukového signálu, ať už pro manipulaci a úpravu melodického hlasu (například software Melodyne), nebo naopak jeho odstranění a vytvoření karaoke doprovodu (\textit{Informed Source Separation}). V neposlední řadě by extrakce melodie pomohla při kategorizaci hudebních dat, například podle žánru (\cite{Salamon2012}) nebo podle zpěváka (\textit{Singer Characterization}). A konečně široké spektrum využití by se nalezlo i v muzikologii (případně etnomuzikologii) pro statistickou i kvalitativní studii hudebních motivů a postupů (V jazzu například \cite{Pfleiderer}).

Příkladem použití ale může být i pomoc při transkripci. Představíme-li si začínajícího hráče na saxofon, který si chce do not přepsat svoje oblíbené jazzové sólo, aby se ho mohl naučit, výstup algoritmu pro přepis melodie mu dá užitečnou informaci o tom, jaký tón zní v jakou chvíli. Z této reprezentace už pak hráči zbývá nalezené tóny projít a zapsat je do notové osnovy ve správných délkách.


\section{Definice melodie}

Rozpoznání melodie hrající skladby je pro většinu posluchačů intuitivní schopností, která je součástí prožitku poslechu hudby, a která jejímu poslechu vůbec dává smysl. Slyšet hudbu a nevnímat melodii je podobné jako poslouchat řeč a nerozumět větám. Ačkoli je melodie tedy termín, který je subjektivně jasný, formální, obecně přijímanou muzikologickou definici, která by se zpětně neodkazovala k posluchači, nemá. 

Z tohoto důvodu si výzkumné týmy zabývající se automatickou transkripcí melodie volí pragmaticky spíše užší definice melodie, se kterými se v jejich kontextu lépe pracuje. Práce \cite{Goto1999}, která je považována za jednu z prvních a důležitých prací v oboru, chápe melodii jako konturu fundamentální frekvence sestávající se z nejsilnějších tónů hrajících v omezeném frekvenčním rozsahu. Tato definice je poměrně úzká, tóny melodie se totiž jistě mohou vyskytovat i mimo autory specifikovaný rozsah a nemusí být vždy v poměru s doprovodem nejsilnější složkou signálu. Z technického hlediska však umožnila autorům implementaci algoritmu běžícího v reálném čase, který poskytoval sémanticky bohatý popis vstupních nahrávek. Navazující články pracují s volnějšími definicemi, které lépe reflektují podstatu melodie. Mimo to se používaná definice proměňuje díky novým datasetům, jejichž autoři tvoří protipól k ryze technickým a objektivním cílům algoritmických metod. Zatímco pro tvorbu algoritmů je praktické zvolit co nejkonkrétnější cíl, při tvorbě datasetu se naopak projevuje lidská subjektivita autorů anotací. 

Kompromisem mezi subjektivní a praktickou definicí se na dlouhou dobu stala \uv{extrakce základní frekvence hlavního melodického hlasu}. Ačkoli melodii v reálném hudebním materiálu obvykle nese více hlasů, které se v hraní střídají (například píseň se zpěvem a kytarovým sólem), v letech 2005 -- 2015 se v soutěži MIREX provádí evaluace pouze nad krátkými výňatky, kde tato definice není omezující. Tento pohled však otevírá také jiné přístupy, například extrakci melodie pomocí modelování hudebního záznamu jako součtu signálu jednoho hlasu a doprovodu \citep{Durrieu2010}, \citep{Bosch2016b} nebo přímo omezení se na separaci lidského zpěvu a doprovodu \citep{Ikemiya2016}. Nově se objevují práce, které \uv{hlavní} melodický hlas neinterpretují nutně jako \uv{nejsilnější}. Skladatelé a hráči používají množství různých postupů, které melodii zvýrazňují --- krom dynamiky ji ovlivňuje například také barva hlasu, vibrato nebo délka not. \cite{Salamon2012a} využívá těchto rysů pro výběr mezi kandidáty na melodickou konturu.

Posunem v rámci MIR komunity bylo zveřejnění nových datasetů MedleyDB \citep{Bittner2014} a ORCHSET \citep{Bosch2016}, oba přináší nová data, ve kterých již melodii nenese pouze jeden hlas po celou dobu skladby. V porovnání s do té doby dostupnými daty jde o mnohem rozmanitější kolekce. V případě MedleyDB jde o první volně dostupný dataset, ve kterém se objevují celé skladby, nikoli pouze výňatky a autoři předkládají rovnou tři verze anotací:

\begin{enumerate}
    \item Základní frekvence nejvýraznějšího melodického hlasu, jehož zdroj zůstává po dobu nahrávky neměnný.
    \item Základní frekvence nejvýraznějšího melodického hlasu, jehož zdroje se mohou měnit.
    \item Základní frekvence všech melodických hlasů, potenciálně pocházejících z více zdrojů.
\end{enumerate}

První formulace je v souladu s doposud používanou definicí. Zbylé dvě se snaží posouvat možné cíle budoucích metod a předložit komunitě nové výzvy, podle \cite{Salamon2014} totiž výzkum začal v letech 2009--2012 stagnovat. Zatímco anotace s jednou melodickou linkou (1. a 2. definice) se v navazujících pracích často používají, zatím žádný článek se nepokusil představit metodu, jejímž cílem by bylo extrahovat více melodických linek (3. definice).

\cite{Bosch2016} při práci na datasetu ORCHSET vychází z článku \cite{Poliner2007}, který definuje melodii jako \uv{jednohlasou sekvenci tónů, kterou bude posluchač nejspíše reprodukovat, pokud jej požádáme o zapískání či zabroukání příslušné skladby}. Přestože nejde o objektivní definici, v praxi se posluchači často na jedné konkrétní sekvenci tónů shodnou, a to jak u populární hudby, kde melodii často nese lidský zpěv, tak u orchestrálních skladeb. Ačkoli se definice neujala pro metody extrakce, \cite{Bosch2016} ji využili pro anotaci výňatků z orchestrálních skladeb, u kterých by předchozí zmíněné definice selhávaly, jelikož pojem melodie je u orchestrální hudby mnohdy komplikovanější než u jiných žánrů. Anotace tak spočívala v přezpívání orchestrálních výňatků skupinou posluchačů a následném srovnání a zpracování těchto nahrávek.
